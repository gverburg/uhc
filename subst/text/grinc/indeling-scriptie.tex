\documentclass{book}

\title{Exceptional GRIN}
\author{Christof Douma}

\newcommand{\todo}[1]{\emph{\textbf{todo:}#1}}
\newcommand{\note}[1]{\emph{\textbf{note:}#1}

}

\newcommand{\cmm}{C\texttt{-\/-}}
\newcommand{\qcmm}{qc\texttt{-\/-}}


\begin{document}

\maketitle{}

\note{Where to put Exceptions. Do I start without, or do I put them into grin part, c-- part etc.
I guess that I just add it as I explain GRIN}

\chapter{Introduction}

\begin{itemize}
	\item background
		\subitem Lazy functional languages
		\subitem graph reduction
		\subitem other compiler back ends
	\item motivation
	\item overview of thesis?
	\item research contributions
\end{itemize}

\chapter{GRIN}

\section{Introduction}

\begin{itemize}
	\item strict functional language
	\item functions are not first order
	\item build in monad
\end{itemize}

\section{Syntax}

\section{Semantics}

\section{Compiling to GRIN}

\begin{itemize}
	\item evaluating suspensions
	\item partial evaluation
	\item difference between apply and the others
	\item whole program model, optimized agressively
	\item exception primitives
\end{itemize}

\chapter{Heap-points-to analysis}

\section{Introduction}

Data flow in grin is control flow in the original language. Heap points to analysis.

\begin{itemize}
	\item example result
	\item flow insensitive
	\item usages of the result
\end{itemize}

\section{Deriving equations}

\begin{itemize}
	\item preparing grin for the analysis
	\item types of equations
	\item flow insensitive
	\item higher order functions
\end{itemize}

\section{Solving equations}

Note: this is a rather small section... might put is somewhere else?
\begin{itemize}
	\item fixpoint computation
\end{itemize}

\section{Possible improvements}

\begin{itemize}
	\item dependency generation
	\item sharing analysis
	\item making lazy apply calls call sensitive
\end{itemize}

\chapter{Transformations} % probably I want to use a different name and different setup for this part. Or don't I...

\section{Simplifying}

\subsection{inline eval}

done

\subsection{inline apply}

done

\subsection{Vectoristation}

done

\subsection{Case simplification}

done

\section{Optimizing}

\subsection{Copy propagation}

done

\subsection{Trivial case elimination}

done

\subsection{Evaluated case elimination}

done

\subsection{Sparse case elimination}

done
\subsection{Dead code elimination}

I only do dead procedure elimination at the moment. CAFs are always retained.

\chapter{\cmm}

\section{Introduction}

\begin{itemize}
	\item what is it - portable assembly
	\item what does it for us
\end{itemize}

Note: it is not realy important, only code generation depends on it, all the
grin transformations are independent of this.

\section{Overview}

Short introduction to the concepts of \cmm.

\chapter{Code generation}

\begin{itemize}
	\item grin input form
	\item mapping between grin concepts and \cmm concepts.
\end{itemize}

\chapter{Conclusions}

\begin{itemize}
	\item Results
	\item Future work
\end{itemize}


\end{document}
