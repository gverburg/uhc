\documentclass[b5paper]{book}
\usepackage{hyperref}
\usepackage{makeidx}
\usepackage{amssymb}
\usepackage{color}
\usepackage{alltt}
\usepackage{graphicx}
\usepackage{layout}
\def\union{\cup}
\def\intersect{\cap}
\def\getsrandom{\stackrel{\rm R}{\gets}}
\def\cross{\times}
\def\cat{\hspace{0.5em} \| \hspace{0.5em}}
\def\catn{$\|$}
\def\divides{\hspace{0.3em} | \hspace{0.3em}}
\def\nequiv{\not\equiv}
\def\approx{\raisebox{0.2ex}{\mbox{\small $\sim$}}}
\def\lcm{{\rm lcm}}
\def\gcd{{\rm gcd}}
\def\log{{\rm log}}
\def\ord{{\rm ord}}
\def\abs{{\mathit abs}}
\def\rep{{\mathit rep}}
\def\mod{{\mathit\ mod\ }}
\renewcommand{\pmod}[1]{\ ({\rm mod\ }{#1})}
\newcommand{\floor}[1]{\left\lfloor{#1}\right\rfloor}
\newcommand{\ceil}[1]{\left\lceil{#1}\right\rceil}
\def\Or{{\rm\ or\ }}
\def\And{{\rm\ and\ }}
\def\iff{\hspace{1em}\Longleftrightarrow\hspace{1em}}
\def\implies{\Rightarrow}
\def\undefined{{\rm ``undefined"}}
\def\Proof{\vspace{1ex}\noindent {\bf Proof:}\hspace{1em}}
\let\oldphi\phi
\def\phi{\varphi}
\def\Pr{{\rm Pr}}
\newcommand{\str}[1]{{\mathbf{#1}}}
\def\F{{\mathbb F}}
\def\N{{\mathbb N}}
\def\Z{{\mathbb Z}}
\def\R{{\mathbb R}}
\def\C{{\mathbb C}}
\def\Q{{\mathbb Q}}
\definecolor{DGray}{gray}{0.5}
\newcommand{\emailaddr}[1]{\mbox{$<${#1}$>$}}
\def\twiddle{\raisebox{0.3ex}{\mbox{\tiny $\sim$}}}
\def\gap{\vspace{0.5ex}}
\makeindex
\begin{document}
\frontmatter
\pagestyle{empty}
\title{LibTomMath User Manual \\ v0.39}
\author{Tom St Denis \\ tomstdenis@iahu.ca}
\maketitle
This text, the library and the accompanying textbook are all hereby placed in the public domain.  This book has been 
formatted for B5 [176x250] paper using the \LaTeX{} {\em book} macro package.

\vspace{10cm}

\begin{flushright}Open Source.  Open Academia.  Open Minds.

\mbox{ }

Tom St Denis,

Ontario, Canada
\end{flushright}

\tableofcontents
\listoffigures
\mainmatter
\pagestyle{headings}
\chapter{Introduction}
\section{What is LibTomMath?}
LibTomMath is a library of source code which provides a series of efficient and carefully written functions for manipulating
large integer numbers.  It was written in portable ISO C source code so that it will build on any platform with a conforming
C compiler.  

In a nutshell the library was written from scratch with verbose comments to help instruct computer science students how
to implement ``bignum'' math.  However, the resulting code has proven to be very useful.  It has been used by numerous 
universities, commercial and open source software developers.  It has been used on a variety of platforms ranging from
Linux and Windows based x86 to ARM based Gameboys and PPC based MacOS machines.  

\section{License}
As of the v0.25 the library source code has been placed in the public domain with every new release.  As of the v0.28
release the textbook ``Implementing Multiple Precision Arithmetic'' has been placed in the public domain with every new
release as well.  This textbook is meant to compliment the project by providing a more solid walkthrough of the development
algorithms used in the library.

Since both\footnote{Note that the MPI files under mtest/ are copyrighted by Michael Fromberger.  They are not required to use LibTomMath.} are in the 
public domain everyone is entitled to do with them as they see fit.

\section{Building LibTomMath}

LibTomMath is meant to be very ``GCC friendly'' as it comes with a makefile well suited for GCC.  However, the library will
also build in MSVC, Borland C out of the box.  For any other ISO C compiler a makefile will have to be made by the end
developer.  

\subsection{Static Libraries}
To build as a static library for GCC issue the following
\begin{alltt}
make
\end{alltt}

command.  This will build the library and archive the object files in ``libtommath.a''.  Now you link against 
that and include ``tommath.h'' within your programs.  Alternatively to build with MSVC issue the following
\begin{alltt}
nmake -f makefile.msvc
\end{alltt}

This will build the library and archive the object files in ``tommath.lib''.  This has been tested with MSVC 
version 6.00 with service pack 5.  

\subsection{Shared Libraries}
To build as a shared library for GCC issue the following
\begin{alltt}
make -f makefile.shared
\end{alltt}
This requires the ``libtool'' package (common on most Linux/BSD systems).  It will build LibTomMath as both shared
and static then install (by default) into /usr/lib as well as install the header files in /usr/include.  The shared 
library (resource) will be called ``libtommath.la'' while the static library called ``libtommath.a''.  Generally 
you use libtool to link your application against the shared object.  

There is limited support for making a ``DLL'' in windows via the ``makefile.cygwin\_dll'' makefile.  It requires 
Cygwin to work with since it requires the auto-export/import functionality.  The resulting DLL and import library 
``libtommath.dll.a'' can be used to link LibTomMath dynamically to any Windows program using Cygwin.

\subsection{Testing}
To build the library and the test harness type

\begin{alltt}
make test
\end{alltt}

This will build the library, ``test'' and ``mtest/mtest''.  The ``test'' program will accept test vectors and verify the
results.  ``mtest/mtest'' will generate test vectors using the MPI library by Michael Fromberger\footnote{A copy of MPI
is included in the package}.  Simply pipe mtest into test using

\begin{alltt}
mtest/mtest | test
\end{alltt}

If you do not have a ``/dev/urandom'' style RNG source you will have to write your own PRNG and simply pipe that into 
mtest.  For example, if your PRNG program is called ``myprng'' simply invoke

\begin{alltt}
myprng | mtest/mtest | test
\end{alltt}

This will output a row of numbers that are increasing.  Each column is a different test (such as addition, multiplication, etc)
that is being performed.  The numbers represent how many times the test was invoked.  If an error is detected the program
will exit with a dump of the relevent numbers it was working with.

\section{Build Configuration}
LibTomMath can configured at build time in three phases we shall call ``depends'', ``tweaks'' and ``trims''.  
Each phase changes how the library is built and they are applied one after another respectively.  

To make the system more powerful you can tweak the build process.  Classes are defined in the file
``tommath\_superclass.h''.  By default, the symbol ``LTM\_ALL'' shall be defined which simply 
instructs the system to build all of the functions.  This is how LibTomMath used to be packaged.  This will give you 
access to every function LibTomMath offers.

However, there are cases where such a build is not optional.  For instance, you want to perform RSA operations.  You 
don't need the vast majority of the library to perform these operations.  Aside from LTM\_ALL there is 
another pre--defined class ``SC\_RSA\_1'' which works in conjunction with the RSA from LibTomCrypt.  Additional 
classes can be defined base on the need of the user.

\subsection{Build Depends}
In the file tommath\_class.h you will see a large list of C ``defines'' followed by a series of ``ifdefs''
which further define symbols.  All of the symbols (technically they're macros $\ldots$) represent a given C source
file.  For instance, BN\_MP\_ADD\_C represents the file ``bn\_mp\_add.c''.  When a define has been enabled the
function in the respective file will be compiled and linked into the library.  Accordingly when the define
is absent the file will not be compiled and not contribute any size to the library.

You will also note that the header tommath\_class.h is actually recursively included (it includes itself twice).  
This is to help resolve as many dependencies as possible.  In the last pass the symbol LTM\_LAST will be defined.  
This is useful for ``trims''.

\subsection{Build Tweaks}
A tweak is an algorithm ``alternative''.  For example, to provide tradeoffs (usually between size and space).
They can be enabled at any pass of the configuration phase.

\begin{small}
\begin{center}
\begin{tabular}{|l|l|}
\hline \textbf{Define} & \textbf{Purpose} \\
\hline BN\_MP\_DIV\_SMALL & Enables a slower, smaller and equally \\
                          & functional mp\_div() function \\
\hline
\end{tabular}
\end{center}
\end{small}

\subsection{Build Trims}
A trim is a manner of removing functionality from a function that is not required.  For instance, to perform
RSA cryptography you only require exponentiation with odd moduli so even moduli support can be safely removed.  
Build trims are meant to be defined on the last pass of the configuration which means they are to be defined
only if LTM\_LAST has been defined.

\subsubsection{Moduli Related}
\begin{small}
\begin{center}
\begin{tabular}{|l|l|}
\hline \textbf{Restriction} & \textbf{Undefine} \\
\hline Exponentiation with odd moduli only & BN\_S\_MP\_EXPTMOD\_C \\
                                           & BN\_MP\_REDUCE\_C \\
                                           & BN\_MP\_REDUCE\_SETUP\_C \\
                                           & BN\_S\_MP\_MUL\_HIGH\_DIGS\_C \\
                                           & BN\_FAST\_S\_MP\_MUL\_HIGH\_DIGS\_C \\
\hline Exponentiation with random odd moduli & (The above plus the following) \\
                                           & BN\_MP\_REDUCE\_2K\_C \\
                                           & BN\_MP\_REDUCE\_2K\_SETUP\_C \\
                                           & BN\_MP\_REDUCE\_IS\_2K\_C \\
                                           & BN\_MP\_DR\_IS\_MODULUS\_C \\
                                           & BN\_MP\_DR\_REDUCE\_C \\
                                           & BN\_MP\_DR\_SETUP\_C \\
\hline Modular inverse odd moduli only     & BN\_MP\_INVMOD\_SLOW\_C \\
\hline Modular inverse (both, smaller/slower) & BN\_FAST\_MP\_INVMOD\_C \\
\hline
\end{tabular}
\end{center}
\end{small}

\subsubsection{Operand Size Related}
\begin{small}
\begin{center}
\begin{tabular}{|l|l|}
\hline \textbf{Restriction} & \textbf{Undefine} \\
\hline Moduli $\le 2560$ bits              & BN\_MP\_MONTGOMERY\_REDUCE\_C \\
                                           & BN\_S\_MP\_MUL\_DIGS\_C \\
                                           & BN\_S\_MP\_MUL\_HIGH\_DIGS\_C \\
                                           & BN\_S\_MP\_SQR\_C \\
\hline Polynomial Schmolynomial            & BN\_MP\_KARATSUBA\_MUL\_C \\
                                           & BN\_MP\_KARATSUBA\_SQR\_C \\
                                           & BN\_MP\_TOOM\_MUL\_C \\ 
                                           & BN\_MP\_TOOM\_SQR\_C \\

\hline
\end{tabular}
\end{center}
\end{small}


\section{Purpose of LibTomMath}
Unlike  GNU MP (GMP) Library, LIP, OpenSSL or various other commercial kits (Miracl), LibTomMath was not written with 
bleeding edge performance in mind.  First and foremost LibTomMath was written to be entirely open.  Not only is the 
source code public domain (unlike various other GPL/etc licensed code), not only is the code freely downloadable but the
source code is also accessible for computer science students attempting to learn ``BigNum'' or multiple precision
arithmetic techniques. 

LibTomMath was written to be an instructive collection of source code.  This is why there are many comments, only one
function per source file and often I use a ``middle-road'' approach where I don't cut corners for an extra 2\% speed
increase.

Source code alone cannot really teach how the algorithms work which is why I also wrote a textbook that accompanies
the library (beat that!).

So you may be thinking ``should I use LibTomMath?'' and the answer is a definite maybe.  Let me tabulate what I think
are the pros and cons of LibTomMath by comparing it to the math routines from GnuPG\footnote{GnuPG v1.2.3 versus LibTomMath v0.28}.

\newpage\begin{figure}[here]
\begin{small}
\begin{center}
\begin{tabular}{|l|c|c|l|}
\hline \textbf{Criteria} & \textbf{Pro} & \textbf{Con} & \textbf{Notes} \\
\hline Few lines of code per file & X & & GnuPG $ = 300.9$, LibTomMath  $ = 71.97$ \\
\hline Commented function prototypes & X && GnuPG function names are cryptic. \\
\hline Speed && X & LibTomMath is slower.  \\
\hline Totally free & X & & GPL has unfavourable restrictions.\\
\hline Large function base & X & & GnuPG is barebones. \\
\hline Five modular reduction algorithms & X & & Faster modular exponentiation for a variety of moduli. \\
\hline Portable & X & & GnuPG requires configuration to build. \\
\hline
\end{tabular}
\end{center}
\end{small}
\caption{LibTomMath Valuation}
\end{figure}

It may seem odd to compare LibTomMath to GnuPG since the math in GnuPG is only a small portion of the entire application. 
However, LibTomMath was written with cryptography in mind.  It provides essentially all of the functions a cryptosystem
would require when working with large integers.  

So it may feel tempting to just rip the math code out of GnuPG (or GnuMP where it was taken from originally) in your
own application but I think there are reasons not to.  While LibTomMath is slower than libraries such as GnuMP it is
not normally significantly slower.  On x86 machines the difference is normally a factor of two when performing modular
exponentiations.  It depends largely on the processor, compiler and the moduli being used.

Essentially the only time you wouldn't use LibTomMath is when blazing speed is the primary concern.  However,
on the other side of the coin LibTomMath offers you a totally free (public domain) well structured math library
that is very flexible, complete and performs well in resource contrained environments.  Fast RSA for example can
be performed with as little as 8KB of ram for data (again depending on build options).  

\chapter{Getting Started with LibTomMath}
\section{Building Programs}
In order to use LibTomMath you must include ``tommath.h'' and link against the appropriate library file (typically 
libtommath.a).  There is no library initialization required and the entire library is thread safe.

\section{Return Codes}
There are three possible return codes a function may return.

\index{MP\_OKAY}\index{MP\_YES}\index{MP\_NO}\index{MP\_VAL}\index{MP\_MEM}
\begin{figure}[here!]
\begin{center}
\begin{small}
\begin{tabular}{|l|l|}
\hline \textbf{Code} & \textbf{Meaning} \\
\hline MP\_OKAY & The function succeeded. \\
\hline MP\_VAL  & The function input was invalid. \\
\hline MP\_MEM  & Heap memory exhausted. \\
\hline &\\
\hline MP\_YES  & Response is yes. \\
\hline MP\_NO   & Response is no. \\
\hline
\end{tabular}
\end{small}
\end{center}
\caption{Return Codes}
\end{figure}

The last two codes listed are not actually ``return'ed'' by a function.  They are placed in an integer (the caller must
provide the address of an integer it can store to) which the caller can access.  To convert one of the three return codes
to a string use the following function.

\index{mp\_error\_to\_string}
\begin{alltt}
char *mp_error_to_string(int code);
\end{alltt}

This will return a pointer to a string which describes the given error code.  It will not work for the return codes 
MP\_YES and MP\_NO.  

\section{Data Types}
The basic ``multiple precision integer'' type is known as the ``mp\_int'' within LibTomMath.  This data type is used to
organize all of the data required to manipulate the integer it represents.  Within LibTomMath it has been prototyped
as the following.

\index{mp\_int}
\begin{alltt}
typedef struct  \{
    int used, alloc, sign;
    mp_digit *dp;
\} mp_int;
\end{alltt}

Where ``mp\_digit'' is a data type that represents individual digits of the integer.  By default, an mp\_digit is the
ISO C ``unsigned long'' data type and each digit is $28-$bits long.  The mp\_digit type can be configured to suit other
platforms by defining the appropriate macros.  

All LTM functions that use the mp\_int type will expect a pointer to mp\_int structure.  You must allocate memory to
hold the structure itself by yourself (whether off stack or heap it doesn't matter).  The very first thing that must be
done to use an mp\_int is that it must be initialized.

\section{Function Organization}

The arithmetic functions of the library are all organized to have the same style prototype.  That is source operands
are passed on the left and the destination is on the right.  For instance,

\begin{alltt}
mp_add(&a, &b, &c);       /* c = a + b */
mp_mul(&a, &a, &c);       /* c = a * a */
mp_div(&a, &b, &c, &d);   /* c = [a/b], d = a mod b */
\end{alltt}

Another feature of the way the functions have been implemented is that source operands can be destination operands as well.
For instance,

\begin{alltt}
mp_add(&a, &b, &b);       /* b = a + b */
mp_div(&a, &b, &a, &c);   /* a = [a/b], c = a mod b */
\end{alltt}

This allows operands to be re-used which can make programming simpler.

\section{Initialization}
\subsection{Single Initialization}
A single mp\_int can be initialized with the ``mp\_init'' function. 

\index{mp\_init}
\begin{alltt}
int mp_init (mp_int * a);
\end{alltt}

This function expects a pointer to an mp\_int structure and will initialize the members of the structure so the mp\_int
represents the default integer which is zero.  If the functions returns MP\_OKAY then the mp\_int is ready to be used
by the other LibTomMath functions.

\begin{small} \begin{alltt}
int main(void)
\{
   mp_int number;
   int result;

   if ((result = mp_init(&number)) != MP_OKAY) \{
      printf("Error initializing the number.  \%s", 
             mp_error_to_string(result));
      return EXIT_FAILURE;
   \}
 
   /* use the number */

   return EXIT_SUCCESS;
\}
\end{alltt} \end{small}

\subsection{Single Free}
When you are finished with an mp\_int it is ideal to return the heap it used back to the system.  The following function 
provides this functionality.

\index{mp\_clear}
\begin{alltt}
void mp_clear (mp_int * a);
\end{alltt}

The function expects a pointer to a previously initialized mp\_int structure and frees the heap it uses.  It sets the 
pointer\footnote{The ``dp'' member.} within the mp\_int to \textbf{NULL} which is used to prevent double free situations. 
Is is legal to call mp\_clear() twice on the same mp\_int in a row.  

\begin{small} \begin{alltt}
int main(void)
\{
   mp_int number;
   int result;

   if ((result = mp_init(&number)) != MP_OKAY) \{
      printf("Error initializing the number.  \%s", 
             mp_error_to_string(result));
      return EXIT_FAILURE;
   \}
 
   /* use the number */

   /* We're done with it. */
   mp_clear(&number);

   return EXIT_SUCCESS;
\}
\end{alltt} \end{small}

\subsection{Multiple Initializations}
Certain algorithms require more than one large integer.  In these instances it is ideal to initialize all of the mp\_int
variables in an ``all or nothing'' fashion.  That is, they are either all initialized successfully or they are all
not initialized.

The  mp\_init\_multi() function provides this functionality.

\index{mp\_init\_multi} \index{mp\_clear\_multi}
\begin{alltt}
int mp_init_multi(mp_int *mp, ...);
\end{alltt}

It accepts a \textbf{NULL} terminated list of pointers to mp\_int structures.  It will attempt to initialize them all
at once.  If the function returns MP\_OKAY then all of the mp\_int variables are ready to use, otherwise none of them
are available for use.  A complementary mp\_clear\_multi() function allows multiple mp\_int variables to be free'd 
from the heap at the same time.  

\begin{small} \begin{alltt}
int main(void)
\{
   mp_int num1, num2, num3;
   int result;

   if ((result = mp_init_multi(&num1, 
                               &num2,
                               &num3, NULL)) != MP\_OKAY) \{      
      printf("Error initializing the numbers.  \%s", 
             mp_error_to_string(result));
      return EXIT_FAILURE;
   \}
 
   /* use the numbers */

   /* We're done with them. */
   mp_clear_multi(&num1, &num2, &num3, NULL);

   return EXIT_SUCCESS;
\}
\end{alltt} \end{small}

\subsection{Other Initializers}
To initialized and make a copy of an mp\_int the mp\_init\_copy() function has been provided.  

\index{mp\_init\_copy}
\begin{alltt}
int mp_init_copy (mp_int * a, mp_int * b);
\end{alltt}

This function will initialize $a$ and make it a copy of $b$ if all goes well.

\begin{small} \begin{alltt}
int main(void)
\{
   mp_int num1, num2;
   int result;

   /* initialize and do work on num1 ... */

   /* We want a copy of num1 in num2 now */
   if ((result = mp_init_copy(&num2, &num1)) != MP_OKAY) \{
     printf("Error initializing the copy.  \%s", 
             mp_error_to_string(result));
      return EXIT_FAILURE;
   \}
 
   /* now num2 is ready and contains a copy of num1 */

   /* We're done with them. */
   mp_clear_multi(&num1, &num2, NULL);

   return EXIT_SUCCESS;
\}
\end{alltt} \end{small}

Another less common initializer is mp\_init\_size() which allows the user to initialize an mp\_int with a given
default number of digits.  By default, all initializers allocate \textbf{MP\_PREC} digits.  This function lets
you override this behaviour.

\index{mp\_init\_size}
\begin{alltt}
int mp_init_size (mp_int * a, int size);
\end{alltt}

The $size$ parameter must be greater than zero.  If the function succeeds the mp\_int $a$ will be initialized
to have $size$ digits (which are all initially zero).  

\begin{small} \begin{alltt}
int main(void)
\{
   mp_int number;
   int result;

   /* we need a 60-digit number */
   if ((result = mp_init_size(&number, 60)) != MP_OKAY) \{
      printf("Error initializing the number.  \%s", 
             mp_error_to_string(result));
      return EXIT_FAILURE;
   \}
 
   /* use the number */

   return EXIT_SUCCESS;
\}
\end{alltt} \end{small}

\section{Maintenance Functions}

\subsection{Reducing Memory Usage}
When an mp\_int is in a state where it won't be changed again\footnote{A Diffie-Hellman modulus for instance.} excess
digits can be removed to return memory to the heap with the mp\_shrink() function.

\index{mp\_shrink}
\begin{alltt}
int mp_shrink (mp_int * a);
\end{alltt}

This will remove excess digits of the mp\_int $a$.  If the operation fails the mp\_int should be intact without the
excess digits being removed.  Note that you can use a shrunk mp\_int in further computations, however, such operations
will require heap operations which can be slow.  It is not ideal to shrink mp\_int variables that you will further
modify in the system (unless you are seriously low on memory).  

\begin{small} \begin{alltt}
int main(void)
\{
   mp_int number;
   int result;

   if ((result = mp_init(&number)) != MP_OKAY) \{
      printf("Error initializing the number.  \%s", 
             mp_error_to_string(result));
      return EXIT_FAILURE;
   \}
 
   /* use the number [e.g. pre-computation]  */

   /* We're done with it for now. */
   if ((result = mp_shrink(&number)) != MP_OKAY) \{
      printf("Error shrinking the number.  \%s", 
             mp_error_to_string(result));
      return EXIT_FAILURE;
   \}

   /* use it .... */


   /* we're done with it. */ 
   mp_clear(&number);

   return EXIT_SUCCESS;
\}
\end{alltt} \end{small}

\subsection{Adding additional digits}

Within the mp\_int structure are two parameters which control the limitations of the array of digits that represent
the integer the mp\_int is meant to equal.   The \textit{used} parameter dictates how many digits are significant, that is,
contribute to the value of the mp\_int.  The \textit{alloc} parameter dictates how many digits are currently available in
the array.  If you need to perform an operation that requires more digits you will have to mp\_grow() the mp\_int to
your desired size.  

\index{mp\_grow}
\begin{alltt}
int mp_grow (mp_int * a, int size);
\end{alltt}

This will grow the array of digits of $a$ to $size$.  If the \textit{alloc} parameter is already bigger than
$size$ the function will not do anything.

\begin{small} \begin{alltt}
int main(void)
\{
   mp_int number;
   int result;

   if ((result = mp_init(&number)) != MP_OKAY) \{
      printf("Error initializing the number.  \%s", 
             mp_error_to_string(result));
      return EXIT_FAILURE;
   \}
 
   /* use the number */

   /* We need to add 20 digits to the number  */
   if ((result = mp_grow(&number, number.alloc + 20)) != MP_OKAY) \{
      printf("Error growing the number.  \%s", 
             mp_error_to_string(result));
      return EXIT_FAILURE;
   \}


   /* use the number */

   /* we're done with it. */ 
   mp_clear(&number);

   return EXIT_SUCCESS;
\}
\end{alltt} \end{small}

\chapter{Basic Operations}
\section{Small Constants}
Setting mp\_ints to small constants is a relatively common operation.  To accomodate these instances there are two
small constant assignment functions.  The first function is used to set a single digit constant while the second sets
an ISO C style ``unsigned long'' constant.  The reason for both functions is efficiency.  Setting a single digit is quick but the
domain of a digit can change (it's always at least $0 \ldots 127$).  

\subsection{Single Digit}

Setting a single digit can be accomplished with the following function.

\index{mp\_set}
\begin{alltt}
void mp_set (mp_int * a, mp_digit b);
\end{alltt}

This will zero the contents of $a$ and make it represent an integer equal to the value of $b$.  Note that this
function has a return type of \textbf{void}.  It cannot cause an error so it is safe to assume the function
succeeded.

\begin{small} \begin{alltt}
int main(void)
\{
   mp_int number;
   int result;

   if ((result = mp_init(&number)) != MP_OKAY) \{
      printf("Error initializing the number.  \%s", 
             mp_error_to_string(result));
      return EXIT_FAILURE;
   \}
 
   /* set the number to 5 */
   mp_set(&number, 5);

   /* we're done with it. */ 
   mp_clear(&number);

   return EXIT_SUCCESS;
\}
\end{alltt} \end{small}

\subsection{Long Constants}

To set a constant that is the size of an ISO C ``unsigned long'' and larger than a single digit the following function 
can be used.

\index{mp\_set\_int}
\begin{alltt}
int mp_set_int (mp_int * a, unsigned long b);
\end{alltt}

This will assign the value of the 32-bit variable $b$ to the mp\_int $a$.  Unlike mp\_set() this function will always
accept a 32-bit input regardless of the size of a single digit.  However, since the value may span several digits 
this function can fail if it runs out of heap memory.

To get the ``unsigned long'' copy of an mp\_int the following function can be used.

\index{mp\_get\_int}
\begin{alltt}
unsigned long mp_get_int (mp_int * a);
\end{alltt}

This will return the 32 least significant bits of the mp\_int $a$.  

\begin{small} \begin{alltt}
int main(void)
\{
   mp_int number;
   int result;

   if ((result = mp_init(&number)) != MP_OKAY) \{
      printf("Error initializing the number.  \%s", 
             mp_error_to_string(result));
      return EXIT_FAILURE;
   \}
 
   /* set the number to 654321 (note this is bigger than 127) */
   if ((result = mp_set_int(&number, 654321)) != MP_OKAY) \{
      printf("Error setting the value of the number.  \%s", 
             mp_error_to_string(result));
      return EXIT_FAILURE;
   \}

   printf("number == \%lu", mp_get_int(&number));

   /* we're done with it. */ 
   mp_clear(&number);

   return EXIT_SUCCESS;
\}
\end{alltt} \end{small}

This should output the following if the program succeeds.

\begin{alltt}
number == 654321
\end{alltt}

\subsection{Initialize and Setting Constants}
To both initialize and set small constants the following two functions are available.
\index{mp\_init\_set} \index{mp\_init\_set\_int}
\begin{alltt}
int mp_init_set (mp_int * a, mp_digit b);
int mp_init_set_int (mp_int * a, unsigned long b);
\end{alltt}

Both functions work like the previous counterparts except they first mp\_init $a$ before setting the values.  

\begin{alltt}
int main(void)
\{
   mp_int number1, number2;
   int    result;

   /* initialize and set a single digit */
   if ((result = mp_init_set(&number1, 100)) != MP_OKAY) \{
      printf("Error setting number1: \%s", 
             mp_error_to_string(result));
      return EXIT_FAILURE;
   \}             

   /* initialize and set a long */
   if ((result = mp_init_set_int(&number2, 1023)) != MP_OKAY) \{
      printf("Error setting number2: \%s", 
             mp_error_to_string(result));
      return EXIT_FAILURE;
   \}

   /* display */
   printf("Number1, Number2 == \%lu, \%lu",
          mp_get_int(&number1), mp_get_int(&number2));

   /* clear */
   mp_clear_multi(&number1, &number2, NULL);

   return EXIT_SUCCESS;
\}
\end{alltt}

If this program succeeds it shall output.
\begin{alltt}
Number1, Number2 == 100, 1023
\end{alltt}

\section{Comparisons}

Comparisons in LibTomMath are always performed in a ``left to right'' fashion.  There are three possible return codes
for any comparison.

\index{MP\_GT} \index{MP\_EQ} \index{MP\_LT}
\begin{figure}[here]
\begin{center}
\begin{tabular}{|c|c|}
\hline \textbf{Result Code} & \textbf{Meaning} \\
\hline MP\_GT & $a > b$ \\
\hline MP\_EQ & $a = b$ \\
\hline MP\_LT & $a < b$ \\
\hline
\end{tabular}
\end{center}
\caption{Comparison Codes for $a, b$}
\label{fig:CMP}
\end{figure}

In figure \ref{fig:CMP} two integers $a$ and $b$ are being compared.  In this case $a$ is said to be ``to the left'' of 
$b$.  

\subsection{Unsigned comparison}

An unsigned comparison considers only the digits themselves and not the associated \textit{sign} flag of the 
mp\_int structures.  This is analogous to an absolute comparison.  The function mp\_cmp\_mag() will compare two
mp\_int variables based on their digits only. 

\index{mp\_cmp\_mag}
\begin{alltt}
int mp_cmp_mag(mp_int * a, mp_int * b);
\end{alltt}
This will compare $a$ to $b$ placing $a$ to the left of $b$.  This function cannot fail and will return one of the
three compare codes listed in figure \ref{fig:CMP}.

\begin{small} \begin{alltt}
int main(void)
\{
   mp_int number1, number2;
   int result;

   if ((result = mp_init_multi(&number1, &number2, NULL)) != MP_OKAY) \{
      printf("Error initializing the numbers.  \%s", 
             mp_error_to_string(result));
      return EXIT_FAILURE;
   \}
 
   /* set the number1 to 5 */
   mp_set(&number1, 5);
  
   /* set the number2 to -6 */
   mp_set(&number2, 6);
   if ((result = mp_neg(&number2, &number2)) != MP_OKAY) \{
      printf("Error negating number2.  \%s", 
             mp_error_to_string(result));
      return EXIT_FAILURE;
   \}

   switch(mp_cmp_mag(&number1, &number2)) \{
       case MP_GT:  printf("|number1| > |number2|"); break;
       case MP_EQ:  printf("|number1| = |number2|"); break;
       case MP_LT:  printf("|number1| < |number2|"); break;
   \}

   /* we're done with it. */ 
   mp_clear_multi(&number1, &number2, NULL);

   return EXIT_SUCCESS;
\}
\end{alltt} \end{small}

If this program\footnote{This function uses the mp\_neg() function which is discussed in section \ref{sec:NEG}.} completes 
successfully it should print the following.

\begin{alltt}
|number1| < |number2|
\end{alltt}

This is because $\vert -6 \vert = 6$ and obviously $5 < 6$.

\subsection{Signed comparison}

To compare two mp\_int variables based on their signed value the mp\_cmp() function is provided.

\index{mp\_cmp}
\begin{alltt}
int mp_cmp(mp_int * a, mp_int * b);
\end{alltt}

This will compare $a$ to the left of $b$.  It will first compare the signs of the two mp\_int variables.  If they
differ it will return immediately based on their signs.  If the signs are equal then it will compare the digits
individually.  This function will return one of the compare conditions codes listed in figure \ref{fig:CMP}.

\begin{small} \begin{alltt}
int main(void)
\{
   mp_int number1, number2;
   int result;

   if ((result = mp_init_multi(&number1, &number2, NULL)) != MP_OKAY) \{
      printf("Error initializing the numbers.  \%s", 
             mp_error_to_string(result));
      return EXIT_FAILURE;
   \}
 
   /* set the number1 to 5 */
   mp_set(&number1, 5);
  
   /* set the number2 to -6 */
   mp_set(&number2, 6);
   if ((result = mp_neg(&number2, &number2)) != MP_OKAY) \{
      printf("Error negating number2.  \%s", 
             mp_error_to_string(result));
      return EXIT_FAILURE;
   \}

   switch(mp_cmp(&number1, &number2)) \{
       case MP_GT:  printf("number1 > number2"); break;
       case MP_EQ:  printf("number1 = number2"); break;
       case MP_LT:  printf("number1 < number2"); break;
   \}

   /* we're done with it. */ 
   mp_clear_multi(&number1, &number2, NULL);

   return EXIT_SUCCESS;
\}
\end{alltt} \end{small}

If this program\footnote{This function uses the mp\_neg() function which is discussed in section \ref{sec:NEG}.} completes 
successfully it should print the following.

\begin{alltt}
number1 > number2
\end{alltt}

\subsection{Single Digit}

To compare a single digit against an mp\_int the following function has been provided.

\index{mp\_cmp\_d}
\begin{alltt}
int mp_cmp_d(mp_int * a, mp_digit b);
\end{alltt}

This will compare $a$ to the left of $b$ using a signed comparison.  Note that it will always treat $b$ as 
positive.  This function is rather handy when you have to compare against small values such as $1$ (which often
comes up in cryptography).  The function cannot fail and will return one of the tree compare condition codes
listed in figure \ref{fig:CMP}.


\begin{small} \begin{alltt}
int main(void)
\{
   mp_int number;
   int result;

   if ((result = mp_init(&number)) != MP_OKAY) \{
      printf("Error initializing the number.  \%s", 
             mp_error_to_string(result));
      return EXIT_FAILURE;
   \}
 
   /* set the number to 5 */
   mp_set(&number, 5);

   switch(mp_cmp_d(&number, 7)) \{
       case MP_GT:  printf("number > 7"); break;
       case MP_EQ:  printf("number = 7"); break;
       case MP_LT:  printf("number < 7"); break;
   \}

   /* we're done with it. */ 
   mp_clear(&number);

   return EXIT_SUCCESS;
\}
\end{alltt} \end{small}

If this program functions properly it will print out the following.

\begin{alltt}
number < 7
\end{alltt}

\section{Logical Operations}

Logical operations are operations that can be performed either with simple shifts or boolean operators such as
AND, XOR and OR directly.  These operations are very quick.

\subsection{Multiplication by two}

Multiplications and divisions by any power of two can be performed with quick logical shifts either left or
right depending on the operation.  

When multiplying or dividing by two a special case routine can be used which are as follows.
\index{mp\_mul\_2} \index{mp\_div\_2}
\begin{alltt}
int mp_mul_2(mp_int * a, mp_int * b);
int mp_div_2(mp_int * a, mp_int * b);
\end{alltt}

The former will assign twice $a$ to $b$ while the latter will assign half $a$ to $b$.  These functions are fast
since the shift counts and maskes are hardcoded into the routines.

\begin{small} \begin{alltt}
int main(void)
\{
   mp_int number;
   int result;

   if ((result = mp_init(&number)) != MP_OKAY) \{
      printf("Error initializing the number.  \%s", 
             mp_error_to_string(result));
      return EXIT_FAILURE;
   \}
 
   /* set the number to 5 */
   mp_set(&number, 5);

   /* multiply by two */
   if ((result = mp\_mul\_2(&number, &number)) != MP_OKAY) \{
      printf("Error multiplying the number.  \%s", 
             mp_error_to_string(result));
      return EXIT_FAILURE;
   \}
   switch(mp_cmp_d(&number, 7)) \{
       case MP_GT:  printf("2*number > 7"); break;
       case MP_EQ:  printf("2*number = 7"); break;
       case MP_LT:  printf("2*number < 7"); break;
   \}

   /* now divide by two */
   if ((result = mp\_div\_2(&number, &number)) != MP_OKAY) \{
      printf("Error dividing the number.  \%s", 
             mp_error_to_string(result));
      return EXIT_FAILURE;
   \}
   switch(mp_cmp_d(&number, 7)) \{
       case MP_GT:  printf("2*number/2 > 7"); break;
       case MP_EQ:  printf("2*number/2 = 7"); break;
       case MP_LT:  printf("2*number/2 < 7"); break;
   \}

   /* we're done with it. */ 
   mp_clear(&number);

   return EXIT_SUCCESS;
\}
\end{alltt} \end{small}

If this program is successful it will print out the following text.

\begin{alltt}
2*number > 7
2*number/2 < 7
\end{alltt}

Since $10 > 7$ and $5 < 7$.  To multiply by a power of two the following function can be used.

\index{mp\_mul\_2d}
\begin{alltt}
int mp_mul_2d(mp_int * a, int b, mp_int * c);
\end{alltt}

This will multiply $a$ by $2^b$ and store the result in ``c''.  If the value of $b$ is less than or equal to 
zero the function will copy $a$ to ``c'' without performing any further actions.  

To divide by a power of two use the following.

\index{mp\_div\_2d}
\begin{alltt}
int mp_div_2d (mp_int * a, int b, mp_int * c, mp_int * d);
\end{alltt}
Which will divide $a$ by $2^b$, store the quotient in ``c'' and the remainder in ``d'.  If $b \le 0$ then the
function simply copies $a$ over to ``c'' and zeroes $d$.  The variable $d$ may be passed as a \textbf{NULL}
value to signal that the remainder is not desired.

\subsection{Polynomial Basis Operations}

Strictly speaking the organization of the integers within the mp\_int structures is what is known as a 
``polynomial basis''.  This simply means a field element is stored by divisions of a radix.  For example, if
$f(x) = \sum_{i=0}^{k} y_ix^k$ for any vector $\vec y$ then the array of digits in $\vec y$ are said to be 
the polynomial basis representation of $z$ if $f(\beta) = z$ for a given radix $\beta$.  

To multiply by the polynomial $g(x) = x$ all you have todo is shift the digits of the basis left one place.  The
following function provides this operation.

\index{mp\_lshd}
\begin{alltt}
int mp_lshd (mp_int * a, int b);
\end{alltt}

This will multiply $a$ in place by $x^b$ which is equivalent to shifting the digits left $b$ places and inserting zeroes
in the least significant digits.  Similarly to divide by a power of $x$ the following function is provided.

\index{mp\_rshd}
\begin{alltt}
void mp_rshd (mp_int * a, int b)
\end{alltt}
This will divide $a$ in place by $x^b$ and discard the remainder.  This function cannot fail as it performs the operations
in place and no new digits are required to complete it.

\subsection{AND, OR and XOR Operations}

While AND, OR and XOR operations are not typical ``bignum functions'' they can be useful in several instances.  The
three functions are prototyped as follows.

\index{mp\_or} \index{mp\_and} \index{mp\_xor}
\begin{alltt}
int mp_or  (mp_int * a, mp_int * b, mp_int * c);
int mp_and (mp_int * a, mp_int * b, mp_int * c);
int mp_xor (mp_int * a, mp_int * b, mp_int * c);
\end{alltt}

Which compute $c = a \odot b$ where $\odot$ is one of OR, AND or XOR.  

\section{Addition and Subtraction}

To compute an addition or subtraction the following two functions can be used.

\index{mp\_add} \index{mp\_sub}
\begin{alltt}
int mp_add (mp_int * a, mp_int * b, mp_int * c);
int mp_sub (mp_int * a, mp_int * b, mp_int * c)
\end{alltt}

Which perform $c = a \odot b$ where $\odot$ is one of signed addition or subtraction.  The operations are fully sign
aware.

\section{Sign Manipulation}
\subsection{Negation}
\label{sec:NEG}
Simple integer negation can be performed with the following.

\index{mp\_neg}
\begin{alltt}
int mp_neg (mp_int * a, mp_int * b);
\end{alltt}

Which assigns $-a$ to $b$.  

\subsection{Absolute}
Simple integer absolutes can be performed with the following.

\index{mp\_neg}
\begin{alltt}
int mp_abs (mp_int * a, mp_int * b);
\end{alltt}

Which assigns $\vert a \vert$ to $b$.  

\section{Integer Division and Remainder}
To perform a complete and general integer division with remainder use the following function.

\index{mp\_div}
\begin{alltt}
int mp_div (mp_int * a, mp_int * b, mp_int * c, mp_int * d);
\end{alltt}
                                                        
This divides $a$ by $b$ and stores the quotient in $c$ and $d$.  The signed quotient is computed such that 
$bc + d = a$.  Note that either of $c$ or $d$ can be set to \textbf{NULL} if their value is not required.  If 
$b$ is zero the function returns \textbf{MP\_VAL}.  


\chapter{Multiplication and Squaring}
\section{Multiplication}
A full signed integer multiplication can be performed with the following.
\index{mp\_mul}
\begin{alltt}
int mp_mul (mp_int * a, mp_int * b, mp_int * c);
\end{alltt}
Which assigns the full signed product $ab$ to $c$.  This function actually breaks into one of four cases which are 
specific multiplication routines optimized for given parameters.  First there are the Toom-Cook multiplications which
should only be used with very large inputs.  This is followed by the Karatsuba multiplications which are for moderate
sized inputs.  Then followed by the Comba and baseline multipliers.

Fortunately for the developer you don't really need to know this unless you really want to fine tune the system.  mp\_mul()
will determine on its own\footnote{Some tweaking may be required.} what routine to use automatically when it is called.

\begin{alltt}
int main(void)
\{
   mp_int number1, number2;
   int result;

   /* Initialize the numbers */
   if ((result = mp_init_multi(&number1, 
                               &number2, NULL)) != MP_OKAY) \{
      printf("Error initializing the numbers.  \%s", 
             mp_error_to_string(result));
      return EXIT_FAILURE;
   \}

   /* set the terms */
   if ((result = mp_set_int(&number, 257)) != MP_OKAY) \{
      printf("Error setting number1.  \%s", 
             mp_error_to_string(result));
      return EXIT_FAILURE;
   \}
 
   if ((result = mp_set_int(&number2, 1023)) != MP_OKAY) \{
      printf("Error setting number2.  \%s", 
             mp_error_to_string(result));
      return EXIT_FAILURE;
   \}

   /* multiply them */
   if ((result = mp_mul(&number1, &number2,
                        &number1)) != MP_OKAY) \{
      printf("Error multiplying terms.  \%s", 
             mp_error_to_string(result));
      return EXIT_FAILURE;
   \}

   /* display */
   printf("number1 * number2 == \%lu", mp_get_int(&number1));

   /* free terms and return */
   mp_clear_multi(&number1, &number2, NULL);

   return EXIT_SUCCESS;
\}
\end{alltt}   

If this program succeeds it shall output the following.

\begin{alltt}
number1 * number2 == 262911
\end{alltt}

\section{Squaring}
Since squaring can be performed faster than multiplication it is performed it's own function instead of just using
mp\_mul().

\index{mp\_sqr}
\begin{alltt}
int mp_sqr (mp_int * a, mp_int * b);
\end{alltt}

Will square $a$ and store it in $b$.  Like the case of multiplication there are four different squaring
algorithms all which can be called from mp\_sqr().  It is ideal to use mp\_sqr over mp\_mul when squaring terms because
of the speed difference.  

\section{Tuning Polynomial Basis Routines}

Both of the Toom-Cook and Karatsuba multiplication algorithms are faster than the traditional $O(n^2)$ approach that
the Comba and baseline algorithms use.  At $O(n^{1.464973})$ and $O(n^{1.584962})$ running times respectively they require 
considerably less work.  For example, a 10000-digit multiplication would take roughly 724,000 single precision
multiplications with Toom-Cook or 100,000,000 single precision multiplications with the standard Comba (a factor
of 138).

So why not always use Karatsuba or Toom-Cook?   The simple answer is that they have so much overhead that they're not
actually faster than Comba until you hit distinct  ``cutoff'' points.  For Karatsuba with the default configuration, 
GCC 3.3.1 and an Athlon XP processor the cutoff point is roughly 110 digits (about 70 for the Intel P4).  That is, at 
110 digits Karatsuba and Comba multiplications just about break even and for 110+ digits Karatsuba is faster.

Toom-Cook has incredible overhead and is probably only useful for very large inputs.  So far no known cutoff points 
exist and for the most part I just set the cutoff points very high to make sure they're not called.

A demo program in the ``etc/'' directory of the project called ``tune.c'' can be used to find the cutoff points.  This
can be built with GCC as follows

\begin{alltt}
make XXX
\end{alltt}
Where ``XXX'' is one of the following entries from the table \ref{fig:tuning}.

\begin{figure}[here]
\begin{center}
\begin{small}
\begin{tabular}{|l|l|}
\hline \textbf{Value of XXX} & \textbf{Meaning} \\
\hline tune & Builds portable tuning application \\
\hline tune86 & Builds x86 (pentium and up) program for COFF \\
\hline tune86c & Builds x86 program for Cygwin \\
\hline tune86l & Builds x86 program for Linux (ELF format) \\
\hline
\end{tabular}
\end{small}
\end{center}
\caption{Build Names for Tuning Programs}
\label{fig:tuning}
\end{figure}

When the program is running it will output a series of measurements for different cutoff points.  It will first find
good Karatsuba squaring and multiplication points.  Then it proceeds to find Toom-Cook points.  Note that the Toom-Cook
tuning takes a very long time as the cutoff points are likely to be very high.

\chapter{Modular Reduction}

Modular reduction is process of taking the remainder of one quantity divided by another.  Expressed 
as (\ref{eqn:mod}) the modular reduction is equivalent to the remainder of $b$ divided by $c$.  

\begin{equation}
a \equiv b \mbox{ (mod }c\mbox{)}
\label{eqn:mod}
\end{equation}

Of particular interest to cryptography are reductions where $b$ is limited to the range $0 \le b < c^2$ since particularly 
fast reduction algorithms can be written for the limited range.  

Note that one of the four optimized reduction algorithms are automatically chosen in the modular exponentiation
algorithm mp\_exptmod when an appropriate modulus is detected.  

\section{Straight Division}
In order to effect an arbitrary modular reduction the following algorithm is provided.

\index{mp\_mod}
\begin{alltt}
int mp_mod(mp_int *a, mp_int *b, mp_int *c);
\end{alltt}

This reduces $a$ modulo $b$ and stores the result in $c$.  The sign of $c$ shall agree with the sign 
of $b$.  This algorithm accepts an input $a$ of any range and is not limited by $0 \le a < b^2$.

\section{Barrett Reduction}

Barrett reduction is a generic optimized reduction algorithm that requires pre--computation to achieve
a decent speedup over straight division.  First a $\mu$ value must be precomputed with the following function.

\index{mp\_reduce\_setup}
\begin{alltt}
int mp_reduce_setup(mp_int *a, mp_int *b);
\end{alltt}

Given a modulus in $b$ this produces the required $\mu$ value in $a$.  For any given modulus this only has to
be computed once.  Modular reduction can now be performed with the following.

\index{mp\_reduce}
\begin{alltt}
int mp_reduce(mp_int *a, mp_int *b, mp_int *c);
\end{alltt}

This will reduce $a$ in place modulo $b$ with the precomputed $\mu$ value in $c$.  $a$ must be in the range
$0 \le a < b^2$.

\begin{alltt}
int main(void)
\{
   mp_int   a, b, c, mu;
   int      result;

   /* initialize a,b to desired values, mp_init mu, 
    * c and set c to 1...we want to compute a^3 mod b 
    */

   /* get mu value */
   if ((result = mp_reduce_setup(&mu, b)) != MP_OKAY) \{
      printf("Error getting mu.  \%s", 
             mp_error_to_string(result));
      return EXIT_FAILURE;
   \}

   /* square a to get c = a^2 */
   if ((result = mp_sqr(&a, &c)) != MP_OKAY) \{
      printf("Error squaring.  \%s", 
             mp_error_to_string(result));
      return EXIT_FAILURE;
   \}

   /* now reduce `c' modulo b */
   if ((result = mp_reduce(&c, &b, &mu)) != MP_OKAY) \{
      printf("Error reducing.  \%s", 
             mp_error_to_string(result));
      return EXIT_FAILURE;
   \}
   
   /* multiply a to get c = a^3 */
   if ((result = mp_mul(&a, &c, &c)) != MP_OKAY) \{
      printf("Error reducing.  \%s", 
             mp_error_to_string(result));
      return EXIT_FAILURE;
   \}

   /* now reduce `c' modulo b  */
   if ((result = mp_reduce(&c, &b, &mu)) != MP_OKAY) \{
      printf("Error reducing.  \%s", 
             mp_error_to_string(result));
      return EXIT_FAILURE;
   \}
  
   /* c now equals a^3 mod b */

   return EXIT_SUCCESS;
\}
\end{alltt} 

This program will calculate $a^3 \mbox{ mod }b$ if all the functions succeed.  

\section{Montgomery Reduction}

Montgomery is a specialized reduction algorithm for any odd moduli.  Like Barrett reduction a pre--computation
step is required.  This is accomplished with the following.

\index{mp\_montgomery\_setup}
\begin{alltt}
int mp_montgomery_setup(mp_int *a, mp_digit *mp);
\end{alltt}

For the given odd moduli $a$ the precomputation value is placed in $mp$.  The reduction is computed with the 
following.

\index{mp\_montgomery\_reduce}
\begin{alltt}
int mp_montgomery_reduce(mp_int *a, mp_int *m, mp_digit mp);
\end{alltt}
This reduces $a$ in place modulo $m$ with the pre--computed value $mp$.   $a$ must be in the range
$0 \le a < b^2$.

Montgomery reduction is faster than Barrett reduction for moduli smaller than the ``comba'' limit.  With the default
setup for instance, the limit is $127$ digits ($3556$--bits).   Note that this function is not limited to
$127$ digits just that it falls back to a baseline algorithm after that point.  

An important observation is that this reduction does not return $a \mbox{ mod }m$ but $aR^{-1} \mbox{ mod }m$ 
where $R = \beta^n$, $n$ is the n number of digits in $m$ and $\beta$ is radix used (default is $2^{28}$).  

To quickly calculate $R$ the following function was provided.

\index{mp\_montgomery\_calc\_normalization}
\begin{alltt}
int mp_montgomery_calc_normalization(mp_int *a, mp_int *b);
\end{alltt}
Which calculates $a = R$ for the odd moduli $b$ without using multiplication or division.  

The normal modus operandi for Montgomery reductions is to normalize the integers before entering the system.  For
example, to calculate $a^3 \mbox { mod }b$ using Montgomery reduction the value of $a$ can be normalized by
multiplying it by $R$.  Consider the following code snippet.

\begin{alltt}
int main(void)
\{
   mp_int   a, b, c, R;
   mp_digit mp;
   int      result;

   /* initialize a,b to desired values, 
    * mp_init R, c and set c to 1.... 
    */

   /* get normalization */
   if ((result = mp_montgomery_calc_normalization(&R, b)) != MP_OKAY) \{
      printf("Error getting norm.  \%s", 
             mp_error_to_string(result));
      return EXIT_FAILURE;
   \}

   /* get mp value */
   if ((result = mp_montgomery_setup(&c, &mp)) != MP_OKAY) \{
      printf("Error setting up montgomery.  \%s", 
             mp_error_to_string(result));
      return EXIT_FAILURE;
   \}

   /* normalize `a' so now a is equal to aR */
   if ((result = mp_mulmod(&a, &R, &b, &a)) != MP_OKAY) \{
      printf("Error computing aR.  \%s", 
             mp_error_to_string(result));
      return EXIT_FAILURE;
   \}

   /* square a to get c = a^2R^2 */
   if ((result = mp_sqr(&a, &c)) != MP_OKAY) \{
      printf("Error squaring.  \%s", 
             mp_error_to_string(result));
      return EXIT_FAILURE;
   \}

   /* now reduce `c' back down to c = a^2R^2 * R^-1 == a^2R */
   if ((result = mp_montgomery_reduce(&c, &b, mp)) != MP_OKAY) \{
      printf("Error reducing.  \%s", 
             mp_error_to_string(result));
      return EXIT_FAILURE;
   \}
   
   /* multiply a to get c = a^3R^2 */
   if ((result = mp_mul(&a, &c, &c)) != MP_OKAY) \{
      printf("Error reducing.  \%s", 
             mp_error_to_string(result));
      return EXIT_FAILURE;
   \}

   /* now reduce `c' back down to c = a^3R^2 * R^-1 == a^3R */
   if ((result = mp_montgomery_reduce(&c, &b, mp)) != MP_OKAY) \{
      printf("Error reducing.  \%s", 
             mp_error_to_string(result));
      return EXIT_FAILURE;
   \}
   
   /* now reduce (again) `c' back down to c = a^3R * R^-1 == a^3 */
   if ((result = mp_montgomery_reduce(&c, &b, mp)) != MP_OKAY) \{
      printf("Error reducing.  \%s", 
             mp_error_to_string(result));
      return EXIT_FAILURE;
   \}

   /* c now equals a^3 mod b */

   return EXIT_SUCCESS;
\}
\end{alltt} 

This particular example does not look too efficient but it demonstrates the point of the algorithm.  By 
normalizing the inputs the reduced results are always of the form $aR$ for some variable $a$.  This allows
a single final reduction to correct for the normalization and the fast reduction used within the algorithm.

For more details consider examining the file \textit{bn\_mp\_exptmod\_fast.c}.

\section{Restricted Dimminished Radix}

``Dimminished Radix'' reduction refers to reduction with respect to moduli that are ameniable to simple
digit shifting and small multiplications.  In this case the ``restricted'' variant refers to moduli of the
form $\beta^k - p$ for some $k \ge 0$ and $0 < p < \beta$ where $\beta$ is the radix (default to $2^{28}$).  

As in the case of Montgomery reduction there is a pre--computation phase required for a given modulus.

\index{mp\_dr\_setup}
\begin{alltt}
void mp_dr_setup(mp_int *a, mp_digit *d);
\end{alltt}

This computes the value required for the modulus $a$ and stores it in $d$.  This function cannot fail
and does not return any error codes.  After the pre--computation a reduction can be performed with the
following.

\index{mp\_dr\_reduce}
\begin{alltt}
int mp_dr_reduce(mp_int *a, mp_int *b, mp_digit mp);
\end{alltt}

This reduces $a$ in place modulo $b$ with the pre--computed value $mp$.  $b$ must be of a restricted
dimminished radix form and $a$ must be in the range $0 \le a < b^2$.  Dimminished radix reductions are 
much faster than both Barrett and Montgomery reductions as they have a much lower asymtotic running time.  

Since the moduli are restricted this algorithm is not particularly useful for something like Rabin, RSA or
BBS cryptographic purposes.  This reduction algorithm is useful for Diffie-Hellman and ECC where fixed
primes are acceptable.  

Note that unlike Montgomery reduction there is no normalization process.  The result of this function is
equal to the correct residue.

\section{Unrestricted Dimminshed Radix}

Unrestricted reductions work much like the restricted counterparts except in this case the moduli is of the 
form $2^k - p$ for $0 < p < \beta$.  In this sense the unrestricted reductions are more flexible as they 
can be applied to a wider range of numbers.  

\index{mp\_reduce\_2k\_setup}
\begin{alltt}
int mp_reduce_2k_setup(mp_int *a, mp_digit *d);
\end{alltt}

This will compute the required $d$ value for the given moduli $a$.  

\index{mp\_reduce\_2k}
\begin{alltt}
int mp_reduce_2k(mp_int *a, mp_int *n, mp_digit d);
\end{alltt}

This will reduce $a$ in place modulo $n$ with the pre--computed value $d$.  From my experience this routine is 
slower than mp\_dr\_reduce but faster for most moduli sizes than the Montgomery reduction.  

\chapter{Exponentiation}
\section{Single Digit Exponentiation}
\index{mp\_expt\_d}
\begin{alltt}
int mp_expt_d (mp_int * a, mp_digit b, mp_int * c)
\end{alltt}
This computes $c = a^b$ using a simple binary left-to-right algorithm.  It is faster than repeated multiplications by 
$a$ for all values of $b$ greater than three.  

\section{Modular Exponentiation}
\index{mp\_exptmod}
\begin{alltt}
int mp_exptmod (mp_int * G, mp_int * X, mp_int * P, mp_int * Y)
\end{alltt}
This computes $Y \equiv G^X \mbox{ (mod }P\mbox{)}$ using a variable width sliding window algorithm.  This function
will automatically detect the fastest modular reduction technique to use during the operation.  For negative values of 
$X$ the operation is performed as $Y \equiv (G^{-1} \mbox{ mod }P)^{\vert X \vert} \mbox{ (mod }P\mbox{)}$ provided that 
$gcd(G, P) = 1$.

This function is actually a shell around the two internal exponentiation functions.  This routine will automatically
detect when Barrett, Montgomery, Restricted and Unrestricted Dimminished Radix based exponentiation can be used.  Generally
moduli of the a ``restricted dimminished radix'' form lead to the fastest modular exponentiations.  Followed by Montgomery
and the other two algorithms.

\section{Root Finding}
\index{mp\_n\_root}
\begin{alltt}
int mp_n_root (mp_int * a, mp_digit b, mp_int * c)
\end{alltt}
This computes $c = a^{1/b}$ such that $c^b \le a$ and $(c+1)^b > a$.  The implementation of this function is not 
ideal for values of $b$ greater than three.  It will work but become very slow.  So unless you are working with very small
numbers (less than 1000 bits) I'd avoid $b > 3$ situations.  Will return a positive root only for even roots and return
a root with the sign of the input for odd roots.  For example, performing $4^{1/2}$ will return $2$ whereas $(-8)^{1/3}$ 
will return $-2$.  

This algorithm uses the ``Newton Approximation'' method and will converge on the correct root fairly quickly.  Since
the algorithm requires raising $a$ to the power of $b$ it is not ideal to attempt to find roots for large
values of $b$.  If particularly large roots are required then a factor method could be used instead.  For example,
$a^{1/16}$ is equivalent to $\left (a^{1/4} \right)^{1/4}$ or simply 
$\left ( \left ( \left ( a^{1/2} \right )^{1/2} \right )^{1/2} \right )^{1/2}$

\chapter{Prime Numbers}
\section{Trial Division}
\index{mp\_prime\_is\_divisible}
\begin{alltt}
int mp_prime_is_divisible (mp_int * a, int *result)
\end{alltt}
This will attempt to evenly divide $a$ by a list of primes\footnote{Default is the first 256 primes.} and store the 
outcome in ``result''.  That is if $result = 0$ then $a$ is not divisible by the primes, otherwise it is.  Note that 
if the function does not return \textbf{MP\_OKAY} the value in ``result'' should be considered undefined\footnote{Currently
the default is to set it to zero first.}.

\section{Fermat Test}
\index{mp\_prime\_fermat}
\begin{alltt}
int mp_prime_fermat (mp_int * a, mp_int * b, int *result)
\end{alltt}
Performs a Fermat primality test to the base $b$.  That is it computes $b^a \mbox{ mod }a$ and tests whether the value is
equal to $b$ or not.  If the values are equal then $a$ is probably prime and $result$ is set to one.  Otherwise $result$
is set to zero.

\section{Miller-Rabin Test}
\index{mp\_prime\_miller\_rabin}
\begin{alltt}
int mp_prime_miller_rabin (mp_int * a, mp_int * b, int *result)
\end{alltt}
Performs a Miller-Rabin test to the base $b$ of $a$.  This test is much stronger than the Fermat test and is very hard to
fool (besides with Carmichael numbers).  If $a$ passes the test (therefore is probably prime) $result$ is set to one.  
Otherwise $result$ is set to zero.  

Note that is suggested that you use the Miller-Rabin test instead of the Fermat test since all of the failures of 
Miller-Rabin are a subset of the failures of the Fermat test.

\subsection{Required Number of Tests}
Generally to ensure a number is very likely to be prime you have to perform the Miller-Rabin with at least a half-dozen
or so unique bases.  However, it has been proven that the probability of failure goes down as the size of the input goes up.
This is why a simple function has been provided to help out.

\index{mp\_prime\_rabin\_miller\_trials}
\begin{alltt}
int mp_prime_rabin_miller_trials(int size)
\end{alltt}
This returns the number of trials required for a $2^{-96}$ (or lower) probability of failure for a given ``size'' expressed
in bits.  This comes in handy specially since larger numbers are slower to test.  For example, a 512-bit number would
require ten tests whereas a 1024-bit number would only require four tests. 

You should always still perform a trial division before a Miller-Rabin test though.

\section{Primality Testing}
\index{mp\_prime\_is\_prime}
\begin{alltt}
int mp_prime_is_prime (mp_int * a, int t, int *result)
\end{alltt}
This will perform a trial division followed by $t$ rounds of Miller-Rabin tests on $a$ and store the result in $result$.  
If $a$ passes all of the tests $result$ is set to one, otherwise it is set to zero.  Note that $t$ is bounded by 
$1 \le t < PRIME\_SIZE$ where $PRIME\_SIZE$ is the number of primes in the prime number table (by default this is $256$).

\section{Next Prime}
\index{mp\_prime\_next\_prime}
\begin{alltt}
int mp_prime_next_prime(mp_int *a, int t, int bbs_style)
\end{alltt}
This finds the next prime after $a$ that passes mp\_prime\_is\_prime() with $t$ tests.  Set $bbs\_style$ to one if you 
want only the next prime congruent to $3 \mbox{ mod } 4$, otherwise set it to zero to find any next prime.  

\section{Random Primes}
\index{mp\_prime\_random}
\begin{alltt}
int mp_prime_random(mp_int *a, int t, int size, int bbs, 
                    ltm_prime_callback cb, void *dat)
\end{alltt}
This will find a prime greater than $256^{size}$ which can be ``bbs\_style'' or not depending on $bbs$ and must pass
$t$ rounds of tests.  The ``ltm\_prime\_callback'' is a typedef for 

\begin{alltt}
typedef int ltm_prime_callback(unsigned char *dst, int len, void *dat);
\end{alltt}

Which is a function that must read $len$ bytes (and return the amount stored) into $dst$.  The $dat$ variable is simply
copied from the original input.  It can be used to pass RNG context data to the callback.  The function 
mp\_prime\_random() is more suitable for generating primes which must be secret (as in the case of RSA) since there 
is no skew on the least significant bits.

\textit{Note:}  As of v0.30 of the LibTomMath library this function has been deprecated.  It is still available
but users are encouraged to use the new mp\_prime\_random\_ex() function instead.

\subsection{Extended Generation}
\index{mp\_prime\_random\_ex}
\begin{alltt}
int mp_prime_random_ex(mp_int *a,    int t, 
                       int     size, int flags, 
                       ltm_prime_callback cb, void *dat);
\end{alltt}
This will generate a prime in $a$ using $t$ tests of the primality testing algorithms.  The variable $size$
specifies the bit length of the prime desired.  The variable $flags$ specifies one of several options available
(see fig. \ref{fig:primeopts}) which can be OR'ed together.  The callback parameters are used as in 
mp\_prime\_random().

\begin{figure}[here]
\begin{center}
\begin{small}
\begin{tabular}{|r|l|}
\hline \textbf{Flag}         & \textbf{Meaning} \\
\hline LTM\_PRIME\_BBS       & Make the prime congruent to $3$ modulo $4$ \\
\hline LTM\_PRIME\_SAFE      & Make a prime $p$ such that $(p - 1)/2$ is also prime. \\
                             & This option implies LTM\_PRIME\_BBS as well. \\
\hline LTM\_PRIME\_2MSB\_OFF & Makes sure that the bit adjacent to the most significant bit \\
                             & Is forced to zero.  \\
\hline LTM\_PRIME\_2MSB\_ON  & Makes sure that the bit adjacent to the most significant bit \\
                             & Is forced to one. \\
\hline
\end{tabular}
\end{small}
\end{center}
\caption{Primality Generation Options}
\label{fig:primeopts}
\end{figure}

\chapter{Input and Output}
\section{ASCII Conversions}
\subsection{To ASCII}
\index{mp\_toradix}
\begin{alltt}
int mp_toradix (mp_int * a, char *str, int radix);
\end{alltt}
This still store $a$ in ``str'' as a base-``radix'' string of ASCII chars.  This function appends a NUL character
to terminate the string.  Valid values of ``radix'' line in the range $[2, 64]$.  To determine the size (exact) required
by the conversion before storing any data use the following function.

\index{mp\_radix\_size}
\begin{alltt}
int mp_radix_size (mp_int * a, int radix, int *size)
\end{alltt}
This stores in ``size'' the number of characters (including space for the NUL terminator) required.  Upon error this 
function returns an error code and ``size'' will be zero.  

\subsection{From ASCII}
\index{mp\_read\_radix}
\begin{alltt}
int mp_read_radix (mp_int * a, char *str, int radix);
\end{alltt}
This will read the base-``radix'' NUL terminated string from ``str'' into $a$.  It will stop reading when it reads a
character it does not recognize (which happens to include th NUL char... imagine that...).  A single leading $-$ sign
can be used to denote a negative number.

\section{Binary Conversions}

Converting an mp\_int to and from binary is another keen idea.

\index{mp\_unsigned\_bin\_size}
\begin{alltt}
int mp_unsigned_bin_size(mp_int *a);
\end{alltt}

This will return the number of bytes (octets) required to store the unsigned copy of the integer $a$.

\index{mp\_to\_unsigned\_bin}
\begin{alltt}
int mp_to_unsigned_bin(mp_int *a, unsigned char *b);
\end{alltt}
This will store $a$ into the buffer $b$ in big--endian format.  Fortunately this is exactly what DER (or is it ASN?)
requires.  It does not store the sign of the integer.

\index{mp\_read\_unsigned\_bin}
\begin{alltt}
int mp_read_unsigned_bin(mp_int *a, unsigned char *b, int c);
\end{alltt}
This will read in an unsigned big--endian array of bytes (octets) from $b$ of length $c$ into $a$.  The resulting
integer $a$ will always be positive.

For those who acknowledge the existence of negative numbers (heretic!) there are ``signed'' versions of the
previous functions.

\begin{alltt}
int mp_signed_bin_size(mp_int *a);
int mp_read_signed_bin(mp_int *a, unsigned char *b, int c);
int mp_to_signed_bin(mp_int *a, unsigned char *b);
\end{alltt}
They operate essentially the same as the unsigned copies except they prefix the data with zero or non--zero
byte depending on the sign.  If the sign is zpos (e.g. not negative) the prefix is zero, otherwise the prefix
is non--zero.  

\chapter{Algebraic Functions}
\section{Extended Euclidean Algorithm}
\index{mp\_exteuclid}
\begin{alltt}
int mp_exteuclid(mp_int *a, mp_int *b, 
                 mp_int *U1, mp_int *U2, mp_int *U3);
\end{alltt}

This finds the triple U1/U2/U3 using the Extended Euclidean algorithm such that the following equation holds.

\begin{equation}
a \cdot U1 + b \cdot U2 = U3
\end{equation}

Any of the U1/U2/U3 paramters can be set to \textbf{NULL} if they are not desired.  

\section{Greatest Common Divisor}
\index{mp\_gcd}
\begin{alltt}
int mp_gcd (mp_int * a, mp_int * b, mp_int * c)
\end{alltt}
This will compute the greatest common divisor of $a$ and $b$ and store it in $c$.

\section{Least Common Multiple}
\index{mp\_lcm}
\begin{alltt}
int mp_lcm (mp_int * a, mp_int * b, mp_int * c)
\end{alltt}
This will compute the least common multiple of $a$ and $b$ and store it in $c$.

\section{Jacobi Symbol}
\index{mp\_jacobi}
\begin{alltt}
int mp_jacobi (mp_int * a, mp_int * p, int *c)
\end{alltt}
This will compute the Jacobi symbol for $a$ with respect to $p$.  If $p$ is prime this essentially computes the Legendre
symbol.  The result is stored in $c$ and can take on one of three values $\lbrace -1, 0, 1 \rbrace$.  If $p$ is prime
then the result will be $-1$ when $a$ is not a quadratic residue modulo $p$.  The result will be $0$ if $a$ divides $p$
and the result will be $1$ if $a$ is a quadratic residue modulo $p$.  

\section{Modular Inverse}
\index{mp\_invmod}
\begin{alltt}
int mp_invmod (mp_int * a, mp_int * b, mp_int * c)
\end{alltt}
Computes the multiplicative inverse of $a$ modulo $b$ and stores the result in $c$ such that $ac \equiv 1 \mbox{ (mod }b\mbox{)}$.

\section{Single Digit Functions}

For those using small numbers (\textit{snicker snicker}) there are several ``helper'' functions

\index{mp\_add\_d} \index{mp\_sub\_d} \index{mp\_mul\_d} \index{mp\_div\_d} \index{mp\_mod\_d}
\begin{alltt}
int mp_add_d(mp_int *a, mp_digit b, mp_int *c);
int mp_sub_d(mp_int *a, mp_digit b, mp_int *c);
int mp_mul_d(mp_int *a, mp_digit b, mp_int *c);
int mp_div_d(mp_int *a, mp_digit b, mp_int *c, mp_digit *d);
int mp_mod_d(mp_int *a, mp_digit b, mp_digit *c);
\end{alltt}

These work like the full mp\_int capable variants except the second parameter $b$ is a mp\_digit.  These
functions fairly handy if you have to work with relatively small numbers since you will not have to allocate
an entire mp\_int to store a number like $1$ or $2$.

\documentclass[b5paper]{book}
\usepackage{hyperref}
\usepackage{makeidx}
\usepackage{amssymb}
\usepackage{color}
\usepackage{alltt}
\usepackage{graphicx}
\usepackage{layout}
\def\union{\cup}
\def\intersect{\cap}
\def\getsrandom{\stackrel{\rm R}{\gets}}
\def\cross{\times}
\def\cat{\hspace{0.5em} \| \hspace{0.5em}}
\def\catn{$\|$}
\def\divides{\hspace{0.3em} | \hspace{0.3em}}
\def\nequiv{\not\equiv}
\def\approx{\raisebox{0.2ex}{\mbox{\small $\sim$}}}
\def\lcm{{\rm lcm}}
\def\gcd{{\rm gcd}}
\def\log{{\rm log}}
\def\ord{{\rm ord}}
\def\abs{{\mathit abs}}
\def\rep{{\mathit rep}}
\def\mod{{\mathit\ mod\ }}
\renewcommand{\pmod}[1]{\ ({\rm mod\ }{#1})}
\newcommand{\floor}[1]{\left\lfloor{#1}\right\rfloor}
\newcommand{\ceil}[1]{\left\lceil{#1}\right\rceil}
\def\Or{{\rm\ or\ }}
\def\And{{\rm\ and\ }}
\def\iff{\hspace{1em}\Longleftrightarrow\hspace{1em}}
\def\implies{\Rightarrow}
\def\undefined{{\rm ``undefined"}}
\def\Proof{\vspace{1ex}\noindent {\bf Proof:}\hspace{1em}}
\let\oldphi\phi
\def\phi{\varphi}
\def\Pr{{\rm Pr}}
\newcommand{\str}[1]{{\mathbf{#1}}}
\def\F{{\mathbb F}}
\def\N{{\mathbb N}}
\def\Z{{\mathbb Z}}
\def\R{{\mathbb R}}
\def\C{{\mathbb C}}
\def\Q{{\mathbb Q}}
\definecolor{DGray}{gray}{0.5}
\newcommand{\emailaddr}[1]{\mbox{$<${#1}$>$}}
\def\twiddle{\raisebox{0.3ex}{\mbox{\tiny $\sim$}}}
\def\gap{\vspace{0.5ex}}
\makeindex
\begin{document}
\frontmatter
\pagestyle{empty}
\title{LibTomMath User Manual \\ v0.39}
\author{Tom St Denis \\ tomstdenis@iahu.ca}
\maketitle
This text, the library and the accompanying textbook are all hereby placed in the public domain.  This book has been 
formatted for B5 [176x250] paper using the \LaTeX{} {\em book} macro package.

\vspace{10cm}

\begin{flushright}Open Source.  Open Academia.  Open Minds.

\mbox{ }

Tom St Denis,

Ontario, Canada
\end{flushright}

\tableofcontents
\listoffigures
\mainmatter
\pagestyle{headings}
\chapter{Introduction}
\section{What is LibTomMath?}
LibTomMath is a library of source code which provides a series of efficient and carefully written functions for manipulating
large integer numbers.  It was written in portable ISO C source code so that it will build on any platform with a conforming
C compiler.  

In a nutshell the library was written from scratch with verbose comments to help instruct computer science students how
to implement ``bignum'' math.  However, the resulting code has proven to be very useful.  It has been used by numerous 
universities, commercial and open source software developers.  It has been used on a variety of platforms ranging from
Linux and Windows based x86 to ARM based Gameboys and PPC based MacOS machines.  

\section{License}
As of the v0.25 the library source code has been placed in the public domain with every new release.  As of the v0.28
release the textbook ``Implementing Multiple Precision Arithmetic'' has been placed in the public domain with every new
release as well.  This textbook is meant to compliment the project by providing a more solid walkthrough of the development
algorithms used in the library.

Since both\footnote{Note that the MPI files under mtest/ are copyrighted by Michael Fromberger.  They are not required to use LibTomMath.} are in the 
public domain everyone is entitled to do with them as they see fit.

\section{Building LibTomMath}

LibTomMath is meant to be very ``GCC friendly'' as it comes with a makefile well suited for GCC.  However, the library will
also build in MSVC, Borland C out of the box.  For any other ISO C compiler a makefile will have to be made by the end
developer.  

\subsection{Static Libraries}
To build as a static library for GCC issue the following
\begin{alltt}
make
\end{alltt}

command.  This will build the library and archive the object files in ``libtommath.a''.  Now you link against 
that and include ``tommath.h'' within your programs.  Alternatively to build with MSVC issue the following
\begin{alltt}
nmake -f makefile.msvc
\end{alltt}

This will build the library and archive the object files in ``tommath.lib''.  This has been tested with MSVC 
version 6.00 with service pack 5.  

\subsection{Shared Libraries}
To build as a shared library for GCC issue the following
\begin{alltt}
make -f makefile.shared
\end{alltt}
This requires the ``libtool'' package (common on most Linux/BSD systems).  It will build LibTomMath as both shared
and static then install (by default) into /usr/lib as well as install the header files in /usr/include.  The shared 
library (resource) will be called ``libtommath.la'' while the static library called ``libtommath.a''.  Generally 
you use libtool to link your application against the shared object.  

There is limited support for making a ``DLL'' in windows via the ``makefile.cygwin\_dll'' makefile.  It requires 
Cygwin to work with since it requires the auto-export/import functionality.  The resulting DLL and import library 
``libtommath.dll.a'' can be used to link LibTomMath dynamically to any Windows program using Cygwin.

\subsection{Testing}
To build the library and the test harness type

\begin{alltt}
make test
\end{alltt}

This will build the library, ``test'' and ``mtest/mtest''.  The ``test'' program will accept test vectors and verify the
results.  ``mtest/mtest'' will generate test vectors using the MPI library by Michael Fromberger\footnote{A copy of MPI
is included in the package}.  Simply pipe mtest into test using

\begin{alltt}
mtest/mtest | test
\end{alltt}

If you do not have a ``/dev/urandom'' style RNG source you will have to write your own PRNG and simply pipe that into 
mtest.  For example, if your PRNG program is called ``myprng'' simply invoke

\begin{alltt}
myprng | mtest/mtest | test
\end{alltt}

This will output a row of numbers that are increasing.  Each column is a different test (such as addition, multiplication, etc)
that is being performed.  The numbers represent how many times the test was invoked.  If an error is detected the program
will exit with a dump of the relevent numbers it was working with.

\section{Build Configuration}
LibTomMath can configured at build time in three phases we shall call ``depends'', ``tweaks'' and ``trims''.  
Each phase changes how the library is built and they are applied one after another respectively.  

To make the system more powerful you can tweak the build process.  Classes are defined in the file
``tommath\_superclass.h''.  By default, the symbol ``LTM\_ALL'' shall be defined which simply 
instructs the system to build all of the functions.  This is how LibTomMath used to be packaged.  This will give you 
access to every function LibTomMath offers.

However, there are cases where such a build is not optional.  For instance, you want to perform RSA operations.  You 
don't need the vast majority of the library to perform these operations.  Aside from LTM\_ALL there is 
another pre--defined class ``SC\_RSA\_1'' which works in conjunction with the RSA from LibTomCrypt.  Additional 
classes can be defined base on the need of the user.

\subsection{Build Depends}
In the file tommath\_class.h you will see a large list of C ``defines'' followed by a series of ``ifdefs''
which further define symbols.  All of the symbols (technically they're macros $\ldots$) represent a given C source
file.  For instance, BN\_MP\_ADD\_C represents the file ``bn\_mp\_add.c''.  When a define has been enabled the
function in the respective file will be compiled and linked into the library.  Accordingly when the define
is absent the file will not be compiled and not contribute any size to the library.

You will also note that the header tommath\_class.h is actually recursively included (it includes itself twice).  
This is to help resolve as many dependencies as possible.  In the last pass the symbol LTM\_LAST will be defined.  
This is useful for ``trims''.

\subsection{Build Tweaks}
A tweak is an algorithm ``alternative''.  For example, to provide tradeoffs (usually between size and space).
They can be enabled at any pass of the configuration phase.

\begin{small}
\begin{center}
\begin{tabular}{|l|l|}
\hline \textbf{Define} & \textbf{Purpose} \\
\hline BN\_MP\_DIV\_SMALL & Enables a slower, smaller and equally \\
                          & functional mp\_div() function \\
\hline
\end{tabular}
\end{center}
\end{small}

\subsection{Build Trims}
A trim is a manner of removing functionality from a function that is not required.  For instance, to perform
RSA cryptography you only require exponentiation with odd moduli so even moduli support can be safely removed.  
Build trims are meant to be defined on the last pass of the configuration which means they are to be defined
only if LTM\_LAST has been defined.

\subsubsection{Moduli Related}
\begin{small}
\begin{center}
\begin{tabular}{|l|l|}
\hline \textbf{Restriction} & \textbf{Undefine} \\
\hline Exponentiation with odd moduli only & BN\_S\_MP\_EXPTMOD\_C \\
                                           & BN\_MP\_REDUCE\_C \\
                                           & BN\_MP\_REDUCE\_SETUP\_C \\
                                           & BN\_S\_MP\_MUL\_HIGH\_DIGS\_C \\
                                           & BN\_FAST\_S\_MP\_MUL\_HIGH\_DIGS\_C \\
\hline Exponentiation with random odd moduli & (The above plus the following) \\
                                           & BN\_MP\_REDUCE\_2K\_C \\
                                           & BN\_MP\_REDUCE\_2K\_SETUP\_C \\
                                           & BN\_MP\_REDUCE\_IS\_2K\_C \\
                                           & BN\_MP\_DR\_IS\_MODULUS\_C \\
                                           & BN\_MP\_DR\_REDUCE\_C \\
                                           & BN\_MP\_DR\_SETUP\_C \\
\hline Modular inverse odd moduli only     & BN\_MP\_INVMOD\_SLOW\_C \\
\hline Modular inverse (both, smaller/slower) & BN\_FAST\_MP\_INVMOD\_C \\
\hline
\end{tabular}
\end{center}
\end{small}

\subsubsection{Operand Size Related}
\begin{small}
\begin{center}
\begin{tabular}{|l|l|}
\hline \textbf{Restriction} & \textbf{Undefine} \\
\hline Moduli $\le 2560$ bits              & BN\_MP\_MONTGOMERY\_REDUCE\_C \\
                                           & BN\_S\_MP\_MUL\_DIGS\_C \\
                                           & BN\_S\_MP\_MUL\_HIGH\_DIGS\_C \\
                                           & BN\_S\_MP\_SQR\_C \\
\hline Polynomial Schmolynomial            & BN\_MP\_KARATSUBA\_MUL\_C \\
                                           & BN\_MP\_KARATSUBA\_SQR\_C \\
                                           & BN\_MP\_TOOM\_MUL\_C \\ 
                                           & BN\_MP\_TOOM\_SQR\_C \\

\hline
\end{tabular}
\end{center}
\end{small}


\section{Purpose of LibTomMath}
Unlike  GNU MP (GMP) Library, LIP, OpenSSL or various other commercial kits (Miracl), LibTomMath was not written with 
bleeding edge performance in mind.  First and foremost LibTomMath was written to be entirely open.  Not only is the 
source code public domain (unlike various other GPL/etc licensed code), not only is the code freely downloadable but the
source code is also accessible for computer science students attempting to learn ``BigNum'' or multiple precision
arithmetic techniques. 

LibTomMath was written to be an instructive collection of source code.  This is why there are many comments, only one
function per source file and often I use a ``middle-road'' approach where I don't cut corners for an extra 2\% speed
increase.

Source code alone cannot really teach how the algorithms work which is why I also wrote a textbook that accompanies
the library (beat that!).

So you may be thinking ``should I use LibTomMath?'' and the answer is a definite maybe.  Let me tabulate what I think
are the pros and cons of LibTomMath by comparing it to the math routines from GnuPG\footnote{GnuPG v1.2.3 versus LibTomMath v0.28}.

\newpage\begin{figure}[here]
\begin{small}
\begin{center}
\begin{tabular}{|l|c|c|l|}
\hline \textbf{Criteria} & \textbf{Pro} & \textbf{Con} & \textbf{Notes} \\
\hline Few lines of code per file & X & & GnuPG $ = 300.9$, LibTomMath  $ = 71.97$ \\
\hline Commented function prototypes & X && GnuPG function names are cryptic. \\
\hline Speed && X & LibTomMath is slower.  \\
\hline Totally free & X & & GPL has unfavourable restrictions.\\
\hline Large function base & X & & GnuPG is barebones. \\
\hline Five modular reduction algorithms & X & & Faster modular exponentiation for a variety of moduli. \\
\hline Portable & X & & GnuPG requires configuration to build. \\
\hline
\end{tabular}
\end{center}
\end{small}
\caption{LibTomMath Valuation}
\end{figure}

It may seem odd to compare LibTomMath to GnuPG since the math in GnuPG is only a small portion of the entire application. 
However, LibTomMath was written with cryptography in mind.  It provides essentially all of the functions a cryptosystem
would require when working with large integers.  

So it may feel tempting to just rip the math code out of GnuPG (or GnuMP where it was taken from originally) in your
own application but I think there are reasons not to.  While LibTomMath is slower than libraries such as GnuMP it is
not normally significantly slower.  On x86 machines the difference is normally a factor of two when performing modular
exponentiations.  It depends largely on the processor, compiler and the moduli being used.

Essentially the only time you wouldn't use LibTomMath is when blazing speed is the primary concern.  However,
on the other side of the coin LibTomMath offers you a totally free (public domain) well structured math library
that is very flexible, complete and performs well in resource contrained environments.  Fast RSA for example can
be performed with as little as 8KB of ram for data (again depending on build options).  

\chapter{Getting Started with LibTomMath}
\section{Building Programs}
In order to use LibTomMath you must include ``tommath.h'' and link against the appropriate library file (typically 
libtommath.a).  There is no library initialization required and the entire library is thread safe.

\section{Return Codes}
There are three possible return codes a function may return.

\index{MP\_OKAY}\index{MP\_YES}\index{MP\_NO}\index{MP\_VAL}\index{MP\_MEM}
\begin{figure}[here!]
\begin{center}
\begin{small}
\begin{tabular}{|l|l|}
\hline \textbf{Code} & \textbf{Meaning} \\
\hline MP\_OKAY & The function succeeded. \\
\hline MP\_VAL  & The function input was invalid. \\
\hline MP\_MEM  & Heap memory exhausted. \\
\hline &\\
\hline MP\_YES  & Response is yes. \\
\hline MP\_NO   & Response is no. \\
\hline
\end{tabular}
\end{small}
\end{center}
\caption{Return Codes}
\end{figure}

The last two codes listed are not actually ``return'ed'' by a function.  They are placed in an integer (the caller must
provide the address of an integer it can store to) which the caller can access.  To convert one of the three return codes
to a string use the following function.

\index{mp\_error\_to\_string}
\begin{alltt}
char *mp_error_to_string(int code);
\end{alltt}

This will return a pointer to a string which describes the given error code.  It will not work for the return codes 
MP\_YES and MP\_NO.  

\section{Data Types}
The basic ``multiple precision integer'' type is known as the ``mp\_int'' within LibTomMath.  This data type is used to
organize all of the data required to manipulate the integer it represents.  Within LibTomMath it has been prototyped
as the following.

\index{mp\_int}
\begin{alltt}
typedef struct  \{
    int used, alloc, sign;
    mp_digit *dp;
\} mp_int;
\end{alltt}

Where ``mp\_digit'' is a data type that represents individual digits of the integer.  By default, an mp\_digit is the
ISO C ``unsigned long'' data type and each digit is $28-$bits long.  The mp\_digit type can be configured to suit other
platforms by defining the appropriate macros.  

All LTM functions that use the mp\_int type will expect a pointer to mp\_int structure.  You must allocate memory to
hold the structure itself by yourself (whether off stack or heap it doesn't matter).  The very first thing that must be
done to use an mp\_int is that it must be initialized.

\section{Function Organization}

The arithmetic functions of the library are all organized to have the same style prototype.  That is source operands
are passed on the left and the destination is on the right.  For instance,

\begin{alltt}
mp_add(&a, &b, &c);       /* c = a + b */
mp_mul(&a, &a, &c);       /* c = a * a */
mp_div(&a, &b, &c, &d);   /* c = [a/b], d = a mod b */
\end{alltt}

Another feature of the way the functions have been implemented is that source operands can be destination operands as well.
For instance,

\begin{alltt}
mp_add(&a, &b, &b);       /* b = a + b */
mp_div(&a, &b, &a, &c);   /* a = [a/b], c = a mod b */
\end{alltt}

This allows operands to be re-used which can make programming simpler.

\section{Initialization}
\subsection{Single Initialization}
A single mp\_int can be initialized with the ``mp\_init'' function. 

\index{mp\_init}
\begin{alltt}
int mp_init (mp_int * a);
\end{alltt}

This function expects a pointer to an mp\_int structure and will initialize the members of the structure so the mp\_int
represents the default integer which is zero.  If the functions returns MP\_OKAY then the mp\_int is ready to be used
by the other LibTomMath functions.

\begin{small} \begin{alltt}
int main(void)
\{
   mp_int number;
   int result;

   if ((result = mp_init(&number)) != MP_OKAY) \{
      printf("Error initializing the number.  \%s", 
             mp_error_to_string(result));
      return EXIT_FAILURE;
   \}
 
   /* use the number */

   return EXIT_SUCCESS;
\}
\end{alltt} \end{small}

\subsection{Single Free}
When you are finished with an mp\_int it is ideal to return the heap it used back to the system.  The following function 
provides this functionality.

\index{mp\_clear}
\begin{alltt}
void mp_clear (mp_int * a);
\end{alltt}

The function expects a pointer to a previously initialized mp\_int structure and frees the heap it uses.  It sets the 
pointer\footnote{The ``dp'' member.} within the mp\_int to \textbf{NULL} which is used to prevent double free situations. 
Is is legal to call mp\_clear() twice on the same mp\_int in a row.  

\begin{small} \begin{alltt}
int main(void)
\{
   mp_int number;
   int result;

   if ((result = mp_init(&number)) != MP_OKAY) \{
      printf("Error initializing the number.  \%s", 
             mp_error_to_string(result));
      return EXIT_FAILURE;
   \}
 
   /* use the number */

   /* We're done with it. */
   mp_clear(&number);

   return EXIT_SUCCESS;
\}
\end{alltt} \end{small}

\subsection{Multiple Initializations}
Certain algorithms require more than one large integer.  In these instances it is ideal to initialize all of the mp\_int
variables in an ``all or nothing'' fashion.  That is, they are either all initialized successfully or they are all
not initialized.

The  mp\_init\_multi() function provides this functionality.

\index{mp\_init\_multi} \index{mp\_clear\_multi}
\begin{alltt}
int mp_init_multi(mp_int *mp, ...);
\end{alltt}

It accepts a \textbf{NULL} terminated list of pointers to mp\_int structures.  It will attempt to initialize them all
at once.  If the function returns MP\_OKAY then all of the mp\_int variables are ready to use, otherwise none of them
are available for use.  A complementary mp\_clear\_multi() function allows multiple mp\_int variables to be free'd 
from the heap at the same time.  

\begin{small} \begin{alltt}
int main(void)
\{
   mp_int num1, num2, num3;
   int result;

   if ((result = mp_init_multi(&num1, 
                               &num2,
                               &num3, NULL)) != MP\_OKAY) \{      
      printf("Error initializing the numbers.  \%s", 
             mp_error_to_string(result));
      return EXIT_FAILURE;
   \}
 
   /* use the numbers */

   /* We're done with them. */
   mp_clear_multi(&num1, &num2, &num3, NULL);

   return EXIT_SUCCESS;
\}
\end{alltt} \end{small}

\subsection{Other Initializers}
To initialized and make a copy of an mp\_int the mp\_init\_copy() function has been provided.  

\index{mp\_init\_copy}
\begin{alltt}
int mp_init_copy (mp_int * a, mp_int * b);
\end{alltt}

This function will initialize $a$ and make it a copy of $b$ if all goes well.

\begin{small} \begin{alltt}
int main(void)
\{
   mp_int num1, num2;
   int result;

   /* initialize and do work on num1 ... */

   /* We want a copy of num1 in num2 now */
   if ((result = mp_init_copy(&num2, &num1)) != MP_OKAY) \{
     printf("Error initializing the copy.  \%s", 
             mp_error_to_string(result));
      return EXIT_FAILURE;
   \}
 
   /* now num2 is ready and contains a copy of num1 */

   /* We're done with them. */
   mp_clear_multi(&num1, &num2, NULL);

   return EXIT_SUCCESS;
\}
\end{alltt} \end{small}

Another less common initializer is mp\_init\_size() which allows the user to initialize an mp\_int with a given
default number of digits.  By default, all initializers allocate \textbf{MP\_PREC} digits.  This function lets
you override this behaviour.

\index{mp\_init\_size}
\begin{alltt}
int mp_init_size (mp_int * a, int size);
\end{alltt}

The $size$ parameter must be greater than zero.  If the function succeeds the mp\_int $a$ will be initialized
to have $size$ digits (which are all initially zero).  

\begin{small} \begin{alltt}
int main(void)
\{
   mp_int number;
   int result;

   /* we need a 60-digit number */
   if ((result = mp_init_size(&number, 60)) != MP_OKAY) \{
      printf("Error initializing the number.  \%s", 
             mp_error_to_string(result));
      return EXIT_FAILURE;
   \}
 
   /* use the number */

   return EXIT_SUCCESS;
\}
\end{alltt} \end{small}

\section{Maintenance Functions}

\subsection{Reducing Memory Usage}
When an mp\_int is in a state where it won't be changed again\footnote{A Diffie-Hellman modulus for instance.} excess
digits can be removed to return memory to the heap with the mp\_shrink() function.

\index{mp\_shrink}
\begin{alltt}
int mp_shrink (mp_int * a);
\end{alltt}

This will remove excess digits of the mp\_int $a$.  If the operation fails the mp\_int should be intact without the
excess digits being removed.  Note that you can use a shrunk mp\_int in further computations, however, such operations
will require heap operations which can be slow.  It is not ideal to shrink mp\_int variables that you will further
modify in the system (unless you are seriously low on memory).  

\begin{small} \begin{alltt}
int main(void)
\{
   mp_int number;
   int result;

   if ((result = mp_init(&number)) != MP_OKAY) \{
      printf("Error initializing the number.  \%s", 
             mp_error_to_string(result));
      return EXIT_FAILURE;
   \}
 
   /* use the number [e.g. pre-computation]  */

   /* We're done with it for now. */
   if ((result = mp_shrink(&number)) != MP_OKAY) \{
      printf("Error shrinking the number.  \%s", 
             mp_error_to_string(result));
      return EXIT_FAILURE;
   \}

   /* use it .... */


   /* we're done with it. */ 
   mp_clear(&number);

   return EXIT_SUCCESS;
\}
\end{alltt} \end{small}

\subsection{Adding additional digits}

Within the mp\_int structure are two parameters which control the limitations of the array of digits that represent
the integer the mp\_int is meant to equal.   The \textit{used} parameter dictates how many digits are significant, that is,
contribute to the value of the mp\_int.  The \textit{alloc} parameter dictates how many digits are currently available in
the array.  If you need to perform an operation that requires more digits you will have to mp\_grow() the mp\_int to
your desired size.  

\index{mp\_grow}
\begin{alltt}
int mp_grow (mp_int * a, int size);
\end{alltt}

This will grow the array of digits of $a$ to $size$.  If the \textit{alloc} parameter is already bigger than
$size$ the function will not do anything.

\begin{small} \begin{alltt}
int main(void)
\{
   mp_int number;
   int result;

   if ((result = mp_init(&number)) != MP_OKAY) \{
      printf("Error initializing the number.  \%s", 
             mp_error_to_string(result));
      return EXIT_FAILURE;
   \}
 
   /* use the number */

   /* We need to add 20 digits to the number  */
   if ((result = mp_grow(&number, number.alloc + 20)) != MP_OKAY) \{
      printf("Error growing the number.  \%s", 
             mp_error_to_string(result));
      return EXIT_FAILURE;
   \}


   /* use the number */

   /* we're done with it. */ 
   mp_clear(&number);

   return EXIT_SUCCESS;
\}
\end{alltt} \end{small}

\chapter{Basic Operations}
\section{Small Constants}
Setting mp\_ints to small constants is a relatively common operation.  To accomodate these instances there are two
small constant assignment functions.  The first function is used to set a single digit constant while the second sets
an ISO C style ``unsigned long'' constant.  The reason for both functions is efficiency.  Setting a single digit is quick but the
domain of a digit can change (it's always at least $0 \ldots 127$).  

\subsection{Single Digit}

Setting a single digit can be accomplished with the following function.

\index{mp\_set}
\begin{alltt}
void mp_set (mp_int * a, mp_digit b);
\end{alltt}

This will zero the contents of $a$ and make it represent an integer equal to the value of $b$.  Note that this
function has a return type of \textbf{void}.  It cannot cause an error so it is safe to assume the function
succeeded.

\begin{small} \begin{alltt}
int main(void)
\{
   mp_int number;
   int result;

   if ((result = mp_init(&number)) != MP_OKAY) \{
      printf("Error initializing the number.  \%s", 
             mp_error_to_string(result));
      return EXIT_FAILURE;
   \}
 
   /* set the number to 5 */
   mp_set(&number, 5);

   /* we're done with it. */ 
   mp_clear(&number);

   return EXIT_SUCCESS;
\}
\end{alltt} \end{small}

\subsection{Long Constants}

To set a constant that is the size of an ISO C ``unsigned long'' and larger than a single digit the following function 
can be used.

\index{mp\_set\_int}
\begin{alltt}
int mp_set_int (mp_int * a, unsigned long b);
\end{alltt}

This will assign the value of the 32-bit variable $b$ to the mp\_int $a$.  Unlike mp\_set() this function will always
accept a 32-bit input regardless of the size of a single digit.  However, since the value may span several digits 
this function can fail if it runs out of heap memory.

To get the ``unsigned long'' copy of an mp\_int the following function can be used.

\index{mp\_get\_int}
\begin{alltt}
unsigned long mp_get_int (mp_int * a);
\end{alltt}

This will return the 32 least significant bits of the mp\_int $a$.  

\begin{small} \begin{alltt}
int main(void)
\{
   mp_int number;
   int result;

   if ((result = mp_init(&number)) != MP_OKAY) \{
      printf("Error initializing the number.  \%s", 
             mp_error_to_string(result));
      return EXIT_FAILURE;
   \}
 
   /* set the number to 654321 (note this is bigger than 127) */
   if ((result = mp_set_int(&number, 654321)) != MP_OKAY) \{
      printf("Error setting the value of the number.  \%s", 
             mp_error_to_string(result));
      return EXIT_FAILURE;
   \}

   printf("number == \%lu", mp_get_int(&number));

   /* we're done with it. */ 
   mp_clear(&number);

   return EXIT_SUCCESS;
\}
\end{alltt} \end{small}

This should output the following if the program succeeds.

\begin{alltt}
number == 654321
\end{alltt}

\subsection{Initialize and Setting Constants}
To both initialize and set small constants the following two functions are available.
\index{mp\_init\_set} \index{mp\_init\_set\_int}
\begin{alltt}
int mp_init_set (mp_int * a, mp_digit b);
int mp_init_set_int (mp_int * a, unsigned long b);
\end{alltt}

Both functions work like the previous counterparts except they first mp\_init $a$ before setting the values.  

\begin{alltt}
int main(void)
\{
   mp_int number1, number2;
   int    result;

   /* initialize and set a single digit */
   if ((result = mp_init_set(&number1, 100)) != MP_OKAY) \{
      printf("Error setting number1: \%s", 
             mp_error_to_string(result));
      return EXIT_FAILURE;
   \}             

   /* initialize and set a long */
   if ((result = mp_init_set_int(&number2, 1023)) != MP_OKAY) \{
      printf("Error setting number2: \%s", 
             mp_error_to_string(result));
      return EXIT_FAILURE;
   \}

   /* display */
   printf("Number1, Number2 == \%lu, \%lu",
          mp_get_int(&number1), mp_get_int(&number2));

   /* clear */
   mp_clear_multi(&number1, &number2, NULL);

   return EXIT_SUCCESS;
\}
\end{alltt}

If this program succeeds it shall output.
\begin{alltt}
Number1, Number2 == 100, 1023
\end{alltt}

\section{Comparisons}

Comparisons in LibTomMath are always performed in a ``left to right'' fashion.  There are three possible return codes
for any comparison.

\index{MP\_GT} \index{MP\_EQ} \index{MP\_LT}
\begin{figure}[here]
\begin{center}
\begin{tabular}{|c|c|}
\hline \textbf{Result Code} & \textbf{Meaning} \\
\hline MP\_GT & $a > b$ \\
\hline MP\_EQ & $a = b$ \\
\hline MP\_LT & $a < b$ \\
\hline
\end{tabular}
\end{center}
\caption{Comparison Codes for $a, b$}
\label{fig:CMP}
\end{figure}

In figure \ref{fig:CMP} two integers $a$ and $b$ are being compared.  In this case $a$ is said to be ``to the left'' of 
$b$.  

\subsection{Unsigned comparison}

An unsigned comparison considers only the digits themselves and not the associated \textit{sign} flag of the 
mp\_int structures.  This is analogous to an absolute comparison.  The function mp\_cmp\_mag() will compare two
mp\_int variables based on their digits only. 

\index{mp\_cmp\_mag}
\begin{alltt}
int mp_cmp_mag(mp_int * a, mp_int * b);
\end{alltt}
This will compare $a$ to $b$ placing $a$ to the left of $b$.  This function cannot fail and will return one of the
three compare codes listed in figure \ref{fig:CMP}.

\begin{small} \begin{alltt}
int main(void)
\{
   mp_int number1, number2;
   int result;

   if ((result = mp_init_multi(&number1, &number2, NULL)) != MP_OKAY) \{
      printf("Error initializing the numbers.  \%s", 
             mp_error_to_string(result));
      return EXIT_FAILURE;
   \}
 
   /* set the number1 to 5 */
   mp_set(&number1, 5);
  
   /* set the number2 to -6 */
   mp_set(&number2, 6);
   if ((result = mp_neg(&number2, &number2)) != MP_OKAY) \{
      printf("Error negating number2.  \%s", 
             mp_error_to_string(result));
      return EXIT_FAILURE;
   \}

   switch(mp_cmp_mag(&number1, &number2)) \{
       case MP_GT:  printf("|number1| > |number2|"); break;
       case MP_EQ:  printf("|number1| = |number2|"); break;
       case MP_LT:  printf("|number1| < |number2|"); break;
   \}

   /* we're done with it. */ 
   mp_clear_multi(&number1, &number2, NULL);

   return EXIT_SUCCESS;
\}
\end{alltt} \end{small}

If this program\footnote{This function uses the mp\_neg() function which is discussed in section \ref{sec:NEG}.} completes 
successfully it should print the following.

\begin{alltt}
|number1| < |number2|
\end{alltt}

This is because $\vert -6 \vert = 6$ and obviously $5 < 6$.

\subsection{Signed comparison}

To compare two mp\_int variables based on their signed value the mp\_cmp() function is provided.

\index{mp\_cmp}
\begin{alltt}
int mp_cmp(mp_int * a, mp_int * b);
\end{alltt}

This will compare $a$ to the left of $b$.  It will first compare the signs of the two mp\_int variables.  If they
differ it will return immediately based on their signs.  If the signs are equal then it will compare the digits
individually.  This function will return one of the compare conditions codes listed in figure \ref{fig:CMP}.

\begin{small} \begin{alltt}
int main(void)
\{
   mp_int number1, number2;
   int result;

   if ((result = mp_init_multi(&number1, &number2, NULL)) != MP_OKAY) \{
      printf("Error initializing the numbers.  \%s", 
             mp_error_to_string(result));
      return EXIT_FAILURE;
   \}
 
   /* set the number1 to 5 */
   mp_set(&number1, 5);
  
   /* set the number2 to -6 */
   mp_set(&number2, 6);
   if ((result = mp_neg(&number2, &number2)) != MP_OKAY) \{
      printf("Error negating number2.  \%s", 
             mp_error_to_string(result));
      return EXIT_FAILURE;
   \}

   switch(mp_cmp(&number1, &number2)) \{
       case MP_GT:  printf("number1 > number2"); break;
       case MP_EQ:  printf("number1 = number2"); break;
       case MP_LT:  printf("number1 < number2"); break;
   \}

   /* we're done with it. */ 
   mp_clear_multi(&number1, &number2, NULL);

   return EXIT_SUCCESS;
\}
\end{alltt} \end{small}

If this program\footnote{This function uses the mp\_neg() function which is discussed in section \ref{sec:NEG}.} completes 
successfully it should print the following.

\begin{alltt}
number1 > number2
\end{alltt}

\subsection{Single Digit}

To compare a single digit against an mp\_int the following function has been provided.

\index{mp\_cmp\_d}
\begin{alltt}
int mp_cmp_d(mp_int * a, mp_digit b);
\end{alltt}

This will compare $a$ to the left of $b$ using a signed comparison.  Note that it will always treat $b$ as 
positive.  This function is rather handy when you have to compare against small values such as $1$ (which often
comes up in cryptography).  The function cannot fail and will return one of the tree compare condition codes
listed in figure \ref{fig:CMP}.


\begin{small} \begin{alltt}
int main(void)
\{
   mp_int number;
   int result;

   if ((result = mp_init(&number)) != MP_OKAY) \{
      printf("Error initializing the number.  \%s", 
             mp_error_to_string(result));
      return EXIT_FAILURE;
   \}
 
   /* set the number to 5 */
   mp_set(&number, 5);

   switch(mp_cmp_d(&number, 7)) \{
       case MP_GT:  printf("number > 7"); break;
       case MP_EQ:  printf("number = 7"); break;
       case MP_LT:  printf("number < 7"); break;
   \}

   /* we're done with it. */ 
   mp_clear(&number);

   return EXIT_SUCCESS;
\}
\end{alltt} \end{small}

If this program functions properly it will print out the following.

\begin{alltt}
number < 7
\end{alltt}

\section{Logical Operations}

Logical operations are operations that can be performed either with simple shifts or boolean operators such as
AND, XOR and OR directly.  These operations are very quick.

\subsection{Multiplication by two}

Multiplications and divisions by any power of two can be performed with quick logical shifts either left or
right depending on the operation.  

When multiplying or dividing by two a special case routine can be used which are as follows.
\index{mp\_mul\_2} \index{mp\_div\_2}
\begin{alltt}
int mp_mul_2(mp_int * a, mp_int * b);
int mp_div_2(mp_int * a, mp_int * b);
\end{alltt}

The former will assign twice $a$ to $b$ while the latter will assign half $a$ to $b$.  These functions are fast
since the shift counts and maskes are hardcoded into the routines.

\begin{small} \begin{alltt}
int main(void)
\{
   mp_int number;
   int result;

   if ((result = mp_init(&number)) != MP_OKAY) \{
      printf("Error initializing the number.  \%s", 
             mp_error_to_string(result));
      return EXIT_FAILURE;
   \}
 
   /* set the number to 5 */
   mp_set(&number, 5);

   /* multiply by two */
   if ((result = mp\_mul\_2(&number, &number)) != MP_OKAY) \{
      printf("Error multiplying the number.  \%s", 
             mp_error_to_string(result));
      return EXIT_FAILURE;
   \}
   switch(mp_cmp_d(&number, 7)) \{
       case MP_GT:  printf("2*number > 7"); break;
       case MP_EQ:  printf("2*number = 7"); break;
       case MP_LT:  printf("2*number < 7"); break;
   \}

   /* now divide by two */
   if ((result = mp\_div\_2(&number, &number)) != MP_OKAY) \{
      printf("Error dividing the number.  \%s", 
             mp_error_to_string(result));
      return EXIT_FAILURE;
   \}
   switch(mp_cmp_d(&number, 7)) \{
       case MP_GT:  printf("2*number/2 > 7"); break;
       case MP_EQ:  printf("2*number/2 = 7"); break;
       case MP_LT:  printf("2*number/2 < 7"); break;
   \}

   /* we're done with it. */ 
   mp_clear(&number);

   return EXIT_SUCCESS;
\}
\end{alltt} \end{small}

If this program is successful it will print out the following text.

\begin{alltt}
2*number > 7
2*number/2 < 7
\end{alltt}

Since $10 > 7$ and $5 < 7$.  To multiply by a power of two the following function can be used.

\index{mp\_mul\_2d}
\begin{alltt}
int mp_mul_2d(mp_int * a, int b, mp_int * c);
\end{alltt}

This will multiply $a$ by $2^b$ and store the result in ``c''.  If the value of $b$ is less than or equal to 
zero the function will copy $a$ to ``c'' without performing any further actions.  

To divide by a power of two use the following.

\index{mp\_div\_2d}
\begin{alltt}
int mp_div_2d (mp_int * a, int b, mp_int * c, mp_int * d);
\end{alltt}
Which will divide $a$ by $2^b$, store the quotient in ``c'' and the remainder in ``d'.  If $b \le 0$ then the
function simply copies $a$ over to ``c'' and zeroes $d$.  The variable $d$ may be passed as a \textbf{NULL}
value to signal that the remainder is not desired.

\subsection{Polynomial Basis Operations}

Strictly speaking the organization of the integers within the mp\_int structures is what is known as a 
``polynomial basis''.  This simply means a field element is stored by divisions of a radix.  For example, if
$f(x) = \sum_{i=0}^{k} y_ix^k$ for any vector $\vec y$ then the array of digits in $\vec y$ are said to be 
the polynomial basis representation of $z$ if $f(\beta) = z$ for a given radix $\beta$.  

To multiply by the polynomial $g(x) = x$ all you have todo is shift the digits of the basis left one place.  The
following function provides this operation.

\index{mp\_lshd}
\begin{alltt}
int mp_lshd (mp_int * a, int b);
\end{alltt}

This will multiply $a$ in place by $x^b$ which is equivalent to shifting the digits left $b$ places and inserting zeroes
in the least significant digits.  Similarly to divide by a power of $x$ the following function is provided.

\index{mp\_rshd}
\begin{alltt}
void mp_rshd (mp_int * a, int b)
\end{alltt}
This will divide $a$ in place by $x^b$ and discard the remainder.  This function cannot fail as it performs the operations
in place and no new digits are required to complete it.

\subsection{AND, OR and XOR Operations}

While AND, OR and XOR operations are not typical ``bignum functions'' they can be useful in several instances.  The
three functions are prototyped as follows.

\index{mp\_or} \index{mp\_and} \index{mp\_xor}
\begin{alltt}
int mp_or  (mp_int * a, mp_int * b, mp_int * c);
int mp_and (mp_int * a, mp_int * b, mp_int * c);
int mp_xor (mp_int * a, mp_int * b, mp_int * c);
\end{alltt}

Which compute $c = a \odot b$ where $\odot$ is one of OR, AND or XOR.  

\section{Addition and Subtraction}

To compute an addition or subtraction the following two functions can be used.

\index{mp\_add} \index{mp\_sub}
\begin{alltt}
int mp_add (mp_int * a, mp_int * b, mp_int * c);
int mp_sub (mp_int * a, mp_int * b, mp_int * c)
\end{alltt}

Which perform $c = a \odot b$ where $\odot$ is one of signed addition or subtraction.  The operations are fully sign
aware.

\section{Sign Manipulation}
\subsection{Negation}
\label{sec:NEG}
Simple integer negation can be performed with the following.

\index{mp\_neg}
\begin{alltt}
int mp_neg (mp_int * a, mp_int * b);
\end{alltt}

Which assigns $-a$ to $b$.  

\subsection{Absolute}
Simple integer absolutes can be performed with the following.

\index{mp\_neg}
\begin{alltt}
int mp_abs (mp_int * a, mp_int * b);
\end{alltt}

Which assigns $\vert a \vert$ to $b$.  

\section{Integer Division and Remainder}
To perform a complete and general integer division with remainder use the following function.

\index{mp\_div}
\begin{alltt}
int mp_div (mp_int * a, mp_int * b, mp_int * c, mp_int * d);
\end{alltt}
                                                        
This divides $a$ by $b$ and stores the quotient in $c$ and $d$.  The signed quotient is computed such that 
$bc + d = a$.  Note that either of $c$ or $d$ can be set to \textbf{NULL} if their value is not required.  If 
$b$ is zero the function returns \textbf{MP\_VAL}.  


\chapter{Multiplication and Squaring}
\section{Multiplication}
A full signed integer multiplication can be performed with the following.
\index{mp\_mul}
\begin{alltt}
int mp_mul (mp_int * a, mp_int * b, mp_int * c);
\end{alltt}
Which assigns the full signed product $ab$ to $c$.  This function actually breaks into one of four cases which are 
specific multiplication routines optimized for given parameters.  First there are the Toom-Cook multiplications which
should only be used with very large inputs.  This is followed by the Karatsuba multiplications which are for moderate
sized inputs.  Then followed by the Comba and baseline multipliers.

Fortunately for the developer you don't really need to know this unless you really want to fine tune the system.  mp\_mul()
will determine on its own\footnote{Some tweaking may be required.} what routine to use automatically when it is called.

\begin{alltt}
int main(void)
\{
   mp_int number1, number2;
   int result;

   /* Initialize the numbers */
   if ((result = mp_init_multi(&number1, 
                               &number2, NULL)) != MP_OKAY) \{
      printf("Error initializing the numbers.  \%s", 
             mp_error_to_string(result));
      return EXIT_FAILURE;
   \}

   /* set the terms */
   if ((result = mp_set_int(&number, 257)) != MP_OKAY) \{
      printf("Error setting number1.  \%s", 
             mp_error_to_string(result));
      return EXIT_FAILURE;
   \}
 
   if ((result = mp_set_int(&number2, 1023)) != MP_OKAY) \{
      printf("Error setting number2.  \%s", 
             mp_error_to_string(result));
      return EXIT_FAILURE;
   \}

   /* multiply them */
   if ((result = mp_mul(&number1, &number2,
                        &number1)) != MP_OKAY) \{
      printf("Error multiplying terms.  \%s", 
             mp_error_to_string(result));
      return EXIT_FAILURE;
   \}

   /* display */
   printf("number1 * number2 == \%lu", mp_get_int(&number1));

   /* free terms and return */
   mp_clear_multi(&number1, &number2, NULL);

   return EXIT_SUCCESS;
\}
\end{alltt}   

If this program succeeds it shall output the following.

\begin{alltt}
number1 * number2 == 262911
\end{alltt}

\section{Squaring}
Since squaring can be performed faster than multiplication it is performed it's own function instead of just using
mp\_mul().

\index{mp\_sqr}
\begin{alltt}
int mp_sqr (mp_int * a, mp_int * b);
\end{alltt}

Will square $a$ and store it in $b$.  Like the case of multiplication there are four different squaring
algorithms all which can be called from mp\_sqr().  It is ideal to use mp\_sqr over mp\_mul when squaring terms because
of the speed difference.  

\section{Tuning Polynomial Basis Routines}

Both of the Toom-Cook and Karatsuba multiplication algorithms are faster than the traditional $O(n^2)$ approach that
the Comba and baseline algorithms use.  At $O(n^{1.464973})$ and $O(n^{1.584962})$ running times respectively they require 
considerably less work.  For example, a 10000-digit multiplication would take roughly 724,000 single precision
multiplications with Toom-Cook or 100,000,000 single precision multiplications with the standard Comba (a factor
of 138).

So why not always use Karatsuba or Toom-Cook?   The simple answer is that they have so much overhead that they're not
actually faster than Comba until you hit distinct  ``cutoff'' points.  For Karatsuba with the default configuration, 
GCC 3.3.1 and an Athlon XP processor the cutoff point is roughly 110 digits (about 70 for the Intel P4).  That is, at 
110 digits Karatsuba and Comba multiplications just about break even and for 110+ digits Karatsuba is faster.

Toom-Cook has incredible overhead and is probably only useful for very large inputs.  So far no known cutoff points 
exist and for the most part I just set the cutoff points very high to make sure they're not called.

A demo program in the ``etc/'' directory of the project called ``tune.c'' can be used to find the cutoff points.  This
can be built with GCC as follows

\begin{alltt}
make XXX
\end{alltt}
Where ``XXX'' is one of the following entries from the table \ref{fig:tuning}.

\begin{figure}[here]
\begin{center}
\begin{small}
\begin{tabular}{|l|l|}
\hline \textbf{Value of XXX} & \textbf{Meaning} \\
\hline tune & Builds portable tuning application \\
\hline tune86 & Builds x86 (pentium and up) program for COFF \\
\hline tune86c & Builds x86 program for Cygwin \\
\hline tune86l & Builds x86 program for Linux (ELF format) \\
\hline
\end{tabular}
\end{small}
\end{center}
\caption{Build Names for Tuning Programs}
\label{fig:tuning}
\end{figure}

When the program is running it will output a series of measurements for different cutoff points.  It will first find
good Karatsuba squaring and multiplication points.  Then it proceeds to find Toom-Cook points.  Note that the Toom-Cook
tuning takes a very long time as the cutoff points are likely to be very high.

\chapter{Modular Reduction}

Modular reduction is process of taking the remainder of one quantity divided by another.  Expressed 
as (\ref{eqn:mod}) the modular reduction is equivalent to the remainder of $b$ divided by $c$.  

\begin{equation}
a \equiv b \mbox{ (mod }c\mbox{)}
\label{eqn:mod}
\end{equation}

Of particular interest to cryptography are reductions where $b$ is limited to the range $0 \le b < c^2$ since particularly 
fast reduction algorithms can be written for the limited range.  

Note that one of the four optimized reduction algorithms are automatically chosen in the modular exponentiation
algorithm mp\_exptmod when an appropriate modulus is detected.  

\section{Straight Division}
In order to effect an arbitrary modular reduction the following algorithm is provided.

\index{mp\_mod}
\begin{alltt}
int mp_mod(mp_int *a, mp_int *b, mp_int *c);
\end{alltt}

This reduces $a$ modulo $b$ and stores the result in $c$.  The sign of $c$ shall agree with the sign 
of $b$.  This algorithm accepts an input $a$ of any range and is not limited by $0 \le a < b^2$.

\section{Barrett Reduction}

Barrett reduction is a generic optimized reduction algorithm that requires pre--computation to achieve
a decent speedup over straight division.  First a $\mu$ value must be precomputed with the following function.

\index{mp\_reduce\_setup}
\begin{alltt}
int mp_reduce_setup(mp_int *a, mp_int *b);
\end{alltt}

Given a modulus in $b$ this produces the required $\mu$ value in $a$.  For any given modulus this only has to
be computed once.  Modular reduction can now be performed with the following.

\index{mp\_reduce}
\begin{alltt}
int mp_reduce(mp_int *a, mp_int *b, mp_int *c);
\end{alltt}

This will reduce $a$ in place modulo $b$ with the precomputed $\mu$ value in $c$.  $a$ must be in the range
$0 \le a < b^2$.

\begin{alltt}
int main(void)
\{
   mp_int   a, b, c, mu;
   int      result;

   /* initialize a,b to desired values, mp_init mu, 
    * c and set c to 1...we want to compute a^3 mod b 
    */

   /* get mu value */
   if ((result = mp_reduce_setup(&mu, b)) != MP_OKAY) \{
      printf("Error getting mu.  \%s", 
             mp_error_to_string(result));
      return EXIT_FAILURE;
   \}

   /* square a to get c = a^2 */
   if ((result = mp_sqr(&a, &c)) != MP_OKAY) \{
      printf("Error squaring.  \%s", 
             mp_error_to_string(result));
      return EXIT_FAILURE;
   \}

   /* now reduce `c' modulo b */
   if ((result = mp_reduce(&c, &b, &mu)) != MP_OKAY) \{
      printf("Error reducing.  \%s", 
             mp_error_to_string(result));
      return EXIT_FAILURE;
   \}
   
   /* multiply a to get c = a^3 */
   if ((result = mp_mul(&a, &c, &c)) != MP_OKAY) \{
      printf("Error reducing.  \%s", 
             mp_error_to_string(result));
      return EXIT_FAILURE;
   \}

   /* now reduce `c' modulo b  */
   if ((result = mp_reduce(&c, &b, &mu)) != MP_OKAY) \{
      printf("Error reducing.  \%s", 
             mp_error_to_string(result));
      return EXIT_FAILURE;
   \}
  
   /* c now equals a^3 mod b */

   return EXIT_SUCCESS;
\}
\end{alltt} 

This program will calculate $a^3 \mbox{ mod }b$ if all the functions succeed.  

\section{Montgomery Reduction}

Montgomery is a specialized reduction algorithm for any odd moduli.  Like Barrett reduction a pre--computation
step is required.  This is accomplished with the following.

\index{mp\_montgomery\_setup}
\begin{alltt}
int mp_montgomery_setup(mp_int *a, mp_digit *mp);
\end{alltt}

For the given odd moduli $a$ the precomputation value is placed in $mp$.  The reduction is computed with the 
following.

\index{mp\_montgomery\_reduce}
\begin{alltt}
int mp_montgomery_reduce(mp_int *a, mp_int *m, mp_digit mp);
\end{alltt}
This reduces $a$ in place modulo $m$ with the pre--computed value $mp$.   $a$ must be in the range
$0 \le a < b^2$.

Montgomery reduction is faster than Barrett reduction for moduli smaller than the ``comba'' limit.  With the default
setup for instance, the limit is $127$ digits ($3556$--bits).   Note that this function is not limited to
$127$ digits just that it falls back to a baseline algorithm after that point.  

An important observation is that this reduction does not return $a \mbox{ mod }m$ but $aR^{-1} \mbox{ mod }m$ 
where $R = \beta^n$, $n$ is the n number of digits in $m$ and $\beta$ is radix used (default is $2^{28}$).  

To quickly calculate $R$ the following function was provided.

\index{mp\_montgomery\_calc\_normalization}
\begin{alltt}
int mp_montgomery_calc_normalization(mp_int *a, mp_int *b);
\end{alltt}
Which calculates $a = R$ for the odd moduli $b$ without using multiplication or division.  

The normal modus operandi for Montgomery reductions is to normalize the integers before entering the system.  For
example, to calculate $a^3 \mbox { mod }b$ using Montgomery reduction the value of $a$ can be normalized by
multiplying it by $R$.  Consider the following code snippet.

\begin{alltt}
int main(void)
\{
   mp_int   a, b, c, R;
   mp_digit mp;
   int      result;

   /* initialize a,b to desired values, 
    * mp_init R, c and set c to 1.... 
    */

   /* get normalization */
   if ((result = mp_montgomery_calc_normalization(&R, b)) != MP_OKAY) \{
      printf("Error getting norm.  \%s", 
             mp_error_to_string(result));
      return EXIT_FAILURE;
   \}

   /* get mp value */
   if ((result = mp_montgomery_setup(&c, &mp)) != MP_OKAY) \{
      printf("Error setting up montgomery.  \%s", 
             mp_error_to_string(result));
      return EXIT_FAILURE;
   \}

   /* normalize `a' so now a is equal to aR */
   if ((result = mp_mulmod(&a, &R, &b, &a)) != MP_OKAY) \{
      printf("Error computing aR.  \%s", 
             mp_error_to_string(result));
      return EXIT_FAILURE;
   \}

   /* square a to get c = a^2R^2 */
   if ((result = mp_sqr(&a, &c)) != MP_OKAY) \{
      printf("Error squaring.  \%s", 
             mp_error_to_string(result));
      return EXIT_FAILURE;
   \}

   /* now reduce `c' back down to c = a^2R^2 * R^-1 == a^2R */
   if ((result = mp_montgomery_reduce(&c, &b, mp)) != MP_OKAY) \{
      printf("Error reducing.  \%s", 
             mp_error_to_string(result));
      return EXIT_FAILURE;
   \}
   
   /* multiply a to get c = a^3R^2 */
   if ((result = mp_mul(&a, &c, &c)) != MP_OKAY) \{
      printf("Error reducing.  \%s", 
             mp_error_to_string(result));
      return EXIT_FAILURE;
   \}

   /* now reduce `c' back down to c = a^3R^2 * R^-1 == a^3R */
   if ((result = mp_montgomery_reduce(&c, &b, mp)) != MP_OKAY) \{
      printf("Error reducing.  \%s", 
             mp_error_to_string(result));
      return EXIT_FAILURE;
   \}
   
   /* now reduce (again) `c' back down to c = a^3R * R^-1 == a^3 */
   if ((result = mp_montgomery_reduce(&c, &b, mp)) != MP_OKAY) \{
      printf("Error reducing.  \%s", 
             mp_error_to_string(result));
      return EXIT_FAILURE;
   \}

   /* c now equals a^3 mod b */

   return EXIT_SUCCESS;
\}
\end{alltt} 

This particular example does not look too efficient but it demonstrates the point of the algorithm.  By 
normalizing the inputs the reduced results are always of the form $aR$ for some variable $a$.  This allows
a single final reduction to correct for the normalization and the fast reduction used within the algorithm.

For more details consider examining the file \textit{bn\_mp\_exptmod\_fast.c}.

\section{Restricted Dimminished Radix}

``Dimminished Radix'' reduction refers to reduction with respect to moduli that are ameniable to simple
digit shifting and small multiplications.  In this case the ``restricted'' variant refers to moduli of the
form $\beta^k - p$ for some $k \ge 0$ and $0 < p < \beta$ where $\beta$ is the radix (default to $2^{28}$).  

As in the case of Montgomery reduction there is a pre--computation phase required for a given modulus.

\index{mp\_dr\_setup}
\begin{alltt}
void mp_dr_setup(mp_int *a, mp_digit *d);
\end{alltt}

This computes the value required for the modulus $a$ and stores it in $d$.  This function cannot fail
and does not return any error codes.  After the pre--computation a reduction can be performed with the
following.

\index{mp\_dr\_reduce}
\begin{alltt}
int mp_dr_reduce(mp_int *a, mp_int *b, mp_digit mp);
\end{alltt}

This reduces $a$ in place modulo $b$ with the pre--computed value $mp$.  $b$ must be of a restricted
dimminished radix form and $a$ must be in the range $0 \le a < b^2$.  Dimminished radix reductions are 
much faster than both Barrett and Montgomery reductions as they have a much lower asymtotic running time.  

Since the moduli are restricted this algorithm is not particularly useful for something like Rabin, RSA or
BBS cryptographic purposes.  This reduction algorithm is useful for Diffie-Hellman and ECC where fixed
primes are acceptable.  

Note that unlike Montgomery reduction there is no normalization process.  The result of this function is
equal to the correct residue.

\section{Unrestricted Dimminshed Radix}

Unrestricted reductions work much like the restricted counterparts except in this case the moduli is of the 
form $2^k - p$ for $0 < p < \beta$.  In this sense the unrestricted reductions are more flexible as they 
can be applied to a wider range of numbers.  

\index{mp\_reduce\_2k\_setup}
\begin{alltt}
int mp_reduce_2k_setup(mp_int *a, mp_digit *d);
\end{alltt}

This will compute the required $d$ value for the given moduli $a$.  

\index{mp\_reduce\_2k}
\begin{alltt}
int mp_reduce_2k(mp_int *a, mp_int *n, mp_digit d);
\end{alltt}

This will reduce $a$ in place modulo $n$ with the pre--computed value $d$.  From my experience this routine is 
slower than mp\_dr\_reduce but faster for most moduli sizes than the Montgomery reduction.  

\chapter{Exponentiation}
\section{Single Digit Exponentiation}
\index{mp\_expt\_d}
\begin{alltt}
int mp_expt_d (mp_int * a, mp_digit b, mp_int * c)
\end{alltt}
This computes $c = a^b$ using a simple binary left-to-right algorithm.  It is faster than repeated multiplications by 
$a$ for all values of $b$ greater than three.  

\section{Modular Exponentiation}
\index{mp\_exptmod}
\begin{alltt}
int mp_exptmod (mp_int * G, mp_int * X, mp_int * P, mp_int * Y)
\end{alltt}
This computes $Y \equiv G^X \mbox{ (mod }P\mbox{)}$ using a variable width sliding window algorithm.  This function
will automatically detect the fastest modular reduction technique to use during the operation.  For negative values of 
$X$ the operation is performed as $Y \equiv (G^{-1} \mbox{ mod }P)^{\vert X \vert} \mbox{ (mod }P\mbox{)}$ provided that 
$gcd(G, P) = 1$.

This function is actually a shell around the two internal exponentiation functions.  This routine will automatically
detect when Barrett, Montgomery, Restricted and Unrestricted Dimminished Radix based exponentiation can be used.  Generally
moduli of the a ``restricted dimminished radix'' form lead to the fastest modular exponentiations.  Followed by Montgomery
and the other two algorithms.

\section{Root Finding}
\index{mp\_n\_root}
\begin{alltt}
int mp_n_root (mp_int * a, mp_digit b, mp_int * c)
\end{alltt}
This computes $c = a^{1/b}$ such that $c^b \le a$ and $(c+1)^b > a$.  The implementation of this function is not 
ideal for values of $b$ greater than three.  It will work but become very slow.  So unless you are working with very small
numbers (less than 1000 bits) I'd avoid $b > 3$ situations.  Will return a positive root only for even roots and return
a root with the sign of the input for odd roots.  For example, performing $4^{1/2}$ will return $2$ whereas $(-8)^{1/3}$ 
will return $-2$.  

This algorithm uses the ``Newton Approximation'' method and will converge on the correct root fairly quickly.  Since
the algorithm requires raising $a$ to the power of $b$ it is not ideal to attempt to find roots for large
values of $b$.  If particularly large roots are required then a factor method could be used instead.  For example,
$a^{1/16}$ is equivalent to $\left (a^{1/4} \right)^{1/4}$ or simply 
$\left ( \left ( \left ( a^{1/2} \right )^{1/2} \right )^{1/2} \right )^{1/2}$

\chapter{Prime Numbers}
\section{Trial Division}
\index{mp\_prime\_is\_divisible}
\begin{alltt}
int mp_prime_is_divisible (mp_int * a, int *result)
\end{alltt}
This will attempt to evenly divide $a$ by a list of primes\footnote{Default is the first 256 primes.} and store the 
outcome in ``result''.  That is if $result = 0$ then $a$ is not divisible by the primes, otherwise it is.  Note that 
if the function does not return \textbf{MP\_OKAY} the value in ``result'' should be considered undefined\footnote{Currently
the default is to set it to zero first.}.

\section{Fermat Test}
\index{mp\_prime\_fermat}
\begin{alltt}
int mp_prime_fermat (mp_int * a, mp_int * b, int *result)
\end{alltt}
Performs a Fermat primality test to the base $b$.  That is it computes $b^a \mbox{ mod }a$ and tests whether the value is
equal to $b$ or not.  If the values are equal then $a$ is probably prime and $result$ is set to one.  Otherwise $result$
is set to zero.

\section{Miller-Rabin Test}
\index{mp\_prime\_miller\_rabin}
\begin{alltt}
int mp_prime_miller_rabin (mp_int * a, mp_int * b, int *result)
\end{alltt}
Performs a Miller-Rabin test to the base $b$ of $a$.  This test is much stronger than the Fermat test and is very hard to
fool (besides with Carmichael numbers).  If $a$ passes the test (therefore is probably prime) $result$ is set to one.  
Otherwise $result$ is set to zero.  

Note that is suggested that you use the Miller-Rabin test instead of the Fermat test since all of the failures of 
Miller-Rabin are a subset of the failures of the Fermat test.

\subsection{Required Number of Tests}
Generally to ensure a number is very likely to be prime you have to perform the Miller-Rabin with at least a half-dozen
or so unique bases.  However, it has been proven that the probability of failure goes down as the size of the input goes up.
This is why a simple function has been provided to help out.

\index{mp\_prime\_rabin\_miller\_trials}
\begin{alltt}
int mp_prime_rabin_miller_trials(int size)
\end{alltt}
This returns the number of trials required for a $2^{-96}$ (or lower) probability of failure for a given ``size'' expressed
in bits.  This comes in handy specially since larger numbers are slower to test.  For example, a 512-bit number would
require ten tests whereas a 1024-bit number would only require four tests. 

You should always still perform a trial division before a Miller-Rabin test though.

\section{Primality Testing}
\index{mp\_prime\_is\_prime}
\begin{alltt}
int mp_prime_is_prime (mp_int * a, int t, int *result)
\end{alltt}
This will perform a trial division followed by $t$ rounds of Miller-Rabin tests on $a$ and store the result in $result$.  
If $a$ passes all of the tests $result$ is set to one, otherwise it is set to zero.  Note that $t$ is bounded by 
$1 \le t < PRIME\_SIZE$ where $PRIME\_SIZE$ is the number of primes in the prime number table (by default this is $256$).

\section{Next Prime}
\index{mp\_prime\_next\_prime}
\begin{alltt}
int mp_prime_next_prime(mp_int *a, int t, int bbs_style)
\end{alltt}
This finds the next prime after $a$ that passes mp\_prime\_is\_prime() with $t$ tests.  Set $bbs\_style$ to one if you 
want only the next prime congruent to $3 \mbox{ mod } 4$, otherwise set it to zero to find any next prime.  

\section{Random Primes}
\index{mp\_prime\_random}
\begin{alltt}
int mp_prime_random(mp_int *a, int t, int size, int bbs, 
                    ltm_prime_callback cb, void *dat)
\end{alltt}
This will find a prime greater than $256^{size}$ which can be ``bbs\_style'' or not depending on $bbs$ and must pass
$t$ rounds of tests.  The ``ltm\_prime\_callback'' is a typedef for 

\begin{alltt}
typedef int ltm_prime_callback(unsigned char *dst, int len, void *dat);
\end{alltt}

Which is a function that must read $len$ bytes (and return the amount stored) into $dst$.  The $dat$ variable is simply
copied from the original input.  It can be used to pass RNG context data to the callback.  The function 
mp\_prime\_random() is more suitable for generating primes which must be secret (as in the case of RSA) since there 
is no skew on the least significant bits.

\textit{Note:}  As of v0.30 of the LibTomMath library this function has been deprecated.  It is still available
but users are encouraged to use the new mp\_prime\_random\_ex() function instead.

\subsection{Extended Generation}
\index{mp\_prime\_random\_ex}
\begin{alltt}
int mp_prime_random_ex(mp_int *a,    int t, 
                       int     size, int flags, 
                       ltm_prime_callback cb, void *dat);
\end{alltt}
This will generate a prime in $a$ using $t$ tests of the primality testing algorithms.  The variable $size$
specifies the bit length of the prime desired.  The variable $flags$ specifies one of several options available
(see fig. \ref{fig:primeopts}) which can be OR'ed together.  The callback parameters are used as in 
mp\_prime\_random().

\begin{figure}[here]
\begin{center}
\begin{small}
\begin{tabular}{|r|l|}
\hline \textbf{Flag}         & \textbf{Meaning} \\
\hline LTM\_PRIME\_BBS       & Make the prime congruent to $3$ modulo $4$ \\
\hline LTM\_PRIME\_SAFE      & Make a prime $p$ such that $(p - 1)/2$ is also prime. \\
                             & This option implies LTM\_PRIME\_BBS as well. \\
\hline LTM\_PRIME\_2MSB\_OFF & Makes sure that the bit adjacent to the most significant bit \\
                             & Is forced to zero.  \\
\hline LTM\_PRIME\_2MSB\_ON  & Makes sure that the bit adjacent to the most significant bit \\
                             & Is forced to one. \\
\hline
\end{tabular}
\end{small}
\end{center}
\caption{Primality Generation Options}
\label{fig:primeopts}
\end{figure}

\chapter{Input and Output}
\section{ASCII Conversions}
\subsection{To ASCII}
\index{mp\_toradix}
\begin{alltt}
int mp_toradix (mp_int * a, char *str, int radix);
\end{alltt}
This still store $a$ in ``str'' as a base-``radix'' string of ASCII chars.  This function appends a NUL character
to terminate the string.  Valid values of ``radix'' line in the range $[2, 64]$.  To determine the size (exact) required
by the conversion before storing any data use the following function.

\index{mp\_radix\_size}
\begin{alltt}
int mp_radix_size (mp_int * a, int radix, int *size)
\end{alltt}
This stores in ``size'' the number of characters (including space for the NUL terminator) required.  Upon error this 
function returns an error code and ``size'' will be zero.  

\subsection{From ASCII}
\index{mp\_read\_radix}
\begin{alltt}
int mp_read_radix (mp_int * a, char *str, int radix);
\end{alltt}
This will read the base-``radix'' NUL terminated string from ``str'' into $a$.  It will stop reading when it reads a
character it does not recognize (which happens to include th NUL char... imagine that...).  A single leading $-$ sign
can be used to denote a negative number.

\section{Binary Conversions}

Converting an mp\_int to and from binary is another keen idea.

\index{mp\_unsigned\_bin\_size}
\begin{alltt}
int mp_unsigned_bin_size(mp_int *a);
\end{alltt}

This will return the number of bytes (octets) required to store the unsigned copy of the integer $a$.

\index{mp\_to\_unsigned\_bin}
\begin{alltt}
int mp_to_unsigned_bin(mp_int *a, unsigned char *b);
\end{alltt}
This will store $a$ into the buffer $b$ in big--endian format.  Fortunately this is exactly what DER (or is it ASN?)
requires.  It does not store the sign of the integer.

\index{mp\_read\_unsigned\_bin}
\begin{alltt}
int mp_read_unsigned_bin(mp_int *a, unsigned char *b, int c);
\end{alltt}
This will read in an unsigned big--endian array of bytes (octets) from $b$ of length $c$ into $a$.  The resulting
integer $a$ will always be positive.

For those who acknowledge the existence of negative numbers (heretic!) there are ``signed'' versions of the
previous functions.

\begin{alltt}
int mp_signed_bin_size(mp_int *a);
int mp_read_signed_bin(mp_int *a, unsigned char *b, int c);
int mp_to_signed_bin(mp_int *a, unsigned char *b);
\end{alltt}
They operate essentially the same as the unsigned copies except they prefix the data with zero or non--zero
byte depending on the sign.  If the sign is zpos (e.g. not negative) the prefix is zero, otherwise the prefix
is non--zero.  

\chapter{Algebraic Functions}
\section{Extended Euclidean Algorithm}
\index{mp\_exteuclid}
\begin{alltt}
int mp_exteuclid(mp_int *a, mp_int *b, 
                 mp_int *U1, mp_int *U2, mp_int *U3);
\end{alltt}

This finds the triple U1/U2/U3 using the Extended Euclidean algorithm such that the following equation holds.

\begin{equation}
a \cdot U1 + b \cdot U2 = U3
\end{equation}

Any of the U1/U2/U3 paramters can be set to \textbf{NULL} if they are not desired.  

\section{Greatest Common Divisor}
\index{mp\_gcd}
\begin{alltt}
int mp_gcd (mp_int * a, mp_int * b, mp_int * c)
\end{alltt}
This will compute the greatest common divisor of $a$ and $b$ and store it in $c$.

\section{Least Common Multiple}
\index{mp\_lcm}
\begin{alltt}
int mp_lcm (mp_int * a, mp_int * b, mp_int * c)
\end{alltt}
This will compute the least common multiple of $a$ and $b$ and store it in $c$.

\section{Jacobi Symbol}
\index{mp\_jacobi}
\begin{alltt}
int mp_jacobi (mp_int * a, mp_int * p, int *c)
\end{alltt}
This will compute the Jacobi symbol for $a$ with respect to $p$.  If $p$ is prime this essentially computes the Legendre
symbol.  The result is stored in $c$ and can take on one of three values $\lbrace -1, 0, 1 \rbrace$.  If $p$ is prime
then the result will be $-1$ when $a$ is not a quadratic residue modulo $p$.  The result will be $0$ if $a$ divides $p$
and the result will be $1$ if $a$ is a quadratic residue modulo $p$.  

\section{Modular Inverse}
\index{mp\_invmod}
\begin{alltt}
int mp_invmod (mp_int * a, mp_int * b, mp_int * c)
\end{alltt}
Computes the multiplicative inverse of $a$ modulo $b$ and stores the result in $c$ such that $ac \equiv 1 \mbox{ (mod }b\mbox{)}$.

\section{Single Digit Functions}

For those using small numbers (\textit{snicker snicker}) there are several ``helper'' functions

\index{mp\_add\_d} \index{mp\_sub\_d} \index{mp\_mul\_d} \index{mp\_div\_d} \index{mp\_mod\_d}
\begin{alltt}
int mp_add_d(mp_int *a, mp_digit b, mp_int *c);
int mp_sub_d(mp_int *a, mp_digit b, mp_int *c);
int mp_mul_d(mp_int *a, mp_digit b, mp_int *c);
int mp_div_d(mp_int *a, mp_digit b, mp_int *c, mp_digit *d);
int mp_mod_d(mp_int *a, mp_digit b, mp_digit *c);
\end{alltt}

These work like the full mp\_int capable variants except the second parameter $b$ is a mp\_digit.  These
functions fairly handy if you have to work with relatively small numbers since you will not have to allocate
an entire mp\_int to store a number like $1$ or $2$.

\documentclass[b5paper]{book}
\usepackage{hyperref}
\usepackage{makeidx}
\usepackage{amssymb}
\usepackage{color}
\usepackage{alltt}
\usepackage{graphicx}
\usepackage{layout}
\def\union{\cup}
\def\intersect{\cap}
\def\getsrandom{\stackrel{\rm R}{\gets}}
\def\cross{\times}
\def\cat{\hspace{0.5em} \| \hspace{0.5em}}
\def\catn{$\|$}
\def\divides{\hspace{0.3em} | \hspace{0.3em}}
\def\nequiv{\not\equiv}
\def\approx{\raisebox{0.2ex}{\mbox{\small $\sim$}}}
\def\lcm{{\rm lcm}}
\def\gcd{{\rm gcd}}
\def\log{{\rm log}}
\def\ord{{\rm ord}}
\def\abs{{\mathit abs}}
\def\rep{{\mathit rep}}
\def\mod{{\mathit\ mod\ }}
\renewcommand{\pmod}[1]{\ ({\rm mod\ }{#1})}
\newcommand{\floor}[1]{\left\lfloor{#1}\right\rfloor}
\newcommand{\ceil}[1]{\left\lceil{#1}\right\rceil}
\def\Or{{\rm\ or\ }}
\def\And{{\rm\ and\ }}
\def\iff{\hspace{1em}\Longleftrightarrow\hspace{1em}}
\def\implies{\Rightarrow}
\def\undefined{{\rm ``undefined"}}
\def\Proof{\vspace{1ex}\noindent {\bf Proof:}\hspace{1em}}
\let\oldphi\phi
\def\phi{\varphi}
\def\Pr{{\rm Pr}}
\newcommand{\str}[1]{{\mathbf{#1}}}
\def\F{{\mathbb F}}
\def\N{{\mathbb N}}
\def\Z{{\mathbb Z}}
\def\R{{\mathbb R}}
\def\C{{\mathbb C}}
\def\Q{{\mathbb Q}}
\definecolor{DGray}{gray}{0.5}
\newcommand{\emailaddr}[1]{\mbox{$<${#1}$>$}}
\def\twiddle{\raisebox{0.3ex}{\mbox{\tiny $\sim$}}}
\def\gap{\vspace{0.5ex}}
\makeindex
\begin{document}
\frontmatter
\pagestyle{empty}
\title{LibTomMath User Manual \\ v0.39}
\author{Tom St Denis \\ tomstdenis@iahu.ca}
\maketitle
This text, the library and the accompanying textbook are all hereby placed in the public domain.  This book has been 
formatted for B5 [176x250] paper using the \LaTeX{} {\em book} macro package.

\vspace{10cm}

\begin{flushright}Open Source.  Open Academia.  Open Minds.

\mbox{ }

Tom St Denis,

Ontario, Canada
\end{flushright}

\tableofcontents
\listoffigures
\mainmatter
\pagestyle{headings}
\chapter{Introduction}
\section{What is LibTomMath?}
LibTomMath is a library of source code which provides a series of efficient and carefully written functions for manipulating
large integer numbers.  It was written in portable ISO C source code so that it will build on any platform with a conforming
C compiler.  

In a nutshell the library was written from scratch with verbose comments to help instruct computer science students how
to implement ``bignum'' math.  However, the resulting code has proven to be very useful.  It has been used by numerous 
universities, commercial and open source software developers.  It has been used on a variety of platforms ranging from
Linux and Windows based x86 to ARM based Gameboys and PPC based MacOS machines.  

\section{License}
As of the v0.25 the library source code has been placed in the public domain with every new release.  As of the v0.28
release the textbook ``Implementing Multiple Precision Arithmetic'' has been placed in the public domain with every new
release as well.  This textbook is meant to compliment the project by providing a more solid walkthrough of the development
algorithms used in the library.

Since both\footnote{Note that the MPI files under mtest/ are copyrighted by Michael Fromberger.  They are not required to use LibTomMath.} are in the 
public domain everyone is entitled to do with them as they see fit.

\section{Building LibTomMath}

LibTomMath is meant to be very ``GCC friendly'' as it comes with a makefile well suited for GCC.  However, the library will
also build in MSVC, Borland C out of the box.  For any other ISO C compiler a makefile will have to be made by the end
developer.  

\subsection{Static Libraries}
To build as a static library for GCC issue the following
\begin{alltt}
make
\end{alltt}

command.  This will build the library and archive the object files in ``libtommath.a''.  Now you link against 
that and include ``tommath.h'' within your programs.  Alternatively to build with MSVC issue the following
\begin{alltt}
nmake -f makefile.msvc
\end{alltt}

This will build the library and archive the object files in ``tommath.lib''.  This has been tested with MSVC 
version 6.00 with service pack 5.  

\subsection{Shared Libraries}
To build as a shared library for GCC issue the following
\begin{alltt}
make -f makefile.shared
\end{alltt}
This requires the ``libtool'' package (common on most Linux/BSD systems).  It will build LibTomMath as both shared
and static then install (by default) into /usr/lib as well as install the header files in /usr/include.  The shared 
library (resource) will be called ``libtommath.la'' while the static library called ``libtommath.a''.  Generally 
you use libtool to link your application against the shared object.  

There is limited support for making a ``DLL'' in windows via the ``makefile.cygwin\_dll'' makefile.  It requires 
Cygwin to work with since it requires the auto-export/import functionality.  The resulting DLL and import library 
``libtommath.dll.a'' can be used to link LibTomMath dynamically to any Windows program using Cygwin.

\subsection{Testing}
To build the library and the test harness type

\begin{alltt}
make test
\end{alltt}

This will build the library, ``test'' and ``mtest/mtest''.  The ``test'' program will accept test vectors and verify the
results.  ``mtest/mtest'' will generate test vectors using the MPI library by Michael Fromberger\footnote{A copy of MPI
is included in the package}.  Simply pipe mtest into test using

\begin{alltt}
mtest/mtest | test
\end{alltt}

If you do not have a ``/dev/urandom'' style RNG source you will have to write your own PRNG and simply pipe that into 
mtest.  For example, if your PRNG program is called ``myprng'' simply invoke

\begin{alltt}
myprng | mtest/mtest | test
\end{alltt}

This will output a row of numbers that are increasing.  Each column is a different test (such as addition, multiplication, etc)
that is being performed.  The numbers represent how many times the test was invoked.  If an error is detected the program
will exit with a dump of the relevent numbers it was working with.

\section{Build Configuration}
LibTomMath can configured at build time in three phases we shall call ``depends'', ``tweaks'' and ``trims''.  
Each phase changes how the library is built and they are applied one after another respectively.  

To make the system more powerful you can tweak the build process.  Classes are defined in the file
``tommath\_superclass.h''.  By default, the symbol ``LTM\_ALL'' shall be defined which simply 
instructs the system to build all of the functions.  This is how LibTomMath used to be packaged.  This will give you 
access to every function LibTomMath offers.

However, there are cases where such a build is not optional.  For instance, you want to perform RSA operations.  You 
don't need the vast majority of the library to perform these operations.  Aside from LTM\_ALL there is 
another pre--defined class ``SC\_RSA\_1'' which works in conjunction with the RSA from LibTomCrypt.  Additional 
classes can be defined base on the need of the user.

\subsection{Build Depends}
In the file tommath\_class.h you will see a large list of C ``defines'' followed by a series of ``ifdefs''
which further define symbols.  All of the symbols (technically they're macros $\ldots$) represent a given C source
file.  For instance, BN\_MP\_ADD\_C represents the file ``bn\_mp\_add.c''.  When a define has been enabled the
function in the respective file will be compiled and linked into the library.  Accordingly when the define
is absent the file will not be compiled and not contribute any size to the library.

You will also note that the header tommath\_class.h is actually recursively included (it includes itself twice).  
This is to help resolve as many dependencies as possible.  In the last pass the symbol LTM\_LAST will be defined.  
This is useful for ``trims''.

\subsection{Build Tweaks}
A tweak is an algorithm ``alternative''.  For example, to provide tradeoffs (usually between size and space).
They can be enabled at any pass of the configuration phase.

\begin{small}
\begin{center}
\begin{tabular}{|l|l|}
\hline \textbf{Define} & \textbf{Purpose} \\
\hline BN\_MP\_DIV\_SMALL & Enables a slower, smaller and equally \\
                          & functional mp\_div() function \\
\hline
\end{tabular}
\end{center}
\end{small}

\subsection{Build Trims}
A trim is a manner of removing functionality from a function that is not required.  For instance, to perform
RSA cryptography you only require exponentiation with odd moduli so even moduli support can be safely removed.  
Build trims are meant to be defined on the last pass of the configuration which means they are to be defined
only if LTM\_LAST has been defined.

\subsubsection{Moduli Related}
\begin{small}
\begin{center}
\begin{tabular}{|l|l|}
\hline \textbf{Restriction} & \textbf{Undefine} \\
\hline Exponentiation with odd moduli only & BN\_S\_MP\_EXPTMOD\_C \\
                                           & BN\_MP\_REDUCE\_C \\
                                           & BN\_MP\_REDUCE\_SETUP\_C \\
                                           & BN\_S\_MP\_MUL\_HIGH\_DIGS\_C \\
                                           & BN\_FAST\_S\_MP\_MUL\_HIGH\_DIGS\_C \\
\hline Exponentiation with random odd moduli & (The above plus the following) \\
                                           & BN\_MP\_REDUCE\_2K\_C \\
                                           & BN\_MP\_REDUCE\_2K\_SETUP\_C \\
                                           & BN\_MP\_REDUCE\_IS\_2K\_C \\
                                           & BN\_MP\_DR\_IS\_MODULUS\_C \\
                                           & BN\_MP\_DR\_REDUCE\_C \\
                                           & BN\_MP\_DR\_SETUP\_C \\
\hline Modular inverse odd moduli only     & BN\_MP\_INVMOD\_SLOW\_C \\
\hline Modular inverse (both, smaller/slower) & BN\_FAST\_MP\_INVMOD\_C \\
\hline
\end{tabular}
\end{center}
\end{small}

\subsubsection{Operand Size Related}
\begin{small}
\begin{center}
\begin{tabular}{|l|l|}
\hline \textbf{Restriction} & \textbf{Undefine} \\
\hline Moduli $\le 2560$ bits              & BN\_MP\_MONTGOMERY\_REDUCE\_C \\
                                           & BN\_S\_MP\_MUL\_DIGS\_C \\
                                           & BN\_S\_MP\_MUL\_HIGH\_DIGS\_C \\
                                           & BN\_S\_MP\_SQR\_C \\
\hline Polynomial Schmolynomial            & BN\_MP\_KARATSUBA\_MUL\_C \\
                                           & BN\_MP\_KARATSUBA\_SQR\_C \\
                                           & BN\_MP\_TOOM\_MUL\_C \\ 
                                           & BN\_MP\_TOOM\_SQR\_C \\

\hline
\end{tabular}
\end{center}
\end{small}


\section{Purpose of LibTomMath}
Unlike  GNU MP (GMP) Library, LIP, OpenSSL or various other commercial kits (Miracl), LibTomMath was not written with 
bleeding edge performance in mind.  First and foremost LibTomMath was written to be entirely open.  Not only is the 
source code public domain (unlike various other GPL/etc licensed code), not only is the code freely downloadable but the
source code is also accessible for computer science students attempting to learn ``BigNum'' or multiple precision
arithmetic techniques. 

LibTomMath was written to be an instructive collection of source code.  This is why there are many comments, only one
function per source file and often I use a ``middle-road'' approach where I don't cut corners for an extra 2\% speed
increase.

Source code alone cannot really teach how the algorithms work which is why I also wrote a textbook that accompanies
the library (beat that!).

So you may be thinking ``should I use LibTomMath?'' and the answer is a definite maybe.  Let me tabulate what I think
are the pros and cons of LibTomMath by comparing it to the math routines from GnuPG\footnote{GnuPG v1.2.3 versus LibTomMath v0.28}.

\newpage\begin{figure}[here]
\begin{small}
\begin{center}
\begin{tabular}{|l|c|c|l|}
\hline \textbf{Criteria} & \textbf{Pro} & \textbf{Con} & \textbf{Notes} \\
\hline Few lines of code per file & X & & GnuPG $ = 300.9$, LibTomMath  $ = 71.97$ \\
\hline Commented function prototypes & X && GnuPG function names are cryptic. \\
\hline Speed && X & LibTomMath is slower.  \\
\hline Totally free & X & & GPL has unfavourable restrictions.\\
\hline Large function base & X & & GnuPG is barebones. \\
\hline Five modular reduction algorithms & X & & Faster modular exponentiation for a variety of moduli. \\
\hline Portable & X & & GnuPG requires configuration to build. \\
\hline
\end{tabular}
\end{center}
\end{small}
\caption{LibTomMath Valuation}
\end{figure}

It may seem odd to compare LibTomMath to GnuPG since the math in GnuPG is only a small portion of the entire application. 
However, LibTomMath was written with cryptography in mind.  It provides essentially all of the functions a cryptosystem
would require when working with large integers.  

So it may feel tempting to just rip the math code out of GnuPG (or GnuMP where it was taken from originally) in your
own application but I think there are reasons not to.  While LibTomMath is slower than libraries such as GnuMP it is
not normally significantly slower.  On x86 machines the difference is normally a factor of two when performing modular
exponentiations.  It depends largely on the processor, compiler and the moduli being used.

Essentially the only time you wouldn't use LibTomMath is when blazing speed is the primary concern.  However,
on the other side of the coin LibTomMath offers you a totally free (public domain) well structured math library
that is very flexible, complete and performs well in resource contrained environments.  Fast RSA for example can
be performed with as little as 8KB of ram for data (again depending on build options).  

\chapter{Getting Started with LibTomMath}
\section{Building Programs}
In order to use LibTomMath you must include ``tommath.h'' and link against the appropriate library file (typically 
libtommath.a).  There is no library initialization required and the entire library is thread safe.

\section{Return Codes}
There are three possible return codes a function may return.

\index{MP\_OKAY}\index{MP\_YES}\index{MP\_NO}\index{MP\_VAL}\index{MP\_MEM}
\begin{figure}[here!]
\begin{center}
\begin{small}
\begin{tabular}{|l|l|}
\hline \textbf{Code} & \textbf{Meaning} \\
\hline MP\_OKAY & The function succeeded. \\
\hline MP\_VAL  & The function input was invalid. \\
\hline MP\_MEM  & Heap memory exhausted. \\
\hline &\\
\hline MP\_YES  & Response is yes. \\
\hline MP\_NO   & Response is no. \\
\hline
\end{tabular}
\end{small}
\end{center}
\caption{Return Codes}
\end{figure}

The last two codes listed are not actually ``return'ed'' by a function.  They are placed in an integer (the caller must
provide the address of an integer it can store to) which the caller can access.  To convert one of the three return codes
to a string use the following function.

\index{mp\_error\_to\_string}
\begin{alltt}
char *mp_error_to_string(int code);
\end{alltt}

This will return a pointer to a string which describes the given error code.  It will not work for the return codes 
MP\_YES and MP\_NO.  

\section{Data Types}
The basic ``multiple precision integer'' type is known as the ``mp\_int'' within LibTomMath.  This data type is used to
organize all of the data required to manipulate the integer it represents.  Within LibTomMath it has been prototyped
as the following.

\index{mp\_int}
\begin{alltt}
typedef struct  \{
    int used, alloc, sign;
    mp_digit *dp;
\} mp_int;
\end{alltt}

Where ``mp\_digit'' is a data type that represents individual digits of the integer.  By default, an mp\_digit is the
ISO C ``unsigned long'' data type and each digit is $28-$bits long.  The mp\_digit type can be configured to suit other
platforms by defining the appropriate macros.  

All LTM functions that use the mp\_int type will expect a pointer to mp\_int structure.  You must allocate memory to
hold the structure itself by yourself (whether off stack or heap it doesn't matter).  The very first thing that must be
done to use an mp\_int is that it must be initialized.

\section{Function Organization}

The arithmetic functions of the library are all organized to have the same style prototype.  That is source operands
are passed on the left and the destination is on the right.  For instance,

\begin{alltt}
mp_add(&a, &b, &c);       /* c = a + b */
mp_mul(&a, &a, &c);       /* c = a * a */
mp_div(&a, &b, &c, &d);   /* c = [a/b], d = a mod b */
\end{alltt}

Another feature of the way the functions have been implemented is that source operands can be destination operands as well.
For instance,

\begin{alltt}
mp_add(&a, &b, &b);       /* b = a + b */
mp_div(&a, &b, &a, &c);   /* a = [a/b], c = a mod b */
\end{alltt}

This allows operands to be re-used which can make programming simpler.

\section{Initialization}
\subsection{Single Initialization}
A single mp\_int can be initialized with the ``mp\_init'' function. 

\index{mp\_init}
\begin{alltt}
int mp_init (mp_int * a);
\end{alltt}

This function expects a pointer to an mp\_int structure and will initialize the members of the structure so the mp\_int
represents the default integer which is zero.  If the functions returns MP\_OKAY then the mp\_int is ready to be used
by the other LibTomMath functions.

\begin{small} \begin{alltt}
int main(void)
\{
   mp_int number;
   int result;

   if ((result = mp_init(&number)) != MP_OKAY) \{
      printf("Error initializing the number.  \%s", 
             mp_error_to_string(result));
      return EXIT_FAILURE;
   \}
 
   /* use the number */

   return EXIT_SUCCESS;
\}
\end{alltt} \end{small}

\subsection{Single Free}
When you are finished with an mp\_int it is ideal to return the heap it used back to the system.  The following function 
provides this functionality.

\index{mp\_clear}
\begin{alltt}
void mp_clear (mp_int * a);
\end{alltt}

The function expects a pointer to a previously initialized mp\_int structure and frees the heap it uses.  It sets the 
pointer\footnote{The ``dp'' member.} within the mp\_int to \textbf{NULL} which is used to prevent double free situations. 
Is is legal to call mp\_clear() twice on the same mp\_int in a row.  

\begin{small} \begin{alltt}
int main(void)
\{
   mp_int number;
   int result;

   if ((result = mp_init(&number)) != MP_OKAY) \{
      printf("Error initializing the number.  \%s", 
             mp_error_to_string(result));
      return EXIT_FAILURE;
   \}
 
   /* use the number */

   /* We're done with it. */
   mp_clear(&number);

   return EXIT_SUCCESS;
\}
\end{alltt} \end{small}

\subsection{Multiple Initializations}
Certain algorithms require more than one large integer.  In these instances it is ideal to initialize all of the mp\_int
variables in an ``all or nothing'' fashion.  That is, they are either all initialized successfully or they are all
not initialized.

The  mp\_init\_multi() function provides this functionality.

\index{mp\_init\_multi} \index{mp\_clear\_multi}
\begin{alltt}
int mp_init_multi(mp_int *mp, ...);
\end{alltt}

It accepts a \textbf{NULL} terminated list of pointers to mp\_int structures.  It will attempt to initialize them all
at once.  If the function returns MP\_OKAY then all of the mp\_int variables are ready to use, otherwise none of them
are available for use.  A complementary mp\_clear\_multi() function allows multiple mp\_int variables to be free'd 
from the heap at the same time.  

\begin{small} \begin{alltt}
int main(void)
\{
   mp_int num1, num2, num3;
   int result;

   if ((result = mp_init_multi(&num1, 
                               &num2,
                               &num3, NULL)) != MP\_OKAY) \{      
      printf("Error initializing the numbers.  \%s", 
             mp_error_to_string(result));
      return EXIT_FAILURE;
   \}
 
   /* use the numbers */

   /* We're done with them. */
   mp_clear_multi(&num1, &num2, &num3, NULL);

   return EXIT_SUCCESS;
\}
\end{alltt} \end{small}

\subsection{Other Initializers}
To initialized and make a copy of an mp\_int the mp\_init\_copy() function has been provided.  

\index{mp\_init\_copy}
\begin{alltt}
int mp_init_copy (mp_int * a, mp_int * b);
\end{alltt}

This function will initialize $a$ and make it a copy of $b$ if all goes well.

\begin{small} \begin{alltt}
int main(void)
\{
   mp_int num1, num2;
   int result;

   /* initialize and do work on num1 ... */

   /* We want a copy of num1 in num2 now */
   if ((result = mp_init_copy(&num2, &num1)) != MP_OKAY) \{
     printf("Error initializing the copy.  \%s", 
             mp_error_to_string(result));
      return EXIT_FAILURE;
   \}
 
   /* now num2 is ready and contains a copy of num1 */

   /* We're done with them. */
   mp_clear_multi(&num1, &num2, NULL);

   return EXIT_SUCCESS;
\}
\end{alltt} \end{small}

Another less common initializer is mp\_init\_size() which allows the user to initialize an mp\_int with a given
default number of digits.  By default, all initializers allocate \textbf{MP\_PREC} digits.  This function lets
you override this behaviour.

\index{mp\_init\_size}
\begin{alltt}
int mp_init_size (mp_int * a, int size);
\end{alltt}

The $size$ parameter must be greater than zero.  If the function succeeds the mp\_int $a$ will be initialized
to have $size$ digits (which are all initially zero).  

\begin{small} \begin{alltt}
int main(void)
\{
   mp_int number;
   int result;

   /* we need a 60-digit number */
   if ((result = mp_init_size(&number, 60)) != MP_OKAY) \{
      printf("Error initializing the number.  \%s", 
             mp_error_to_string(result));
      return EXIT_FAILURE;
   \}
 
   /* use the number */

   return EXIT_SUCCESS;
\}
\end{alltt} \end{small}

\section{Maintenance Functions}

\subsection{Reducing Memory Usage}
When an mp\_int is in a state where it won't be changed again\footnote{A Diffie-Hellman modulus for instance.} excess
digits can be removed to return memory to the heap with the mp\_shrink() function.

\index{mp\_shrink}
\begin{alltt}
int mp_shrink (mp_int * a);
\end{alltt}

This will remove excess digits of the mp\_int $a$.  If the operation fails the mp\_int should be intact without the
excess digits being removed.  Note that you can use a shrunk mp\_int in further computations, however, such operations
will require heap operations which can be slow.  It is not ideal to shrink mp\_int variables that you will further
modify in the system (unless you are seriously low on memory).  

\begin{small} \begin{alltt}
int main(void)
\{
   mp_int number;
   int result;

   if ((result = mp_init(&number)) != MP_OKAY) \{
      printf("Error initializing the number.  \%s", 
             mp_error_to_string(result));
      return EXIT_FAILURE;
   \}
 
   /* use the number [e.g. pre-computation]  */

   /* We're done with it for now. */
   if ((result = mp_shrink(&number)) != MP_OKAY) \{
      printf("Error shrinking the number.  \%s", 
             mp_error_to_string(result));
      return EXIT_FAILURE;
   \}

   /* use it .... */


   /* we're done with it. */ 
   mp_clear(&number);

   return EXIT_SUCCESS;
\}
\end{alltt} \end{small}

\subsection{Adding additional digits}

Within the mp\_int structure are two parameters which control the limitations of the array of digits that represent
the integer the mp\_int is meant to equal.   The \textit{used} parameter dictates how many digits are significant, that is,
contribute to the value of the mp\_int.  The \textit{alloc} parameter dictates how many digits are currently available in
the array.  If you need to perform an operation that requires more digits you will have to mp\_grow() the mp\_int to
your desired size.  

\index{mp\_grow}
\begin{alltt}
int mp_grow (mp_int * a, int size);
\end{alltt}

This will grow the array of digits of $a$ to $size$.  If the \textit{alloc} parameter is already bigger than
$size$ the function will not do anything.

\begin{small} \begin{alltt}
int main(void)
\{
   mp_int number;
   int result;

   if ((result = mp_init(&number)) != MP_OKAY) \{
      printf("Error initializing the number.  \%s", 
             mp_error_to_string(result));
      return EXIT_FAILURE;
   \}
 
   /* use the number */

   /* We need to add 20 digits to the number  */
   if ((result = mp_grow(&number, number.alloc + 20)) != MP_OKAY) \{
      printf("Error growing the number.  \%s", 
             mp_error_to_string(result));
      return EXIT_FAILURE;
   \}


   /* use the number */

   /* we're done with it. */ 
   mp_clear(&number);

   return EXIT_SUCCESS;
\}
\end{alltt} \end{small}

\chapter{Basic Operations}
\section{Small Constants}
Setting mp\_ints to small constants is a relatively common operation.  To accomodate these instances there are two
small constant assignment functions.  The first function is used to set a single digit constant while the second sets
an ISO C style ``unsigned long'' constant.  The reason for both functions is efficiency.  Setting a single digit is quick but the
domain of a digit can change (it's always at least $0 \ldots 127$).  

\subsection{Single Digit}

Setting a single digit can be accomplished with the following function.

\index{mp\_set}
\begin{alltt}
void mp_set (mp_int * a, mp_digit b);
\end{alltt}

This will zero the contents of $a$ and make it represent an integer equal to the value of $b$.  Note that this
function has a return type of \textbf{void}.  It cannot cause an error so it is safe to assume the function
succeeded.

\begin{small} \begin{alltt}
int main(void)
\{
   mp_int number;
   int result;

   if ((result = mp_init(&number)) != MP_OKAY) \{
      printf("Error initializing the number.  \%s", 
             mp_error_to_string(result));
      return EXIT_FAILURE;
   \}
 
   /* set the number to 5 */
   mp_set(&number, 5);

   /* we're done with it. */ 
   mp_clear(&number);

   return EXIT_SUCCESS;
\}
\end{alltt} \end{small}

\subsection{Long Constants}

To set a constant that is the size of an ISO C ``unsigned long'' and larger than a single digit the following function 
can be used.

\index{mp\_set\_int}
\begin{alltt}
int mp_set_int (mp_int * a, unsigned long b);
\end{alltt}

This will assign the value of the 32-bit variable $b$ to the mp\_int $a$.  Unlike mp\_set() this function will always
accept a 32-bit input regardless of the size of a single digit.  However, since the value may span several digits 
this function can fail if it runs out of heap memory.

To get the ``unsigned long'' copy of an mp\_int the following function can be used.

\index{mp\_get\_int}
\begin{alltt}
unsigned long mp_get_int (mp_int * a);
\end{alltt}

This will return the 32 least significant bits of the mp\_int $a$.  

\begin{small} \begin{alltt}
int main(void)
\{
   mp_int number;
   int result;

   if ((result = mp_init(&number)) != MP_OKAY) \{
      printf("Error initializing the number.  \%s", 
             mp_error_to_string(result));
      return EXIT_FAILURE;
   \}
 
   /* set the number to 654321 (note this is bigger than 127) */
   if ((result = mp_set_int(&number, 654321)) != MP_OKAY) \{
      printf("Error setting the value of the number.  \%s", 
             mp_error_to_string(result));
      return EXIT_FAILURE;
   \}

   printf("number == \%lu", mp_get_int(&number));

   /* we're done with it. */ 
   mp_clear(&number);

   return EXIT_SUCCESS;
\}
\end{alltt} \end{small}

This should output the following if the program succeeds.

\begin{alltt}
number == 654321
\end{alltt}

\subsection{Initialize and Setting Constants}
To both initialize and set small constants the following two functions are available.
\index{mp\_init\_set} \index{mp\_init\_set\_int}
\begin{alltt}
int mp_init_set (mp_int * a, mp_digit b);
int mp_init_set_int (mp_int * a, unsigned long b);
\end{alltt}

Both functions work like the previous counterparts except they first mp\_init $a$ before setting the values.  

\begin{alltt}
int main(void)
\{
   mp_int number1, number2;
   int    result;

   /* initialize and set a single digit */
   if ((result = mp_init_set(&number1, 100)) != MP_OKAY) \{
      printf("Error setting number1: \%s", 
             mp_error_to_string(result));
      return EXIT_FAILURE;
   \}             

   /* initialize and set a long */
   if ((result = mp_init_set_int(&number2, 1023)) != MP_OKAY) \{
      printf("Error setting number2: \%s", 
             mp_error_to_string(result));
      return EXIT_FAILURE;
   \}

   /* display */
   printf("Number1, Number2 == \%lu, \%lu",
          mp_get_int(&number1), mp_get_int(&number2));

   /* clear */
   mp_clear_multi(&number1, &number2, NULL);

   return EXIT_SUCCESS;
\}
\end{alltt}

If this program succeeds it shall output.
\begin{alltt}
Number1, Number2 == 100, 1023
\end{alltt}

\section{Comparisons}

Comparisons in LibTomMath are always performed in a ``left to right'' fashion.  There are three possible return codes
for any comparison.

\index{MP\_GT} \index{MP\_EQ} \index{MP\_LT}
\begin{figure}[here]
\begin{center}
\begin{tabular}{|c|c|}
\hline \textbf{Result Code} & \textbf{Meaning} \\
\hline MP\_GT & $a > b$ \\
\hline MP\_EQ & $a = b$ \\
\hline MP\_LT & $a < b$ \\
\hline
\end{tabular}
\end{center}
\caption{Comparison Codes for $a, b$}
\label{fig:CMP}
\end{figure}

In figure \ref{fig:CMP} two integers $a$ and $b$ are being compared.  In this case $a$ is said to be ``to the left'' of 
$b$.  

\subsection{Unsigned comparison}

An unsigned comparison considers only the digits themselves and not the associated \textit{sign} flag of the 
mp\_int structures.  This is analogous to an absolute comparison.  The function mp\_cmp\_mag() will compare two
mp\_int variables based on their digits only. 

\index{mp\_cmp\_mag}
\begin{alltt}
int mp_cmp_mag(mp_int * a, mp_int * b);
\end{alltt}
This will compare $a$ to $b$ placing $a$ to the left of $b$.  This function cannot fail and will return one of the
three compare codes listed in figure \ref{fig:CMP}.

\begin{small} \begin{alltt}
int main(void)
\{
   mp_int number1, number2;
   int result;

   if ((result = mp_init_multi(&number1, &number2, NULL)) != MP_OKAY) \{
      printf("Error initializing the numbers.  \%s", 
             mp_error_to_string(result));
      return EXIT_FAILURE;
   \}
 
   /* set the number1 to 5 */
   mp_set(&number1, 5);
  
   /* set the number2 to -6 */
   mp_set(&number2, 6);
   if ((result = mp_neg(&number2, &number2)) != MP_OKAY) \{
      printf("Error negating number2.  \%s", 
             mp_error_to_string(result));
      return EXIT_FAILURE;
   \}

   switch(mp_cmp_mag(&number1, &number2)) \{
       case MP_GT:  printf("|number1| > |number2|"); break;
       case MP_EQ:  printf("|number1| = |number2|"); break;
       case MP_LT:  printf("|number1| < |number2|"); break;
   \}

   /* we're done with it. */ 
   mp_clear_multi(&number1, &number2, NULL);

   return EXIT_SUCCESS;
\}
\end{alltt} \end{small}

If this program\footnote{This function uses the mp\_neg() function which is discussed in section \ref{sec:NEG}.} completes 
successfully it should print the following.

\begin{alltt}
|number1| < |number2|
\end{alltt}

This is because $\vert -6 \vert = 6$ and obviously $5 < 6$.

\subsection{Signed comparison}

To compare two mp\_int variables based on their signed value the mp\_cmp() function is provided.

\index{mp\_cmp}
\begin{alltt}
int mp_cmp(mp_int * a, mp_int * b);
\end{alltt}

This will compare $a$ to the left of $b$.  It will first compare the signs of the two mp\_int variables.  If they
differ it will return immediately based on their signs.  If the signs are equal then it will compare the digits
individually.  This function will return one of the compare conditions codes listed in figure \ref{fig:CMP}.

\begin{small} \begin{alltt}
int main(void)
\{
   mp_int number1, number2;
   int result;

   if ((result = mp_init_multi(&number1, &number2, NULL)) != MP_OKAY) \{
      printf("Error initializing the numbers.  \%s", 
             mp_error_to_string(result));
      return EXIT_FAILURE;
   \}
 
   /* set the number1 to 5 */
   mp_set(&number1, 5);
  
   /* set the number2 to -6 */
   mp_set(&number2, 6);
   if ((result = mp_neg(&number2, &number2)) != MP_OKAY) \{
      printf("Error negating number2.  \%s", 
             mp_error_to_string(result));
      return EXIT_FAILURE;
   \}

   switch(mp_cmp(&number1, &number2)) \{
       case MP_GT:  printf("number1 > number2"); break;
       case MP_EQ:  printf("number1 = number2"); break;
       case MP_LT:  printf("number1 < number2"); break;
   \}

   /* we're done with it. */ 
   mp_clear_multi(&number1, &number2, NULL);

   return EXIT_SUCCESS;
\}
\end{alltt} \end{small}

If this program\footnote{This function uses the mp\_neg() function which is discussed in section \ref{sec:NEG}.} completes 
successfully it should print the following.

\begin{alltt}
number1 > number2
\end{alltt}

\subsection{Single Digit}

To compare a single digit against an mp\_int the following function has been provided.

\index{mp\_cmp\_d}
\begin{alltt}
int mp_cmp_d(mp_int * a, mp_digit b);
\end{alltt}

This will compare $a$ to the left of $b$ using a signed comparison.  Note that it will always treat $b$ as 
positive.  This function is rather handy when you have to compare against small values such as $1$ (which often
comes up in cryptography).  The function cannot fail and will return one of the tree compare condition codes
listed in figure \ref{fig:CMP}.


\begin{small} \begin{alltt}
int main(void)
\{
   mp_int number;
   int result;

   if ((result = mp_init(&number)) != MP_OKAY) \{
      printf("Error initializing the number.  \%s", 
             mp_error_to_string(result));
      return EXIT_FAILURE;
   \}
 
   /* set the number to 5 */
   mp_set(&number, 5);

   switch(mp_cmp_d(&number, 7)) \{
       case MP_GT:  printf("number > 7"); break;
       case MP_EQ:  printf("number = 7"); break;
       case MP_LT:  printf("number < 7"); break;
   \}

   /* we're done with it. */ 
   mp_clear(&number);

   return EXIT_SUCCESS;
\}
\end{alltt} \end{small}

If this program functions properly it will print out the following.

\begin{alltt}
number < 7
\end{alltt}

\section{Logical Operations}

Logical operations are operations that can be performed either with simple shifts or boolean operators such as
AND, XOR and OR directly.  These operations are very quick.

\subsection{Multiplication by two}

Multiplications and divisions by any power of two can be performed with quick logical shifts either left or
right depending on the operation.  

When multiplying or dividing by two a special case routine can be used which are as follows.
\index{mp\_mul\_2} \index{mp\_div\_2}
\begin{alltt}
int mp_mul_2(mp_int * a, mp_int * b);
int mp_div_2(mp_int * a, mp_int * b);
\end{alltt}

The former will assign twice $a$ to $b$ while the latter will assign half $a$ to $b$.  These functions are fast
since the shift counts and maskes are hardcoded into the routines.

\begin{small} \begin{alltt}
int main(void)
\{
   mp_int number;
   int result;

   if ((result = mp_init(&number)) != MP_OKAY) \{
      printf("Error initializing the number.  \%s", 
             mp_error_to_string(result));
      return EXIT_FAILURE;
   \}
 
   /* set the number to 5 */
   mp_set(&number, 5);

   /* multiply by two */
   if ((result = mp\_mul\_2(&number, &number)) != MP_OKAY) \{
      printf("Error multiplying the number.  \%s", 
             mp_error_to_string(result));
      return EXIT_FAILURE;
   \}
   switch(mp_cmp_d(&number, 7)) \{
       case MP_GT:  printf("2*number > 7"); break;
       case MP_EQ:  printf("2*number = 7"); break;
       case MP_LT:  printf("2*number < 7"); break;
   \}

   /* now divide by two */
   if ((result = mp\_div\_2(&number, &number)) != MP_OKAY) \{
      printf("Error dividing the number.  \%s", 
             mp_error_to_string(result));
      return EXIT_FAILURE;
   \}
   switch(mp_cmp_d(&number, 7)) \{
       case MP_GT:  printf("2*number/2 > 7"); break;
       case MP_EQ:  printf("2*number/2 = 7"); break;
       case MP_LT:  printf("2*number/2 < 7"); break;
   \}

   /* we're done with it. */ 
   mp_clear(&number);

   return EXIT_SUCCESS;
\}
\end{alltt} \end{small}

If this program is successful it will print out the following text.

\begin{alltt}
2*number > 7
2*number/2 < 7
\end{alltt}

Since $10 > 7$ and $5 < 7$.  To multiply by a power of two the following function can be used.

\index{mp\_mul\_2d}
\begin{alltt}
int mp_mul_2d(mp_int * a, int b, mp_int * c);
\end{alltt}

This will multiply $a$ by $2^b$ and store the result in ``c''.  If the value of $b$ is less than or equal to 
zero the function will copy $a$ to ``c'' without performing any further actions.  

To divide by a power of two use the following.

\index{mp\_div\_2d}
\begin{alltt}
int mp_div_2d (mp_int * a, int b, mp_int * c, mp_int * d);
\end{alltt}
Which will divide $a$ by $2^b$, store the quotient in ``c'' and the remainder in ``d'.  If $b \le 0$ then the
function simply copies $a$ over to ``c'' and zeroes $d$.  The variable $d$ may be passed as a \textbf{NULL}
value to signal that the remainder is not desired.

\subsection{Polynomial Basis Operations}

Strictly speaking the organization of the integers within the mp\_int structures is what is known as a 
``polynomial basis''.  This simply means a field element is stored by divisions of a radix.  For example, if
$f(x) = \sum_{i=0}^{k} y_ix^k$ for any vector $\vec y$ then the array of digits in $\vec y$ are said to be 
the polynomial basis representation of $z$ if $f(\beta) = z$ for a given radix $\beta$.  

To multiply by the polynomial $g(x) = x$ all you have todo is shift the digits of the basis left one place.  The
following function provides this operation.

\index{mp\_lshd}
\begin{alltt}
int mp_lshd (mp_int * a, int b);
\end{alltt}

This will multiply $a$ in place by $x^b$ which is equivalent to shifting the digits left $b$ places and inserting zeroes
in the least significant digits.  Similarly to divide by a power of $x$ the following function is provided.

\index{mp\_rshd}
\begin{alltt}
void mp_rshd (mp_int * a, int b)
\end{alltt}
This will divide $a$ in place by $x^b$ and discard the remainder.  This function cannot fail as it performs the operations
in place and no new digits are required to complete it.

\subsection{AND, OR and XOR Operations}

While AND, OR and XOR operations are not typical ``bignum functions'' they can be useful in several instances.  The
three functions are prototyped as follows.

\index{mp\_or} \index{mp\_and} \index{mp\_xor}
\begin{alltt}
int mp_or  (mp_int * a, mp_int * b, mp_int * c);
int mp_and (mp_int * a, mp_int * b, mp_int * c);
int mp_xor (mp_int * a, mp_int * b, mp_int * c);
\end{alltt}

Which compute $c = a \odot b$ where $\odot$ is one of OR, AND or XOR.  

\section{Addition and Subtraction}

To compute an addition or subtraction the following two functions can be used.

\index{mp\_add} \index{mp\_sub}
\begin{alltt}
int mp_add (mp_int * a, mp_int * b, mp_int * c);
int mp_sub (mp_int * a, mp_int * b, mp_int * c)
\end{alltt}

Which perform $c = a \odot b$ where $\odot$ is one of signed addition or subtraction.  The operations are fully sign
aware.

\section{Sign Manipulation}
\subsection{Negation}
\label{sec:NEG}
Simple integer negation can be performed with the following.

\index{mp\_neg}
\begin{alltt}
int mp_neg (mp_int * a, mp_int * b);
\end{alltt}

Which assigns $-a$ to $b$.  

\subsection{Absolute}
Simple integer absolutes can be performed with the following.

\index{mp\_neg}
\begin{alltt}
int mp_abs (mp_int * a, mp_int * b);
\end{alltt}

Which assigns $\vert a \vert$ to $b$.  

\section{Integer Division and Remainder}
To perform a complete and general integer division with remainder use the following function.

\index{mp\_div}
\begin{alltt}
int mp_div (mp_int * a, mp_int * b, mp_int * c, mp_int * d);
\end{alltt}
                                                        
This divides $a$ by $b$ and stores the quotient in $c$ and $d$.  The signed quotient is computed such that 
$bc + d = a$.  Note that either of $c$ or $d$ can be set to \textbf{NULL} if their value is not required.  If 
$b$ is zero the function returns \textbf{MP\_VAL}.  


\chapter{Multiplication and Squaring}
\section{Multiplication}
A full signed integer multiplication can be performed with the following.
\index{mp\_mul}
\begin{alltt}
int mp_mul (mp_int * a, mp_int * b, mp_int * c);
\end{alltt}
Which assigns the full signed product $ab$ to $c$.  This function actually breaks into one of four cases which are 
specific multiplication routines optimized for given parameters.  First there are the Toom-Cook multiplications which
should only be used with very large inputs.  This is followed by the Karatsuba multiplications which are for moderate
sized inputs.  Then followed by the Comba and baseline multipliers.

Fortunately for the developer you don't really need to know this unless you really want to fine tune the system.  mp\_mul()
will determine on its own\footnote{Some tweaking may be required.} what routine to use automatically when it is called.

\begin{alltt}
int main(void)
\{
   mp_int number1, number2;
   int result;

   /* Initialize the numbers */
   if ((result = mp_init_multi(&number1, 
                               &number2, NULL)) != MP_OKAY) \{
      printf("Error initializing the numbers.  \%s", 
             mp_error_to_string(result));
      return EXIT_FAILURE;
   \}

   /* set the terms */
   if ((result = mp_set_int(&number, 257)) != MP_OKAY) \{
      printf("Error setting number1.  \%s", 
             mp_error_to_string(result));
      return EXIT_FAILURE;
   \}
 
   if ((result = mp_set_int(&number2, 1023)) != MP_OKAY) \{
      printf("Error setting number2.  \%s", 
             mp_error_to_string(result));
      return EXIT_FAILURE;
   \}

   /* multiply them */
   if ((result = mp_mul(&number1, &number2,
                        &number1)) != MP_OKAY) \{
      printf("Error multiplying terms.  \%s", 
             mp_error_to_string(result));
      return EXIT_FAILURE;
   \}

   /* display */
   printf("number1 * number2 == \%lu", mp_get_int(&number1));

   /* free terms and return */
   mp_clear_multi(&number1, &number2, NULL);

   return EXIT_SUCCESS;
\}
\end{alltt}   

If this program succeeds it shall output the following.

\begin{alltt}
number1 * number2 == 262911
\end{alltt}

\section{Squaring}
Since squaring can be performed faster than multiplication it is performed it's own function instead of just using
mp\_mul().

\index{mp\_sqr}
\begin{alltt}
int mp_sqr (mp_int * a, mp_int * b);
\end{alltt}

Will square $a$ and store it in $b$.  Like the case of multiplication there are four different squaring
algorithms all which can be called from mp\_sqr().  It is ideal to use mp\_sqr over mp\_mul when squaring terms because
of the speed difference.  

\section{Tuning Polynomial Basis Routines}

Both of the Toom-Cook and Karatsuba multiplication algorithms are faster than the traditional $O(n^2)$ approach that
the Comba and baseline algorithms use.  At $O(n^{1.464973})$ and $O(n^{1.584962})$ running times respectively they require 
considerably less work.  For example, a 10000-digit multiplication would take roughly 724,000 single precision
multiplications with Toom-Cook or 100,000,000 single precision multiplications with the standard Comba (a factor
of 138).

So why not always use Karatsuba or Toom-Cook?   The simple answer is that they have so much overhead that they're not
actually faster than Comba until you hit distinct  ``cutoff'' points.  For Karatsuba with the default configuration, 
GCC 3.3.1 and an Athlon XP processor the cutoff point is roughly 110 digits (about 70 for the Intel P4).  That is, at 
110 digits Karatsuba and Comba multiplications just about break even and for 110+ digits Karatsuba is faster.

Toom-Cook has incredible overhead and is probably only useful for very large inputs.  So far no known cutoff points 
exist and for the most part I just set the cutoff points very high to make sure they're not called.

A demo program in the ``etc/'' directory of the project called ``tune.c'' can be used to find the cutoff points.  This
can be built with GCC as follows

\begin{alltt}
make XXX
\end{alltt}
Where ``XXX'' is one of the following entries from the table \ref{fig:tuning}.

\begin{figure}[here]
\begin{center}
\begin{small}
\begin{tabular}{|l|l|}
\hline \textbf{Value of XXX} & \textbf{Meaning} \\
\hline tune & Builds portable tuning application \\
\hline tune86 & Builds x86 (pentium and up) program for COFF \\
\hline tune86c & Builds x86 program for Cygwin \\
\hline tune86l & Builds x86 program for Linux (ELF format) \\
\hline
\end{tabular}
\end{small}
\end{center}
\caption{Build Names for Tuning Programs}
\label{fig:tuning}
\end{figure}

When the program is running it will output a series of measurements for different cutoff points.  It will first find
good Karatsuba squaring and multiplication points.  Then it proceeds to find Toom-Cook points.  Note that the Toom-Cook
tuning takes a very long time as the cutoff points are likely to be very high.

\chapter{Modular Reduction}

Modular reduction is process of taking the remainder of one quantity divided by another.  Expressed 
as (\ref{eqn:mod}) the modular reduction is equivalent to the remainder of $b$ divided by $c$.  

\begin{equation}
a \equiv b \mbox{ (mod }c\mbox{)}
\label{eqn:mod}
\end{equation}

Of particular interest to cryptography are reductions where $b$ is limited to the range $0 \le b < c^2$ since particularly 
fast reduction algorithms can be written for the limited range.  

Note that one of the four optimized reduction algorithms are automatically chosen in the modular exponentiation
algorithm mp\_exptmod when an appropriate modulus is detected.  

\section{Straight Division}
In order to effect an arbitrary modular reduction the following algorithm is provided.

\index{mp\_mod}
\begin{alltt}
int mp_mod(mp_int *a, mp_int *b, mp_int *c);
\end{alltt}

This reduces $a$ modulo $b$ and stores the result in $c$.  The sign of $c$ shall agree with the sign 
of $b$.  This algorithm accepts an input $a$ of any range and is not limited by $0 \le a < b^2$.

\section{Barrett Reduction}

Barrett reduction is a generic optimized reduction algorithm that requires pre--computation to achieve
a decent speedup over straight division.  First a $\mu$ value must be precomputed with the following function.

\index{mp\_reduce\_setup}
\begin{alltt}
int mp_reduce_setup(mp_int *a, mp_int *b);
\end{alltt}

Given a modulus in $b$ this produces the required $\mu$ value in $a$.  For any given modulus this only has to
be computed once.  Modular reduction can now be performed with the following.

\index{mp\_reduce}
\begin{alltt}
int mp_reduce(mp_int *a, mp_int *b, mp_int *c);
\end{alltt}

This will reduce $a$ in place modulo $b$ with the precomputed $\mu$ value in $c$.  $a$ must be in the range
$0 \le a < b^2$.

\begin{alltt}
int main(void)
\{
   mp_int   a, b, c, mu;
   int      result;

   /* initialize a,b to desired values, mp_init mu, 
    * c and set c to 1...we want to compute a^3 mod b 
    */

   /* get mu value */
   if ((result = mp_reduce_setup(&mu, b)) != MP_OKAY) \{
      printf("Error getting mu.  \%s", 
             mp_error_to_string(result));
      return EXIT_FAILURE;
   \}

   /* square a to get c = a^2 */
   if ((result = mp_sqr(&a, &c)) != MP_OKAY) \{
      printf("Error squaring.  \%s", 
             mp_error_to_string(result));
      return EXIT_FAILURE;
   \}

   /* now reduce `c' modulo b */
   if ((result = mp_reduce(&c, &b, &mu)) != MP_OKAY) \{
      printf("Error reducing.  \%s", 
             mp_error_to_string(result));
      return EXIT_FAILURE;
   \}
   
   /* multiply a to get c = a^3 */
   if ((result = mp_mul(&a, &c, &c)) != MP_OKAY) \{
      printf("Error reducing.  \%s", 
             mp_error_to_string(result));
      return EXIT_FAILURE;
   \}

   /* now reduce `c' modulo b  */
   if ((result = mp_reduce(&c, &b, &mu)) != MP_OKAY) \{
      printf("Error reducing.  \%s", 
             mp_error_to_string(result));
      return EXIT_FAILURE;
   \}
  
   /* c now equals a^3 mod b */

   return EXIT_SUCCESS;
\}
\end{alltt} 

This program will calculate $a^3 \mbox{ mod }b$ if all the functions succeed.  

\section{Montgomery Reduction}

Montgomery is a specialized reduction algorithm for any odd moduli.  Like Barrett reduction a pre--computation
step is required.  This is accomplished with the following.

\index{mp\_montgomery\_setup}
\begin{alltt}
int mp_montgomery_setup(mp_int *a, mp_digit *mp);
\end{alltt}

For the given odd moduli $a$ the precomputation value is placed in $mp$.  The reduction is computed with the 
following.

\index{mp\_montgomery\_reduce}
\begin{alltt}
int mp_montgomery_reduce(mp_int *a, mp_int *m, mp_digit mp);
\end{alltt}
This reduces $a$ in place modulo $m$ with the pre--computed value $mp$.   $a$ must be in the range
$0 \le a < b^2$.

Montgomery reduction is faster than Barrett reduction for moduli smaller than the ``comba'' limit.  With the default
setup for instance, the limit is $127$ digits ($3556$--bits).   Note that this function is not limited to
$127$ digits just that it falls back to a baseline algorithm after that point.  

An important observation is that this reduction does not return $a \mbox{ mod }m$ but $aR^{-1} \mbox{ mod }m$ 
where $R = \beta^n$, $n$ is the n number of digits in $m$ and $\beta$ is radix used (default is $2^{28}$).  

To quickly calculate $R$ the following function was provided.

\index{mp\_montgomery\_calc\_normalization}
\begin{alltt}
int mp_montgomery_calc_normalization(mp_int *a, mp_int *b);
\end{alltt}
Which calculates $a = R$ for the odd moduli $b$ without using multiplication or division.  

The normal modus operandi for Montgomery reductions is to normalize the integers before entering the system.  For
example, to calculate $a^3 \mbox { mod }b$ using Montgomery reduction the value of $a$ can be normalized by
multiplying it by $R$.  Consider the following code snippet.

\begin{alltt}
int main(void)
\{
   mp_int   a, b, c, R;
   mp_digit mp;
   int      result;

   /* initialize a,b to desired values, 
    * mp_init R, c and set c to 1.... 
    */

   /* get normalization */
   if ((result = mp_montgomery_calc_normalization(&R, b)) != MP_OKAY) \{
      printf("Error getting norm.  \%s", 
             mp_error_to_string(result));
      return EXIT_FAILURE;
   \}

   /* get mp value */
   if ((result = mp_montgomery_setup(&c, &mp)) != MP_OKAY) \{
      printf("Error setting up montgomery.  \%s", 
             mp_error_to_string(result));
      return EXIT_FAILURE;
   \}

   /* normalize `a' so now a is equal to aR */
   if ((result = mp_mulmod(&a, &R, &b, &a)) != MP_OKAY) \{
      printf("Error computing aR.  \%s", 
             mp_error_to_string(result));
      return EXIT_FAILURE;
   \}

   /* square a to get c = a^2R^2 */
   if ((result = mp_sqr(&a, &c)) != MP_OKAY) \{
      printf("Error squaring.  \%s", 
             mp_error_to_string(result));
      return EXIT_FAILURE;
   \}

   /* now reduce `c' back down to c = a^2R^2 * R^-1 == a^2R */
   if ((result = mp_montgomery_reduce(&c, &b, mp)) != MP_OKAY) \{
      printf("Error reducing.  \%s", 
             mp_error_to_string(result));
      return EXIT_FAILURE;
   \}
   
   /* multiply a to get c = a^3R^2 */
   if ((result = mp_mul(&a, &c, &c)) != MP_OKAY) \{
      printf("Error reducing.  \%s", 
             mp_error_to_string(result));
      return EXIT_FAILURE;
   \}

   /* now reduce `c' back down to c = a^3R^2 * R^-1 == a^3R */
   if ((result = mp_montgomery_reduce(&c, &b, mp)) != MP_OKAY) \{
      printf("Error reducing.  \%s", 
             mp_error_to_string(result));
      return EXIT_FAILURE;
   \}
   
   /* now reduce (again) `c' back down to c = a^3R * R^-1 == a^3 */
   if ((result = mp_montgomery_reduce(&c, &b, mp)) != MP_OKAY) \{
      printf("Error reducing.  \%s", 
             mp_error_to_string(result));
      return EXIT_FAILURE;
   \}

   /* c now equals a^3 mod b */

   return EXIT_SUCCESS;
\}
\end{alltt} 

This particular example does not look too efficient but it demonstrates the point of the algorithm.  By 
normalizing the inputs the reduced results are always of the form $aR$ for some variable $a$.  This allows
a single final reduction to correct for the normalization and the fast reduction used within the algorithm.

For more details consider examining the file \textit{bn\_mp\_exptmod\_fast.c}.

\section{Restricted Dimminished Radix}

``Dimminished Radix'' reduction refers to reduction with respect to moduli that are ameniable to simple
digit shifting and small multiplications.  In this case the ``restricted'' variant refers to moduli of the
form $\beta^k - p$ for some $k \ge 0$ and $0 < p < \beta$ where $\beta$ is the radix (default to $2^{28}$).  

As in the case of Montgomery reduction there is a pre--computation phase required for a given modulus.

\index{mp\_dr\_setup}
\begin{alltt}
void mp_dr_setup(mp_int *a, mp_digit *d);
\end{alltt}

This computes the value required for the modulus $a$ and stores it in $d$.  This function cannot fail
and does not return any error codes.  After the pre--computation a reduction can be performed with the
following.

\index{mp\_dr\_reduce}
\begin{alltt}
int mp_dr_reduce(mp_int *a, mp_int *b, mp_digit mp);
\end{alltt}

This reduces $a$ in place modulo $b$ with the pre--computed value $mp$.  $b$ must be of a restricted
dimminished radix form and $a$ must be in the range $0 \le a < b^2$.  Dimminished radix reductions are 
much faster than both Barrett and Montgomery reductions as they have a much lower asymtotic running time.  

Since the moduli are restricted this algorithm is not particularly useful for something like Rabin, RSA or
BBS cryptographic purposes.  This reduction algorithm is useful for Diffie-Hellman and ECC where fixed
primes are acceptable.  

Note that unlike Montgomery reduction there is no normalization process.  The result of this function is
equal to the correct residue.

\section{Unrestricted Dimminshed Radix}

Unrestricted reductions work much like the restricted counterparts except in this case the moduli is of the 
form $2^k - p$ for $0 < p < \beta$.  In this sense the unrestricted reductions are more flexible as they 
can be applied to a wider range of numbers.  

\index{mp\_reduce\_2k\_setup}
\begin{alltt}
int mp_reduce_2k_setup(mp_int *a, mp_digit *d);
\end{alltt}

This will compute the required $d$ value for the given moduli $a$.  

\index{mp\_reduce\_2k}
\begin{alltt}
int mp_reduce_2k(mp_int *a, mp_int *n, mp_digit d);
\end{alltt}

This will reduce $a$ in place modulo $n$ with the pre--computed value $d$.  From my experience this routine is 
slower than mp\_dr\_reduce but faster for most moduli sizes than the Montgomery reduction.  

\chapter{Exponentiation}
\section{Single Digit Exponentiation}
\index{mp\_expt\_d}
\begin{alltt}
int mp_expt_d (mp_int * a, mp_digit b, mp_int * c)
\end{alltt}
This computes $c = a^b$ using a simple binary left-to-right algorithm.  It is faster than repeated multiplications by 
$a$ for all values of $b$ greater than three.  

\section{Modular Exponentiation}
\index{mp\_exptmod}
\begin{alltt}
int mp_exptmod (mp_int * G, mp_int * X, mp_int * P, mp_int * Y)
\end{alltt}
This computes $Y \equiv G^X \mbox{ (mod }P\mbox{)}$ using a variable width sliding window algorithm.  This function
will automatically detect the fastest modular reduction technique to use during the operation.  For negative values of 
$X$ the operation is performed as $Y \equiv (G^{-1} \mbox{ mod }P)^{\vert X \vert} \mbox{ (mod }P\mbox{)}$ provided that 
$gcd(G, P) = 1$.

This function is actually a shell around the two internal exponentiation functions.  This routine will automatically
detect when Barrett, Montgomery, Restricted and Unrestricted Dimminished Radix based exponentiation can be used.  Generally
moduli of the a ``restricted dimminished radix'' form lead to the fastest modular exponentiations.  Followed by Montgomery
and the other two algorithms.

\section{Root Finding}
\index{mp\_n\_root}
\begin{alltt}
int mp_n_root (mp_int * a, mp_digit b, mp_int * c)
\end{alltt}
This computes $c = a^{1/b}$ such that $c^b \le a$ and $(c+1)^b > a$.  The implementation of this function is not 
ideal for values of $b$ greater than three.  It will work but become very slow.  So unless you are working with very small
numbers (less than 1000 bits) I'd avoid $b > 3$ situations.  Will return a positive root only for even roots and return
a root with the sign of the input for odd roots.  For example, performing $4^{1/2}$ will return $2$ whereas $(-8)^{1/3}$ 
will return $-2$.  

This algorithm uses the ``Newton Approximation'' method and will converge on the correct root fairly quickly.  Since
the algorithm requires raising $a$ to the power of $b$ it is not ideal to attempt to find roots for large
values of $b$.  If particularly large roots are required then a factor method could be used instead.  For example,
$a^{1/16}$ is equivalent to $\left (a^{1/4} \right)^{1/4}$ or simply 
$\left ( \left ( \left ( a^{1/2} \right )^{1/2} \right )^{1/2} \right )^{1/2}$

\chapter{Prime Numbers}
\section{Trial Division}
\index{mp\_prime\_is\_divisible}
\begin{alltt}
int mp_prime_is_divisible (mp_int * a, int *result)
\end{alltt}
This will attempt to evenly divide $a$ by a list of primes\footnote{Default is the first 256 primes.} and store the 
outcome in ``result''.  That is if $result = 0$ then $a$ is not divisible by the primes, otherwise it is.  Note that 
if the function does not return \textbf{MP\_OKAY} the value in ``result'' should be considered undefined\footnote{Currently
the default is to set it to zero first.}.

\section{Fermat Test}
\index{mp\_prime\_fermat}
\begin{alltt}
int mp_prime_fermat (mp_int * a, mp_int * b, int *result)
\end{alltt}
Performs a Fermat primality test to the base $b$.  That is it computes $b^a \mbox{ mod }a$ and tests whether the value is
equal to $b$ or not.  If the values are equal then $a$ is probably prime and $result$ is set to one.  Otherwise $result$
is set to zero.

\section{Miller-Rabin Test}
\index{mp\_prime\_miller\_rabin}
\begin{alltt}
int mp_prime_miller_rabin (mp_int * a, mp_int * b, int *result)
\end{alltt}
Performs a Miller-Rabin test to the base $b$ of $a$.  This test is much stronger than the Fermat test and is very hard to
fool (besides with Carmichael numbers).  If $a$ passes the test (therefore is probably prime) $result$ is set to one.  
Otherwise $result$ is set to zero.  

Note that is suggested that you use the Miller-Rabin test instead of the Fermat test since all of the failures of 
Miller-Rabin are a subset of the failures of the Fermat test.

\subsection{Required Number of Tests}
Generally to ensure a number is very likely to be prime you have to perform the Miller-Rabin with at least a half-dozen
or so unique bases.  However, it has been proven that the probability of failure goes down as the size of the input goes up.
This is why a simple function has been provided to help out.

\index{mp\_prime\_rabin\_miller\_trials}
\begin{alltt}
int mp_prime_rabin_miller_trials(int size)
\end{alltt}
This returns the number of trials required for a $2^{-96}$ (or lower) probability of failure for a given ``size'' expressed
in bits.  This comes in handy specially since larger numbers are slower to test.  For example, a 512-bit number would
require ten tests whereas a 1024-bit number would only require four tests. 

You should always still perform a trial division before a Miller-Rabin test though.

\section{Primality Testing}
\index{mp\_prime\_is\_prime}
\begin{alltt}
int mp_prime_is_prime (mp_int * a, int t, int *result)
\end{alltt}
This will perform a trial division followed by $t$ rounds of Miller-Rabin tests on $a$ and store the result in $result$.  
If $a$ passes all of the tests $result$ is set to one, otherwise it is set to zero.  Note that $t$ is bounded by 
$1 \le t < PRIME\_SIZE$ where $PRIME\_SIZE$ is the number of primes in the prime number table (by default this is $256$).

\section{Next Prime}
\index{mp\_prime\_next\_prime}
\begin{alltt}
int mp_prime_next_prime(mp_int *a, int t, int bbs_style)
\end{alltt}
This finds the next prime after $a$ that passes mp\_prime\_is\_prime() with $t$ tests.  Set $bbs\_style$ to one if you 
want only the next prime congruent to $3 \mbox{ mod } 4$, otherwise set it to zero to find any next prime.  

\section{Random Primes}
\index{mp\_prime\_random}
\begin{alltt}
int mp_prime_random(mp_int *a, int t, int size, int bbs, 
                    ltm_prime_callback cb, void *dat)
\end{alltt}
This will find a prime greater than $256^{size}$ which can be ``bbs\_style'' or not depending on $bbs$ and must pass
$t$ rounds of tests.  The ``ltm\_prime\_callback'' is a typedef for 

\begin{alltt}
typedef int ltm_prime_callback(unsigned char *dst, int len, void *dat);
\end{alltt}

Which is a function that must read $len$ bytes (and return the amount stored) into $dst$.  The $dat$ variable is simply
copied from the original input.  It can be used to pass RNG context data to the callback.  The function 
mp\_prime\_random() is more suitable for generating primes which must be secret (as in the case of RSA) since there 
is no skew on the least significant bits.

\textit{Note:}  As of v0.30 of the LibTomMath library this function has been deprecated.  It is still available
but users are encouraged to use the new mp\_prime\_random\_ex() function instead.

\subsection{Extended Generation}
\index{mp\_prime\_random\_ex}
\begin{alltt}
int mp_prime_random_ex(mp_int *a,    int t, 
                       int     size, int flags, 
                       ltm_prime_callback cb, void *dat);
\end{alltt}
This will generate a prime in $a$ using $t$ tests of the primality testing algorithms.  The variable $size$
specifies the bit length of the prime desired.  The variable $flags$ specifies one of several options available
(see fig. \ref{fig:primeopts}) which can be OR'ed together.  The callback parameters are used as in 
mp\_prime\_random().

\begin{figure}[here]
\begin{center}
\begin{small}
\begin{tabular}{|r|l|}
\hline \textbf{Flag}         & \textbf{Meaning} \\
\hline LTM\_PRIME\_BBS       & Make the prime congruent to $3$ modulo $4$ \\
\hline LTM\_PRIME\_SAFE      & Make a prime $p$ such that $(p - 1)/2$ is also prime. \\
                             & This option implies LTM\_PRIME\_BBS as well. \\
\hline LTM\_PRIME\_2MSB\_OFF & Makes sure that the bit adjacent to the most significant bit \\
                             & Is forced to zero.  \\
\hline LTM\_PRIME\_2MSB\_ON  & Makes sure that the bit adjacent to the most significant bit \\
                             & Is forced to one. \\
\hline
\end{tabular}
\end{small}
\end{center}
\caption{Primality Generation Options}
\label{fig:primeopts}
\end{figure}

\chapter{Input and Output}
\section{ASCII Conversions}
\subsection{To ASCII}
\index{mp\_toradix}
\begin{alltt}
int mp_toradix (mp_int * a, char *str, int radix);
\end{alltt}
This still store $a$ in ``str'' as a base-``radix'' string of ASCII chars.  This function appends a NUL character
to terminate the string.  Valid values of ``radix'' line in the range $[2, 64]$.  To determine the size (exact) required
by the conversion before storing any data use the following function.

\index{mp\_radix\_size}
\begin{alltt}
int mp_radix_size (mp_int * a, int radix, int *size)
\end{alltt}
This stores in ``size'' the number of characters (including space for the NUL terminator) required.  Upon error this 
function returns an error code and ``size'' will be zero.  

\subsection{From ASCII}
\index{mp\_read\_radix}
\begin{alltt}
int mp_read_radix (mp_int * a, char *str, int radix);
\end{alltt}
This will read the base-``radix'' NUL terminated string from ``str'' into $a$.  It will stop reading when it reads a
character it does not recognize (which happens to include th NUL char... imagine that...).  A single leading $-$ sign
can be used to denote a negative number.

\section{Binary Conversions}

Converting an mp\_int to and from binary is another keen idea.

\index{mp\_unsigned\_bin\_size}
\begin{alltt}
int mp_unsigned_bin_size(mp_int *a);
\end{alltt}

This will return the number of bytes (octets) required to store the unsigned copy of the integer $a$.

\index{mp\_to\_unsigned\_bin}
\begin{alltt}
int mp_to_unsigned_bin(mp_int *a, unsigned char *b);
\end{alltt}
This will store $a$ into the buffer $b$ in big--endian format.  Fortunately this is exactly what DER (or is it ASN?)
requires.  It does not store the sign of the integer.

\index{mp\_read\_unsigned\_bin}
\begin{alltt}
int mp_read_unsigned_bin(mp_int *a, unsigned char *b, int c);
\end{alltt}
This will read in an unsigned big--endian array of bytes (octets) from $b$ of length $c$ into $a$.  The resulting
integer $a$ will always be positive.

For those who acknowledge the existence of negative numbers (heretic!) there are ``signed'' versions of the
previous functions.

\begin{alltt}
int mp_signed_bin_size(mp_int *a);
int mp_read_signed_bin(mp_int *a, unsigned char *b, int c);
int mp_to_signed_bin(mp_int *a, unsigned char *b);
\end{alltt}
They operate essentially the same as the unsigned copies except they prefix the data with zero or non--zero
byte depending on the sign.  If the sign is zpos (e.g. not negative) the prefix is zero, otherwise the prefix
is non--zero.  

\chapter{Algebraic Functions}
\section{Extended Euclidean Algorithm}
\index{mp\_exteuclid}
\begin{alltt}
int mp_exteuclid(mp_int *a, mp_int *b, 
                 mp_int *U1, mp_int *U2, mp_int *U3);
\end{alltt}

This finds the triple U1/U2/U3 using the Extended Euclidean algorithm such that the following equation holds.

\begin{equation}
a \cdot U1 + b \cdot U2 = U3
\end{equation}

Any of the U1/U2/U3 paramters can be set to \textbf{NULL} if they are not desired.  

\section{Greatest Common Divisor}
\index{mp\_gcd}
\begin{alltt}
int mp_gcd (mp_int * a, mp_int * b, mp_int * c)
\end{alltt}
This will compute the greatest common divisor of $a$ and $b$ and store it in $c$.

\section{Least Common Multiple}
\index{mp\_lcm}
\begin{alltt}
int mp_lcm (mp_int * a, mp_int * b, mp_int * c)
\end{alltt}
This will compute the least common multiple of $a$ and $b$ and store it in $c$.

\section{Jacobi Symbol}
\index{mp\_jacobi}
\begin{alltt}
int mp_jacobi (mp_int * a, mp_int * p, int *c)
\end{alltt}
This will compute the Jacobi symbol for $a$ with respect to $p$.  If $p$ is prime this essentially computes the Legendre
symbol.  The result is stored in $c$ and can take on one of three values $\lbrace -1, 0, 1 \rbrace$.  If $p$ is prime
then the result will be $-1$ when $a$ is not a quadratic residue modulo $p$.  The result will be $0$ if $a$ divides $p$
and the result will be $1$ if $a$ is a quadratic residue modulo $p$.  

\section{Modular Inverse}
\index{mp\_invmod}
\begin{alltt}
int mp_invmod (mp_int * a, mp_int * b, mp_int * c)
\end{alltt}
Computes the multiplicative inverse of $a$ modulo $b$ and stores the result in $c$ such that $ac \equiv 1 \mbox{ (mod }b\mbox{)}$.

\section{Single Digit Functions}

For those using small numbers (\textit{snicker snicker}) there are several ``helper'' functions

\index{mp\_add\_d} \index{mp\_sub\_d} \index{mp\_mul\_d} \index{mp\_div\_d} \index{mp\_mod\_d}
\begin{alltt}
int mp_add_d(mp_int *a, mp_digit b, mp_int *c);
int mp_sub_d(mp_int *a, mp_digit b, mp_int *c);
int mp_mul_d(mp_int *a, mp_digit b, mp_int *c);
int mp_div_d(mp_int *a, mp_digit b, mp_int *c, mp_digit *d);
int mp_mod_d(mp_int *a, mp_digit b, mp_digit *c);
\end{alltt}

These work like the full mp\_int capable variants except the second parameter $b$ is a mp\_digit.  These
functions fairly handy if you have to work with relatively small numbers since you will not have to allocate
an entire mp\_int to store a number like $1$ or $2$.

\documentclass[b5paper]{book}
\usepackage{hyperref}
\usepackage{makeidx}
\usepackage{amssymb}
\usepackage{color}
\usepackage{alltt}
\usepackage{graphicx}
\usepackage{layout}
\def\union{\cup}
\def\intersect{\cap}
\def\getsrandom{\stackrel{\rm R}{\gets}}
\def\cross{\times}
\def\cat{\hspace{0.5em} \| \hspace{0.5em}}
\def\catn{$\|$}
\def\divides{\hspace{0.3em} | \hspace{0.3em}}
\def\nequiv{\not\equiv}
\def\approx{\raisebox{0.2ex}{\mbox{\small $\sim$}}}
\def\lcm{{\rm lcm}}
\def\gcd{{\rm gcd}}
\def\log{{\rm log}}
\def\ord{{\rm ord}}
\def\abs{{\mathit abs}}
\def\rep{{\mathit rep}}
\def\mod{{\mathit\ mod\ }}
\renewcommand{\pmod}[1]{\ ({\rm mod\ }{#1})}
\newcommand{\floor}[1]{\left\lfloor{#1}\right\rfloor}
\newcommand{\ceil}[1]{\left\lceil{#1}\right\rceil}
\def\Or{{\rm\ or\ }}
\def\And{{\rm\ and\ }}
\def\iff{\hspace{1em}\Longleftrightarrow\hspace{1em}}
\def\implies{\Rightarrow}
\def\undefined{{\rm ``undefined"}}
\def\Proof{\vspace{1ex}\noindent {\bf Proof:}\hspace{1em}}
\let\oldphi\phi
\def\phi{\varphi}
\def\Pr{{\rm Pr}}
\newcommand{\str}[1]{{\mathbf{#1}}}
\def\F{{\mathbb F}}
\def\N{{\mathbb N}}
\def\Z{{\mathbb Z}}
\def\R{{\mathbb R}}
\def\C{{\mathbb C}}
\def\Q{{\mathbb Q}}
\definecolor{DGray}{gray}{0.5}
\newcommand{\emailaddr}[1]{\mbox{$<${#1}$>$}}
\def\twiddle{\raisebox{0.3ex}{\mbox{\tiny $\sim$}}}
\def\gap{\vspace{0.5ex}}
\makeindex
\begin{document}
\frontmatter
\pagestyle{empty}
\title{LibTomMath User Manual \\ v0.39}
\author{Tom St Denis \\ tomstdenis@iahu.ca}
\maketitle
This text, the library and the accompanying textbook are all hereby placed in the public domain.  This book has been 
formatted for B5 [176x250] paper using the \LaTeX{} {\em book} macro package.

\vspace{10cm}

\begin{flushright}Open Source.  Open Academia.  Open Minds.

\mbox{ }

Tom St Denis,

Ontario, Canada
\end{flushright}

\tableofcontents
\listoffigures
\mainmatter
\pagestyle{headings}
\chapter{Introduction}
\section{What is LibTomMath?}
LibTomMath is a library of source code which provides a series of efficient and carefully written functions for manipulating
large integer numbers.  It was written in portable ISO C source code so that it will build on any platform with a conforming
C compiler.  

In a nutshell the library was written from scratch with verbose comments to help instruct computer science students how
to implement ``bignum'' math.  However, the resulting code has proven to be very useful.  It has been used by numerous 
universities, commercial and open source software developers.  It has been used on a variety of platforms ranging from
Linux and Windows based x86 to ARM based Gameboys and PPC based MacOS machines.  

\section{License}
As of the v0.25 the library source code has been placed in the public domain with every new release.  As of the v0.28
release the textbook ``Implementing Multiple Precision Arithmetic'' has been placed in the public domain with every new
release as well.  This textbook is meant to compliment the project by providing a more solid walkthrough of the development
algorithms used in the library.

Since both\footnote{Note that the MPI files under mtest/ are copyrighted by Michael Fromberger.  They are not required to use LibTomMath.} are in the 
public domain everyone is entitled to do with them as they see fit.

\section{Building LibTomMath}

LibTomMath is meant to be very ``GCC friendly'' as it comes with a makefile well suited for GCC.  However, the library will
also build in MSVC, Borland C out of the box.  For any other ISO C compiler a makefile will have to be made by the end
developer.  

\subsection{Static Libraries}
To build as a static library for GCC issue the following
\begin{alltt}
make
\end{alltt}

command.  This will build the library and archive the object files in ``libtommath.a''.  Now you link against 
that and include ``tommath.h'' within your programs.  Alternatively to build with MSVC issue the following
\begin{alltt}
nmake -f makefile.msvc
\end{alltt}

This will build the library and archive the object files in ``tommath.lib''.  This has been tested with MSVC 
version 6.00 with service pack 5.  

\subsection{Shared Libraries}
To build as a shared library for GCC issue the following
\begin{alltt}
make -f makefile.shared
\end{alltt}
This requires the ``libtool'' package (common on most Linux/BSD systems).  It will build LibTomMath as both shared
and static then install (by default) into /usr/lib as well as install the header files in /usr/include.  The shared 
library (resource) will be called ``libtommath.la'' while the static library called ``libtommath.a''.  Generally 
you use libtool to link your application against the shared object.  

There is limited support for making a ``DLL'' in windows via the ``makefile.cygwin\_dll'' makefile.  It requires 
Cygwin to work with since it requires the auto-export/import functionality.  The resulting DLL and import library 
``libtommath.dll.a'' can be used to link LibTomMath dynamically to any Windows program using Cygwin.

\subsection{Testing}
To build the library and the test harness type

\begin{alltt}
make test
\end{alltt}

This will build the library, ``test'' and ``mtest/mtest''.  The ``test'' program will accept test vectors and verify the
results.  ``mtest/mtest'' will generate test vectors using the MPI library by Michael Fromberger\footnote{A copy of MPI
is included in the package}.  Simply pipe mtest into test using

\begin{alltt}
mtest/mtest | test
\end{alltt}

If you do not have a ``/dev/urandom'' style RNG source you will have to write your own PRNG and simply pipe that into 
mtest.  For example, if your PRNG program is called ``myprng'' simply invoke

\begin{alltt}
myprng | mtest/mtest | test
\end{alltt}

This will output a row of numbers that are increasing.  Each column is a different test (such as addition, multiplication, etc)
that is being performed.  The numbers represent how many times the test was invoked.  If an error is detected the program
will exit with a dump of the relevent numbers it was working with.

\section{Build Configuration}
LibTomMath can configured at build time in three phases we shall call ``depends'', ``tweaks'' and ``trims''.  
Each phase changes how the library is built and they are applied one after another respectively.  

To make the system more powerful you can tweak the build process.  Classes are defined in the file
``tommath\_superclass.h''.  By default, the symbol ``LTM\_ALL'' shall be defined which simply 
instructs the system to build all of the functions.  This is how LibTomMath used to be packaged.  This will give you 
access to every function LibTomMath offers.

However, there are cases where such a build is not optional.  For instance, you want to perform RSA operations.  You 
don't need the vast majority of the library to perform these operations.  Aside from LTM\_ALL there is 
another pre--defined class ``SC\_RSA\_1'' which works in conjunction with the RSA from LibTomCrypt.  Additional 
classes can be defined base on the need of the user.

\subsection{Build Depends}
In the file tommath\_class.h you will see a large list of C ``defines'' followed by a series of ``ifdefs''
which further define symbols.  All of the symbols (technically they're macros $\ldots$) represent a given C source
file.  For instance, BN\_MP\_ADD\_C represents the file ``bn\_mp\_add.c''.  When a define has been enabled the
function in the respective file will be compiled and linked into the library.  Accordingly when the define
is absent the file will not be compiled and not contribute any size to the library.

You will also note that the header tommath\_class.h is actually recursively included (it includes itself twice).  
This is to help resolve as many dependencies as possible.  In the last pass the symbol LTM\_LAST will be defined.  
This is useful for ``trims''.

\subsection{Build Tweaks}
A tweak is an algorithm ``alternative''.  For example, to provide tradeoffs (usually between size and space).
They can be enabled at any pass of the configuration phase.

\begin{small}
\begin{center}
\begin{tabular}{|l|l|}
\hline \textbf{Define} & \textbf{Purpose} \\
\hline BN\_MP\_DIV\_SMALL & Enables a slower, smaller and equally \\
                          & functional mp\_div() function \\
\hline
\end{tabular}
\end{center}
\end{small}

\subsection{Build Trims}
A trim is a manner of removing functionality from a function that is not required.  For instance, to perform
RSA cryptography you only require exponentiation with odd moduli so even moduli support can be safely removed.  
Build trims are meant to be defined on the last pass of the configuration which means they are to be defined
only if LTM\_LAST has been defined.

\subsubsection{Moduli Related}
\begin{small}
\begin{center}
\begin{tabular}{|l|l|}
\hline \textbf{Restriction} & \textbf{Undefine} \\
\hline Exponentiation with odd moduli only & BN\_S\_MP\_EXPTMOD\_C \\
                                           & BN\_MP\_REDUCE\_C \\
                                           & BN\_MP\_REDUCE\_SETUP\_C \\
                                           & BN\_S\_MP\_MUL\_HIGH\_DIGS\_C \\
                                           & BN\_FAST\_S\_MP\_MUL\_HIGH\_DIGS\_C \\
\hline Exponentiation with random odd moduli & (The above plus the following) \\
                                           & BN\_MP\_REDUCE\_2K\_C \\
                                           & BN\_MP\_REDUCE\_2K\_SETUP\_C \\
                                           & BN\_MP\_REDUCE\_IS\_2K\_C \\
                                           & BN\_MP\_DR\_IS\_MODULUS\_C \\
                                           & BN\_MP\_DR\_REDUCE\_C \\
                                           & BN\_MP\_DR\_SETUP\_C \\
\hline Modular inverse odd moduli only     & BN\_MP\_INVMOD\_SLOW\_C \\
\hline Modular inverse (both, smaller/slower) & BN\_FAST\_MP\_INVMOD\_C \\
\hline
\end{tabular}
\end{center}
\end{small}

\subsubsection{Operand Size Related}
\begin{small}
\begin{center}
\begin{tabular}{|l|l|}
\hline \textbf{Restriction} & \textbf{Undefine} \\
\hline Moduli $\le 2560$ bits              & BN\_MP\_MONTGOMERY\_REDUCE\_C \\
                                           & BN\_S\_MP\_MUL\_DIGS\_C \\
                                           & BN\_S\_MP\_MUL\_HIGH\_DIGS\_C \\
                                           & BN\_S\_MP\_SQR\_C \\
\hline Polynomial Schmolynomial            & BN\_MP\_KARATSUBA\_MUL\_C \\
                                           & BN\_MP\_KARATSUBA\_SQR\_C \\
                                           & BN\_MP\_TOOM\_MUL\_C \\ 
                                           & BN\_MP\_TOOM\_SQR\_C \\

\hline
\end{tabular}
\end{center}
\end{small}


\section{Purpose of LibTomMath}
Unlike  GNU MP (GMP) Library, LIP, OpenSSL or various other commercial kits (Miracl), LibTomMath was not written with 
bleeding edge performance in mind.  First and foremost LibTomMath was written to be entirely open.  Not only is the 
source code public domain (unlike various other GPL/etc licensed code), not only is the code freely downloadable but the
source code is also accessible for computer science students attempting to learn ``BigNum'' or multiple precision
arithmetic techniques. 

LibTomMath was written to be an instructive collection of source code.  This is why there are many comments, only one
function per source file and often I use a ``middle-road'' approach where I don't cut corners for an extra 2\% speed
increase.

Source code alone cannot really teach how the algorithms work which is why I also wrote a textbook that accompanies
the library (beat that!).

So you may be thinking ``should I use LibTomMath?'' and the answer is a definite maybe.  Let me tabulate what I think
are the pros and cons of LibTomMath by comparing it to the math routines from GnuPG\footnote{GnuPG v1.2.3 versus LibTomMath v0.28}.

\newpage\begin{figure}[here]
\begin{small}
\begin{center}
\begin{tabular}{|l|c|c|l|}
\hline \textbf{Criteria} & \textbf{Pro} & \textbf{Con} & \textbf{Notes} \\
\hline Few lines of code per file & X & & GnuPG $ = 300.9$, LibTomMath  $ = 71.97$ \\
\hline Commented function prototypes & X && GnuPG function names are cryptic. \\
\hline Speed && X & LibTomMath is slower.  \\
\hline Totally free & X & & GPL has unfavourable restrictions.\\
\hline Large function base & X & & GnuPG is barebones. \\
\hline Five modular reduction algorithms & X & & Faster modular exponentiation for a variety of moduli. \\
\hline Portable & X & & GnuPG requires configuration to build. \\
\hline
\end{tabular}
\end{center}
\end{small}
\caption{LibTomMath Valuation}
\end{figure}

It may seem odd to compare LibTomMath to GnuPG since the math in GnuPG is only a small portion of the entire application. 
However, LibTomMath was written with cryptography in mind.  It provides essentially all of the functions a cryptosystem
would require when working with large integers.  

So it may feel tempting to just rip the math code out of GnuPG (or GnuMP where it was taken from originally) in your
own application but I think there are reasons not to.  While LibTomMath is slower than libraries such as GnuMP it is
not normally significantly slower.  On x86 machines the difference is normally a factor of two when performing modular
exponentiations.  It depends largely on the processor, compiler and the moduli being used.

Essentially the only time you wouldn't use LibTomMath is when blazing speed is the primary concern.  However,
on the other side of the coin LibTomMath offers you a totally free (public domain) well structured math library
that is very flexible, complete and performs well in resource contrained environments.  Fast RSA for example can
be performed with as little as 8KB of ram for data (again depending on build options).  

\chapter{Getting Started with LibTomMath}
\section{Building Programs}
In order to use LibTomMath you must include ``tommath.h'' and link against the appropriate library file (typically 
libtommath.a).  There is no library initialization required and the entire library is thread safe.

\section{Return Codes}
There are three possible return codes a function may return.

\index{MP\_OKAY}\index{MP\_YES}\index{MP\_NO}\index{MP\_VAL}\index{MP\_MEM}
\begin{figure}[here!]
\begin{center}
\begin{small}
\begin{tabular}{|l|l|}
\hline \textbf{Code} & \textbf{Meaning} \\
\hline MP\_OKAY & The function succeeded. \\
\hline MP\_VAL  & The function input was invalid. \\
\hline MP\_MEM  & Heap memory exhausted. \\
\hline &\\
\hline MP\_YES  & Response is yes. \\
\hline MP\_NO   & Response is no. \\
\hline
\end{tabular}
\end{small}
\end{center}
\caption{Return Codes}
\end{figure}

The last two codes listed are not actually ``return'ed'' by a function.  They are placed in an integer (the caller must
provide the address of an integer it can store to) which the caller can access.  To convert one of the three return codes
to a string use the following function.

\index{mp\_error\_to\_string}
\begin{alltt}
char *mp_error_to_string(int code);
\end{alltt}

This will return a pointer to a string which describes the given error code.  It will not work for the return codes 
MP\_YES and MP\_NO.  

\section{Data Types}
The basic ``multiple precision integer'' type is known as the ``mp\_int'' within LibTomMath.  This data type is used to
organize all of the data required to manipulate the integer it represents.  Within LibTomMath it has been prototyped
as the following.

\index{mp\_int}
\begin{alltt}
typedef struct  \{
    int used, alloc, sign;
    mp_digit *dp;
\} mp_int;
\end{alltt}

Where ``mp\_digit'' is a data type that represents individual digits of the integer.  By default, an mp\_digit is the
ISO C ``unsigned long'' data type and each digit is $28-$bits long.  The mp\_digit type can be configured to suit other
platforms by defining the appropriate macros.  

All LTM functions that use the mp\_int type will expect a pointer to mp\_int structure.  You must allocate memory to
hold the structure itself by yourself (whether off stack or heap it doesn't matter).  The very first thing that must be
done to use an mp\_int is that it must be initialized.

\section{Function Organization}

The arithmetic functions of the library are all organized to have the same style prototype.  That is source operands
are passed on the left and the destination is on the right.  For instance,

\begin{alltt}
mp_add(&a, &b, &c);       /* c = a + b */
mp_mul(&a, &a, &c);       /* c = a * a */
mp_div(&a, &b, &c, &d);   /* c = [a/b], d = a mod b */
\end{alltt}

Another feature of the way the functions have been implemented is that source operands can be destination operands as well.
For instance,

\begin{alltt}
mp_add(&a, &b, &b);       /* b = a + b */
mp_div(&a, &b, &a, &c);   /* a = [a/b], c = a mod b */
\end{alltt}

This allows operands to be re-used which can make programming simpler.

\section{Initialization}
\subsection{Single Initialization}
A single mp\_int can be initialized with the ``mp\_init'' function. 

\index{mp\_init}
\begin{alltt}
int mp_init (mp_int * a);
\end{alltt}

This function expects a pointer to an mp\_int structure and will initialize the members of the structure so the mp\_int
represents the default integer which is zero.  If the functions returns MP\_OKAY then the mp\_int is ready to be used
by the other LibTomMath functions.

\begin{small} \begin{alltt}
int main(void)
\{
   mp_int number;
   int result;

   if ((result = mp_init(&number)) != MP_OKAY) \{
      printf("Error initializing the number.  \%s", 
             mp_error_to_string(result));
      return EXIT_FAILURE;
   \}
 
   /* use the number */

   return EXIT_SUCCESS;
\}
\end{alltt} \end{small}

\subsection{Single Free}
When you are finished with an mp\_int it is ideal to return the heap it used back to the system.  The following function 
provides this functionality.

\index{mp\_clear}
\begin{alltt}
void mp_clear (mp_int * a);
\end{alltt}

The function expects a pointer to a previously initialized mp\_int structure and frees the heap it uses.  It sets the 
pointer\footnote{The ``dp'' member.} within the mp\_int to \textbf{NULL} which is used to prevent double free situations. 
Is is legal to call mp\_clear() twice on the same mp\_int in a row.  

\begin{small} \begin{alltt}
int main(void)
\{
   mp_int number;
   int result;

   if ((result = mp_init(&number)) != MP_OKAY) \{
      printf("Error initializing the number.  \%s", 
             mp_error_to_string(result));
      return EXIT_FAILURE;
   \}
 
   /* use the number */

   /* We're done with it. */
   mp_clear(&number);

   return EXIT_SUCCESS;
\}
\end{alltt} \end{small}

\subsection{Multiple Initializations}
Certain algorithms require more than one large integer.  In these instances it is ideal to initialize all of the mp\_int
variables in an ``all or nothing'' fashion.  That is, they are either all initialized successfully or they are all
not initialized.

The  mp\_init\_multi() function provides this functionality.

\index{mp\_init\_multi} \index{mp\_clear\_multi}
\begin{alltt}
int mp_init_multi(mp_int *mp, ...);
\end{alltt}

It accepts a \textbf{NULL} terminated list of pointers to mp\_int structures.  It will attempt to initialize them all
at once.  If the function returns MP\_OKAY then all of the mp\_int variables are ready to use, otherwise none of them
are available for use.  A complementary mp\_clear\_multi() function allows multiple mp\_int variables to be free'd 
from the heap at the same time.  

\begin{small} \begin{alltt}
int main(void)
\{
   mp_int num1, num2, num3;
   int result;

   if ((result = mp_init_multi(&num1, 
                               &num2,
                               &num3, NULL)) != MP\_OKAY) \{      
      printf("Error initializing the numbers.  \%s", 
             mp_error_to_string(result));
      return EXIT_FAILURE;
   \}
 
   /* use the numbers */

   /* We're done with them. */
   mp_clear_multi(&num1, &num2, &num3, NULL);

   return EXIT_SUCCESS;
\}
\end{alltt} \end{small}

\subsection{Other Initializers}
To initialized and make a copy of an mp\_int the mp\_init\_copy() function has been provided.  

\index{mp\_init\_copy}
\begin{alltt}
int mp_init_copy (mp_int * a, mp_int * b);
\end{alltt}

This function will initialize $a$ and make it a copy of $b$ if all goes well.

\begin{small} \begin{alltt}
int main(void)
\{
   mp_int num1, num2;
   int result;

   /* initialize and do work on num1 ... */

   /* We want a copy of num1 in num2 now */
   if ((result = mp_init_copy(&num2, &num1)) != MP_OKAY) \{
     printf("Error initializing the copy.  \%s", 
             mp_error_to_string(result));
      return EXIT_FAILURE;
   \}
 
   /* now num2 is ready and contains a copy of num1 */

   /* We're done with them. */
   mp_clear_multi(&num1, &num2, NULL);

   return EXIT_SUCCESS;
\}
\end{alltt} \end{small}

Another less common initializer is mp\_init\_size() which allows the user to initialize an mp\_int with a given
default number of digits.  By default, all initializers allocate \textbf{MP\_PREC} digits.  This function lets
you override this behaviour.

\index{mp\_init\_size}
\begin{alltt}
int mp_init_size (mp_int * a, int size);
\end{alltt}

The $size$ parameter must be greater than zero.  If the function succeeds the mp\_int $a$ will be initialized
to have $size$ digits (which are all initially zero).  

\begin{small} \begin{alltt}
int main(void)
\{
   mp_int number;
   int result;

   /* we need a 60-digit number */
   if ((result = mp_init_size(&number, 60)) != MP_OKAY) \{
      printf("Error initializing the number.  \%s", 
             mp_error_to_string(result));
      return EXIT_FAILURE;
   \}
 
   /* use the number */

   return EXIT_SUCCESS;
\}
\end{alltt} \end{small}

\section{Maintenance Functions}

\subsection{Reducing Memory Usage}
When an mp\_int is in a state where it won't be changed again\footnote{A Diffie-Hellman modulus for instance.} excess
digits can be removed to return memory to the heap with the mp\_shrink() function.

\index{mp\_shrink}
\begin{alltt}
int mp_shrink (mp_int * a);
\end{alltt}

This will remove excess digits of the mp\_int $a$.  If the operation fails the mp\_int should be intact without the
excess digits being removed.  Note that you can use a shrunk mp\_int in further computations, however, such operations
will require heap operations which can be slow.  It is not ideal to shrink mp\_int variables that you will further
modify in the system (unless you are seriously low on memory).  

\begin{small} \begin{alltt}
int main(void)
\{
   mp_int number;
   int result;

   if ((result = mp_init(&number)) != MP_OKAY) \{
      printf("Error initializing the number.  \%s", 
             mp_error_to_string(result));
      return EXIT_FAILURE;
   \}
 
   /* use the number [e.g. pre-computation]  */

   /* We're done with it for now. */
   if ((result = mp_shrink(&number)) != MP_OKAY) \{
      printf("Error shrinking the number.  \%s", 
             mp_error_to_string(result));
      return EXIT_FAILURE;
   \}

   /* use it .... */


   /* we're done with it. */ 
   mp_clear(&number);

   return EXIT_SUCCESS;
\}
\end{alltt} \end{small}

\subsection{Adding additional digits}

Within the mp\_int structure are two parameters which control the limitations of the array of digits that represent
the integer the mp\_int is meant to equal.   The \textit{used} parameter dictates how many digits are significant, that is,
contribute to the value of the mp\_int.  The \textit{alloc} parameter dictates how many digits are currently available in
the array.  If you need to perform an operation that requires more digits you will have to mp\_grow() the mp\_int to
your desired size.  

\index{mp\_grow}
\begin{alltt}
int mp_grow (mp_int * a, int size);
\end{alltt}

This will grow the array of digits of $a$ to $size$.  If the \textit{alloc} parameter is already bigger than
$size$ the function will not do anything.

\begin{small} \begin{alltt}
int main(void)
\{
   mp_int number;
   int result;

   if ((result = mp_init(&number)) != MP_OKAY) \{
      printf("Error initializing the number.  \%s", 
             mp_error_to_string(result));
      return EXIT_FAILURE;
   \}
 
   /* use the number */

   /* We need to add 20 digits to the number  */
   if ((result = mp_grow(&number, number.alloc + 20)) != MP_OKAY) \{
      printf("Error growing the number.  \%s", 
             mp_error_to_string(result));
      return EXIT_FAILURE;
   \}


   /* use the number */

   /* we're done with it. */ 
   mp_clear(&number);

   return EXIT_SUCCESS;
\}
\end{alltt} \end{small}

\chapter{Basic Operations}
\section{Small Constants}
Setting mp\_ints to small constants is a relatively common operation.  To accomodate these instances there are two
small constant assignment functions.  The first function is used to set a single digit constant while the second sets
an ISO C style ``unsigned long'' constant.  The reason for both functions is efficiency.  Setting a single digit is quick but the
domain of a digit can change (it's always at least $0 \ldots 127$).  

\subsection{Single Digit}

Setting a single digit can be accomplished with the following function.

\index{mp\_set}
\begin{alltt}
void mp_set (mp_int * a, mp_digit b);
\end{alltt}

This will zero the contents of $a$ and make it represent an integer equal to the value of $b$.  Note that this
function has a return type of \textbf{void}.  It cannot cause an error so it is safe to assume the function
succeeded.

\begin{small} \begin{alltt}
int main(void)
\{
   mp_int number;
   int result;

   if ((result = mp_init(&number)) != MP_OKAY) \{
      printf("Error initializing the number.  \%s", 
             mp_error_to_string(result));
      return EXIT_FAILURE;
   \}
 
   /* set the number to 5 */
   mp_set(&number, 5);

   /* we're done with it. */ 
   mp_clear(&number);

   return EXIT_SUCCESS;
\}
\end{alltt} \end{small}

\subsection{Long Constants}

To set a constant that is the size of an ISO C ``unsigned long'' and larger than a single digit the following function 
can be used.

\index{mp\_set\_int}
\begin{alltt}
int mp_set_int (mp_int * a, unsigned long b);
\end{alltt}

This will assign the value of the 32-bit variable $b$ to the mp\_int $a$.  Unlike mp\_set() this function will always
accept a 32-bit input regardless of the size of a single digit.  However, since the value may span several digits 
this function can fail if it runs out of heap memory.

To get the ``unsigned long'' copy of an mp\_int the following function can be used.

\index{mp\_get\_int}
\begin{alltt}
unsigned long mp_get_int (mp_int * a);
\end{alltt}

This will return the 32 least significant bits of the mp\_int $a$.  

\begin{small} \begin{alltt}
int main(void)
\{
   mp_int number;
   int result;

   if ((result = mp_init(&number)) != MP_OKAY) \{
      printf("Error initializing the number.  \%s", 
             mp_error_to_string(result));
      return EXIT_FAILURE;
   \}
 
   /* set the number to 654321 (note this is bigger than 127) */
   if ((result = mp_set_int(&number, 654321)) != MP_OKAY) \{
      printf("Error setting the value of the number.  \%s", 
             mp_error_to_string(result));
      return EXIT_FAILURE;
   \}

   printf("number == \%lu", mp_get_int(&number));

   /* we're done with it. */ 
   mp_clear(&number);

   return EXIT_SUCCESS;
\}
\end{alltt} \end{small}

This should output the following if the program succeeds.

\begin{alltt}
number == 654321
\end{alltt}

\subsection{Initialize and Setting Constants}
To both initialize and set small constants the following two functions are available.
\index{mp\_init\_set} \index{mp\_init\_set\_int}
\begin{alltt}
int mp_init_set (mp_int * a, mp_digit b);
int mp_init_set_int (mp_int * a, unsigned long b);
\end{alltt}

Both functions work like the previous counterparts except they first mp\_init $a$ before setting the values.  

\begin{alltt}
int main(void)
\{
   mp_int number1, number2;
   int    result;

   /* initialize and set a single digit */
   if ((result = mp_init_set(&number1, 100)) != MP_OKAY) \{
      printf("Error setting number1: \%s", 
             mp_error_to_string(result));
      return EXIT_FAILURE;
   \}             

   /* initialize and set a long */
   if ((result = mp_init_set_int(&number2, 1023)) != MP_OKAY) \{
      printf("Error setting number2: \%s", 
             mp_error_to_string(result));
      return EXIT_FAILURE;
   \}

   /* display */
   printf("Number1, Number2 == \%lu, \%lu",
          mp_get_int(&number1), mp_get_int(&number2));

   /* clear */
   mp_clear_multi(&number1, &number2, NULL);

   return EXIT_SUCCESS;
\}
\end{alltt}

If this program succeeds it shall output.
\begin{alltt}
Number1, Number2 == 100, 1023
\end{alltt}

\section{Comparisons}

Comparisons in LibTomMath are always performed in a ``left to right'' fashion.  There are three possible return codes
for any comparison.

\index{MP\_GT} \index{MP\_EQ} \index{MP\_LT}
\begin{figure}[here]
\begin{center}
\begin{tabular}{|c|c|}
\hline \textbf{Result Code} & \textbf{Meaning} \\
\hline MP\_GT & $a > b$ \\
\hline MP\_EQ & $a = b$ \\
\hline MP\_LT & $a < b$ \\
\hline
\end{tabular}
\end{center}
\caption{Comparison Codes for $a, b$}
\label{fig:CMP}
\end{figure}

In figure \ref{fig:CMP} two integers $a$ and $b$ are being compared.  In this case $a$ is said to be ``to the left'' of 
$b$.  

\subsection{Unsigned comparison}

An unsigned comparison considers only the digits themselves and not the associated \textit{sign} flag of the 
mp\_int structures.  This is analogous to an absolute comparison.  The function mp\_cmp\_mag() will compare two
mp\_int variables based on their digits only. 

\index{mp\_cmp\_mag}
\begin{alltt}
int mp_cmp_mag(mp_int * a, mp_int * b);
\end{alltt}
This will compare $a$ to $b$ placing $a$ to the left of $b$.  This function cannot fail and will return one of the
three compare codes listed in figure \ref{fig:CMP}.

\begin{small} \begin{alltt}
int main(void)
\{
   mp_int number1, number2;
   int result;

   if ((result = mp_init_multi(&number1, &number2, NULL)) != MP_OKAY) \{
      printf("Error initializing the numbers.  \%s", 
             mp_error_to_string(result));
      return EXIT_FAILURE;
   \}
 
   /* set the number1 to 5 */
   mp_set(&number1, 5);
  
   /* set the number2 to -6 */
   mp_set(&number2, 6);
   if ((result = mp_neg(&number2, &number2)) != MP_OKAY) \{
      printf("Error negating number2.  \%s", 
             mp_error_to_string(result));
      return EXIT_FAILURE;
   \}

   switch(mp_cmp_mag(&number1, &number2)) \{
       case MP_GT:  printf("|number1| > |number2|"); break;
       case MP_EQ:  printf("|number1| = |number2|"); break;
       case MP_LT:  printf("|number1| < |number2|"); break;
   \}

   /* we're done with it. */ 
   mp_clear_multi(&number1, &number2, NULL);

   return EXIT_SUCCESS;
\}
\end{alltt} \end{small}

If this program\footnote{This function uses the mp\_neg() function which is discussed in section \ref{sec:NEG}.} completes 
successfully it should print the following.

\begin{alltt}
|number1| < |number2|
\end{alltt}

This is because $\vert -6 \vert = 6$ and obviously $5 < 6$.

\subsection{Signed comparison}

To compare two mp\_int variables based on their signed value the mp\_cmp() function is provided.

\index{mp\_cmp}
\begin{alltt}
int mp_cmp(mp_int * a, mp_int * b);
\end{alltt}

This will compare $a$ to the left of $b$.  It will first compare the signs of the two mp\_int variables.  If they
differ it will return immediately based on their signs.  If the signs are equal then it will compare the digits
individually.  This function will return one of the compare conditions codes listed in figure \ref{fig:CMP}.

\begin{small} \begin{alltt}
int main(void)
\{
   mp_int number1, number2;
   int result;

   if ((result = mp_init_multi(&number1, &number2, NULL)) != MP_OKAY) \{
      printf("Error initializing the numbers.  \%s", 
             mp_error_to_string(result));
      return EXIT_FAILURE;
   \}
 
   /* set the number1 to 5 */
   mp_set(&number1, 5);
  
   /* set the number2 to -6 */
   mp_set(&number2, 6);
   if ((result = mp_neg(&number2, &number2)) != MP_OKAY) \{
      printf("Error negating number2.  \%s", 
             mp_error_to_string(result));
      return EXIT_FAILURE;
   \}

   switch(mp_cmp(&number1, &number2)) \{
       case MP_GT:  printf("number1 > number2"); break;
       case MP_EQ:  printf("number1 = number2"); break;
       case MP_LT:  printf("number1 < number2"); break;
   \}

   /* we're done with it. */ 
   mp_clear_multi(&number1, &number2, NULL);

   return EXIT_SUCCESS;
\}
\end{alltt} \end{small}

If this program\footnote{This function uses the mp\_neg() function which is discussed in section \ref{sec:NEG}.} completes 
successfully it should print the following.

\begin{alltt}
number1 > number2
\end{alltt}

\subsection{Single Digit}

To compare a single digit against an mp\_int the following function has been provided.

\index{mp\_cmp\_d}
\begin{alltt}
int mp_cmp_d(mp_int * a, mp_digit b);
\end{alltt}

This will compare $a$ to the left of $b$ using a signed comparison.  Note that it will always treat $b$ as 
positive.  This function is rather handy when you have to compare against small values such as $1$ (which often
comes up in cryptography).  The function cannot fail and will return one of the tree compare condition codes
listed in figure \ref{fig:CMP}.


\begin{small} \begin{alltt}
int main(void)
\{
   mp_int number;
   int result;

   if ((result = mp_init(&number)) != MP_OKAY) \{
      printf("Error initializing the number.  \%s", 
             mp_error_to_string(result));
      return EXIT_FAILURE;
   \}
 
   /* set the number to 5 */
   mp_set(&number, 5);

   switch(mp_cmp_d(&number, 7)) \{
       case MP_GT:  printf("number > 7"); break;
       case MP_EQ:  printf("number = 7"); break;
       case MP_LT:  printf("number < 7"); break;
   \}

   /* we're done with it. */ 
   mp_clear(&number);

   return EXIT_SUCCESS;
\}
\end{alltt} \end{small}

If this program functions properly it will print out the following.

\begin{alltt}
number < 7
\end{alltt}

\section{Logical Operations}

Logical operations are operations that can be performed either with simple shifts or boolean operators such as
AND, XOR and OR directly.  These operations are very quick.

\subsection{Multiplication by two}

Multiplications and divisions by any power of two can be performed with quick logical shifts either left or
right depending on the operation.  

When multiplying or dividing by two a special case routine can be used which are as follows.
\index{mp\_mul\_2} \index{mp\_div\_2}
\begin{alltt}
int mp_mul_2(mp_int * a, mp_int * b);
int mp_div_2(mp_int * a, mp_int * b);
\end{alltt}

The former will assign twice $a$ to $b$ while the latter will assign half $a$ to $b$.  These functions are fast
since the shift counts and maskes are hardcoded into the routines.

\begin{small} \begin{alltt}
int main(void)
\{
   mp_int number;
   int result;

   if ((result = mp_init(&number)) != MP_OKAY) \{
      printf("Error initializing the number.  \%s", 
             mp_error_to_string(result));
      return EXIT_FAILURE;
   \}
 
   /* set the number to 5 */
   mp_set(&number, 5);

   /* multiply by two */
   if ((result = mp\_mul\_2(&number, &number)) != MP_OKAY) \{
      printf("Error multiplying the number.  \%s", 
             mp_error_to_string(result));
      return EXIT_FAILURE;
   \}
   switch(mp_cmp_d(&number, 7)) \{
       case MP_GT:  printf("2*number > 7"); break;
       case MP_EQ:  printf("2*number = 7"); break;
       case MP_LT:  printf("2*number < 7"); break;
   \}

   /* now divide by two */
   if ((result = mp\_div\_2(&number, &number)) != MP_OKAY) \{
      printf("Error dividing the number.  \%s", 
             mp_error_to_string(result));
      return EXIT_FAILURE;
   \}
   switch(mp_cmp_d(&number, 7)) \{
       case MP_GT:  printf("2*number/2 > 7"); break;
       case MP_EQ:  printf("2*number/2 = 7"); break;
       case MP_LT:  printf("2*number/2 < 7"); break;
   \}

   /* we're done with it. */ 
   mp_clear(&number);

   return EXIT_SUCCESS;
\}
\end{alltt} \end{small}

If this program is successful it will print out the following text.

\begin{alltt}
2*number > 7
2*number/2 < 7
\end{alltt}

Since $10 > 7$ and $5 < 7$.  To multiply by a power of two the following function can be used.

\index{mp\_mul\_2d}
\begin{alltt}
int mp_mul_2d(mp_int * a, int b, mp_int * c);
\end{alltt}

This will multiply $a$ by $2^b$ and store the result in ``c''.  If the value of $b$ is less than or equal to 
zero the function will copy $a$ to ``c'' without performing any further actions.  

To divide by a power of two use the following.

\index{mp\_div\_2d}
\begin{alltt}
int mp_div_2d (mp_int * a, int b, mp_int * c, mp_int * d);
\end{alltt}
Which will divide $a$ by $2^b$, store the quotient in ``c'' and the remainder in ``d'.  If $b \le 0$ then the
function simply copies $a$ over to ``c'' and zeroes $d$.  The variable $d$ may be passed as a \textbf{NULL}
value to signal that the remainder is not desired.

\subsection{Polynomial Basis Operations}

Strictly speaking the organization of the integers within the mp\_int structures is what is known as a 
``polynomial basis''.  This simply means a field element is stored by divisions of a radix.  For example, if
$f(x) = \sum_{i=0}^{k} y_ix^k$ for any vector $\vec y$ then the array of digits in $\vec y$ are said to be 
the polynomial basis representation of $z$ if $f(\beta) = z$ for a given radix $\beta$.  

To multiply by the polynomial $g(x) = x$ all you have todo is shift the digits of the basis left one place.  The
following function provides this operation.

\index{mp\_lshd}
\begin{alltt}
int mp_lshd (mp_int * a, int b);
\end{alltt}

This will multiply $a$ in place by $x^b$ which is equivalent to shifting the digits left $b$ places and inserting zeroes
in the least significant digits.  Similarly to divide by a power of $x$ the following function is provided.

\index{mp\_rshd}
\begin{alltt}
void mp_rshd (mp_int * a, int b)
\end{alltt}
This will divide $a$ in place by $x^b$ and discard the remainder.  This function cannot fail as it performs the operations
in place and no new digits are required to complete it.

\subsection{AND, OR and XOR Operations}

While AND, OR and XOR operations are not typical ``bignum functions'' they can be useful in several instances.  The
three functions are prototyped as follows.

\index{mp\_or} \index{mp\_and} \index{mp\_xor}
\begin{alltt}
int mp_or  (mp_int * a, mp_int * b, mp_int * c);
int mp_and (mp_int * a, mp_int * b, mp_int * c);
int mp_xor (mp_int * a, mp_int * b, mp_int * c);
\end{alltt}

Which compute $c = a \odot b$ where $\odot$ is one of OR, AND or XOR.  

\section{Addition and Subtraction}

To compute an addition or subtraction the following two functions can be used.

\index{mp\_add} \index{mp\_sub}
\begin{alltt}
int mp_add (mp_int * a, mp_int * b, mp_int * c);
int mp_sub (mp_int * a, mp_int * b, mp_int * c)
\end{alltt}

Which perform $c = a \odot b$ where $\odot$ is one of signed addition or subtraction.  The operations are fully sign
aware.

\section{Sign Manipulation}
\subsection{Negation}
\label{sec:NEG}
Simple integer negation can be performed with the following.

\index{mp\_neg}
\begin{alltt}
int mp_neg (mp_int * a, mp_int * b);
\end{alltt}

Which assigns $-a$ to $b$.  

\subsection{Absolute}
Simple integer absolutes can be performed with the following.

\index{mp\_neg}
\begin{alltt}
int mp_abs (mp_int * a, mp_int * b);
\end{alltt}

Which assigns $\vert a \vert$ to $b$.  

\section{Integer Division and Remainder}
To perform a complete and general integer division with remainder use the following function.

\index{mp\_div}
\begin{alltt}
int mp_div (mp_int * a, mp_int * b, mp_int * c, mp_int * d);
\end{alltt}
                                                        
This divides $a$ by $b$ and stores the quotient in $c$ and $d$.  The signed quotient is computed such that 
$bc + d = a$.  Note that either of $c$ or $d$ can be set to \textbf{NULL} if their value is not required.  If 
$b$ is zero the function returns \textbf{MP\_VAL}.  


\chapter{Multiplication and Squaring}
\section{Multiplication}
A full signed integer multiplication can be performed with the following.
\index{mp\_mul}
\begin{alltt}
int mp_mul (mp_int * a, mp_int * b, mp_int * c);
\end{alltt}
Which assigns the full signed product $ab$ to $c$.  This function actually breaks into one of four cases which are 
specific multiplication routines optimized for given parameters.  First there are the Toom-Cook multiplications which
should only be used with very large inputs.  This is followed by the Karatsuba multiplications which are for moderate
sized inputs.  Then followed by the Comba and baseline multipliers.

Fortunately for the developer you don't really need to know this unless you really want to fine tune the system.  mp\_mul()
will determine on its own\footnote{Some tweaking may be required.} what routine to use automatically when it is called.

\begin{alltt}
int main(void)
\{
   mp_int number1, number2;
   int result;

   /* Initialize the numbers */
   if ((result = mp_init_multi(&number1, 
                               &number2, NULL)) != MP_OKAY) \{
      printf("Error initializing the numbers.  \%s", 
             mp_error_to_string(result));
      return EXIT_FAILURE;
   \}

   /* set the terms */
   if ((result = mp_set_int(&number, 257)) != MP_OKAY) \{
      printf("Error setting number1.  \%s", 
             mp_error_to_string(result));
      return EXIT_FAILURE;
   \}
 
   if ((result = mp_set_int(&number2, 1023)) != MP_OKAY) \{
      printf("Error setting number2.  \%s", 
             mp_error_to_string(result));
      return EXIT_FAILURE;
   \}

   /* multiply them */
   if ((result = mp_mul(&number1, &number2,
                        &number1)) != MP_OKAY) \{
      printf("Error multiplying terms.  \%s", 
             mp_error_to_string(result));
      return EXIT_FAILURE;
   \}

   /* display */
   printf("number1 * number2 == \%lu", mp_get_int(&number1));

   /* free terms and return */
   mp_clear_multi(&number1, &number2, NULL);

   return EXIT_SUCCESS;
\}
\end{alltt}   

If this program succeeds it shall output the following.

\begin{alltt}
number1 * number2 == 262911
\end{alltt}

\section{Squaring}
Since squaring can be performed faster than multiplication it is performed it's own function instead of just using
mp\_mul().

\index{mp\_sqr}
\begin{alltt}
int mp_sqr (mp_int * a, mp_int * b);
\end{alltt}

Will square $a$ and store it in $b$.  Like the case of multiplication there are four different squaring
algorithms all which can be called from mp\_sqr().  It is ideal to use mp\_sqr over mp\_mul when squaring terms because
of the speed difference.  

\section{Tuning Polynomial Basis Routines}

Both of the Toom-Cook and Karatsuba multiplication algorithms are faster than the traditional $O(n^2)$ approach that
the Comba and baseline algorithms use.  At $O(n^{1.464973})$ and $O(n^{1.584962})$ running times respectively they require 
considerably less work.  For example, a 10000-digit multiplication would take roughly 724,000 single precision
multiplications with Toom-Cook or 100,000,000 single precision multiplications with the standard Comba (a factor
of 138).

So why not always use Karatsuba or Toom-Cook?   The simple answer is that they have so much overhead that they're not
actually faster than Comba until you hit distinct  ``cutoff'' points.  For Karatsuba with the default configuration, 
GCC 3.3.1 and an Athlon XP processor the cutoff point is roughly 110 digits (about 70 for the Intel P4).  That is, at 
110 digits Karatsuba and Comba multiplications just about break even and for 110+ digits Karatsuba is faster.

Toom-Cook has incredible overhead and is probably only useful for very large inputs.  So far no known cutoff points 
exist and for the most part I just set the cutoff points very high to make sure they're not called.

A demo program in the ``etc/'' directory of the project called ``tune.c'' can be used to find the cutoff points.  This
can be built with GCC as follows

\begin{alltt}
make XXX
\end{alltt}
Where ``XXX'' is one of the following entries from the table \ref{fig:tuning}.

\begin{figure}[here]
\begin{center}
\begin{small}
\begin{tabular}{|l|l|}
\hline \textbf{Value of XXX} & \textbf{Meaning} \\
\hline tune & Builds portable tuning application \\
\hline tune86 & Builds x86 (pentium and up) program for COFF \\
\hline tune86c & Builds x86 program for Cygwin \\
\hline tune86l & Builds x86 program for Linux (ELF format) \\
\hline
\end{tabular}
\end{small}
\end{center}
\caption{Build Names for Tuning Programs}
\label{fig:tuning}
\end{figure}

When the program is running it will output a series of measurements for different cutoff points.  It will first find
good Karatsuba squaring and multiplication points.  Then it proceeds to find Toom-Cook points.  Note that the Toom-Cook
tuning takes a very long time as the cutoff points are likely to be very high.

\chapter{Modular Reduction}

Modular reduction is process of taking the remainder of one quantity divided by another.  Expressed 
as (\ref{eqn:mod}) the modular reduction is equivalent to the remainder of $b$ divided by $c$.  

\begin{equation}
a \equiv b \mbox{ (mod }c\mbox{)}
\label{eqn:mod}
\end{equation}

Of particular interest to cryptography are reductions where $b$ is limited to the range $0 \le b < c^2$ since particularly 
fast reduction algorithms can be written for the limited range.  

Note that one of the four optimized reduction algorithms are automatically chosen in the modular exponentiation
algorithm mp\_exptmod when an appropriate modulus is detected.  

\section{Straight Division}
In order to effect an arbitrary modular reduction the following algorithm is provided.

\index{mp\_mod}
\begin{alltt}
int mp_mod(mp_int *a, mp_int *b, mp_int *c);
\end{alltt}

This reduces $a$ modulo $b$ and stores the result in $c$.  The sign of $c$ shall agree with the sign 
of $b$.  This algorithm accepts an input $a$ of any range and is not limited by $0 \le a < b^2$.

\section{Barrett Reduction}

Barrett reduction is a generic optimized reduction algorithm that requires pre--computation to achieve
a decent speedup over straight division.  First a $\mu$ value must be precomputed with the following function.

\index{mp\_reduce\_setup}
\begin{alltt}
int mp_reduce_setup(mp_int *a, mp_int *b);
\end{alltt}

Given a modulus in $b$ this produces the required $\mu$ value in $a$.  For any given modulus this only has to
be computed once.  Modular reduction can now be performed with the following.

\index{mp\_reduce}
\begin{alltt}
int mp_reduce(mp_int *a, mp_int *b, mp_int *c);
\end{alltt}

This will reduce $a$ in place modulo $b$ with the precomputed $\mu$ value in $c$.  $a$ must be in the range
$0 \le a < b^2$.

\begin{alltt}
int main(void)
\{
   mp_int   a, b, c, mu;
   int      result;

   /* initialize a,b to desired values, mp_init mu, 
    * c and set c to 1...we want to compute a^3 mod b 
    */

   /* get mu value */
   if ((result = mp_reduce_setup(&mu, b)) != MP_OKAY) \{
      printf("Error getting mu.  \%s", 
             mp_error_to_string(result));
      return EXIT_FAILURE;
   \}

   /* square a to get c = a^2 */
   if ((result = mp_sqr(&a, &c)) != MP_OKAY) \{
      printf("Error squaring.  \%s", 
             mp_error_to_string(result));
      return EXIT_FAILURE;
   \}

   /* now reduce `c' modulo b */
   if ((result = mp_reduce(&c, &b, &mu)) != MP_OKAY) \{
      printf("Error reducing.  \%s", 
             mp_error_to_string(result));
      return EXIT_FAILURE;
   \}
   
   /* multiply a to get c = a^3 */
   if ((result = mp_mul(&a, &c, &c)) != MP_OKAY) \{
      printf("Error reducing.  \%s", 
             mp_error_to_string(result));
      return EXIT_FAILURE;
   \}

   /* now reduce `c' modulo b  */
   if ((result = mp_reduce(&c, &b, &mu)) != MP_OKAY) \{
      printf("Error reducing.  \%s", 
             mp_error_to_string(result));
      return EXIT_FAILURE;
   \}
  
   /* c now equals a^3 mod b */

   return EXIT_SUCCESS;
\}
\end{alltt} 

This program will calculate $a^3 \mbox{ mod }b$ if all the functions succeed.  

\section{Montgomery Reduction}

Montgomery is a specialized reduction algorithm for any odd moduli.  Like Barrett reduction a pre--computation
step is required.  This is accomplished with the following.

\index{mp\_montgomery\_setup}
\begin{alltt}
int mp_montgomery_setup(mp_int *a, mp_digit *mp);
\end{alltt}

For the given odd moduli $a$ the precomputation value is placed in $mp$.  The reduction is computed with the 
following.

\index{mp\_montgomery\_reduce}
\begin{alltt}
int mp_montgomery_reduce(mp_int *a, mp_int *m, mp_digit mp);
\end{alltt}
This reduces $a$ in place modulo $m$ with the pre--computed value $mp$.   $a$ must be in the range
$0 \le a < b^2$.

Montgomery reduction is faster than Barrett reduction for moduli smaller than the ``comba'' limit.  With the default
setup for instance, the limit is $127$ digits ($3556$--bits).   Note that this function is not limited to
$127$ digits just that it falls back to a baseline algorithm after that point.  

An important observation is that this reduction does not return $a \mbox{ mod }m$ but $aR^{-1} \mbox{ mod }m$ 
where $R = \beta^n$, $n$ is the n number of digits in $m$ and $\beta$ is radix used (default is $2^{28}$).  

To quickly calculate $R$ the following function was provided.

\index{mp\_montgomery\_calc\_normalization}
\begin{alltt}
int mp_montgomery_calc_normalization(mp_int *a, mp_int *b);
\end{alltt}
Which calculates $a = R$ for the odd moduli $b$ without using multiplication or division.  

The normal modus operandi for Montgomery reductions is to normalize the integers before entering the system.  For
example, to calculate $a^3 \mbox { mod }b$ using Montgomery reduction the value of $a$ can be normalized by
multiplying it by $R$.  Consider the following code snippet.

\begin{alltt}
int main(void)
\{
   mp_int   a, b, c, R;
   mp_digit mp;
   int      result;

   /* initialize a,b to desired values, 
    * mp_init R, c and set c to 1.... 
    */

   /* get normalization */
   if ((result = mp_montgomery_calc_normalization(&R, b)) != MP_OKAY) \{
      printf("Error getting norm.  \%s", 
             mp_error_to_string(result));
      return EXIT_FAILURE;
   \}

   /* get mp value */
   if ((result = mp_montgomery_setup(&c, &mp)) != MP_OKAY) \{
      printf("Error setting up montgomery.  \%s", 
             mp_error_to_string(result));
      return EXIT_FAILURE;
   \}

   /* normalize `a' so now a is equal to aR */
   if ((result = mp_mulmod(&a, &R, &b, &a)) != MP_OKAY) \{
      printf("Error computing aR.  \%s", 
             mp_error_to_string(result));
      return EXIT_FAILURE;
   \}

   /* square a to get c = a^2R^2 */
   if ((result = mp_sqr(&a, &c)) != MP_OKAY) \{
      printf("Error squaring.  \%s", 
             mp_error_to_string(result));
      return EXIT_FAILURE;
   \}

   /* now reduce `c' back down to c = a^2R^2 * R^-1 == a^2R */
   if ((result = mp_montgomery_reduce(&c, &b, mp)) != MP_OKAY) \{
      printf("Error reducing.  \%s", 
             mp_error_to_string(result));
      return EXIT_FAILURE;
   \}
   
   /* multiply a to get c = a^3R^2 */
   if ((result = mp_mul(&a, &c, &c)) != MP_OKAY) \{
      printf("Error reducing.  \%s", 
             mp_error_to_string(result));
      return EXIT_FAILURE;
   \}

   /* now reduce `c' back down to c = a^3R^2 * R^-1 == a^3R */
   if ((result = mp_montgomery_reduce(&c, &b, mp)) != MP_OKAY) \{
      printf("Error reducing.  \%s", 
             mp_error_to_string(result));
      return EXIT_FAILURE;
   \}
   
   /* now reduce (again) `c' back down to c = a^3R * R^-1 == a^3 */
   if ((result = mp_montgomery_reduce(&c, &b, mp)) != MP_OKAY) \{
      printf("Error reducing.  \%s", 
             mp_error_to_string(result));
      return EXIT_FAILURE;
   \}

   /* c now equals a^3 mod b */

   return EXIT_SUCCESS;
\}
\end{alltt} 

This particular example does not look too efficient but it demonstrates the point of the algorithm.  By 
normalizing the inputs the reduced results are always of the form $aR$ for some variable $a$.  This allows
a single final reduction to correct for the normalization and the fast reduction used within the algorithm.

For more details consider examining the file \textit{bn\_mp\_exptmod\_fast.c}.

\section{Restricted Dimminished Radix}

``Dimminished Radix'' reduction refers to reduction with respect to moduli that are ameniable to simple
digit shifting and small multiplications.  In this case the ``restricted'' variant refers to moduli of the
form $\beta^k - p$ for some $k \ge 0$ and $0 < p < \beta$ where $\beta$ is the radix (default to $2^{28}$).  

As in the case of Montgomery reduction there is a pre--computation phase required for a given modulus.

\index{mp\_dr\_setup}
\begin{alltt}
void mp_dr_setup(mp_int *a, mp_digit *d);
\end{alltt}

This computes the value required for the modulus $a$ and stores it in $d$.  This function cannot fail
and does not return any error codes.  After the pre--computation a reduction can be performed with the
following.

\index{mp\_dr\_reduce}
\begin{alltt}
int mp_dr_reduce(mp_int *a, mp_int *b, mp_digit mp);
\end{alltt}

This reduces $a$ in place modulo $b$ with the pre--computed value $mp$.  $b$ must be of a restricted
dimminished radix form and $a$ must be in the range $0 \le a < b^2$.  Dimminished radix reductions are 
much faster than both Barrett and Montgomery reductions as they have a much lower asymtotic running time.  

Since the moduli are restricted this algorithm is not particularly useful for something like Rabin, RSA or
BBS cryptographic purposes.  This reduction algorithm is useful for Diffie-Hellman and ECC where fixed
primes are acceptable.  

Note that unlike Montgomery reduction there is no normalization process.  The result of this function is
equal to the correct residue.

\section{Unrestricted Dimminshed Radix}

Unrestricted reductions work much like the restricted counterparts except in this case the moduli is of the 
form $2^k - p$ for $0 < p < \beta$.  In this sense the unrestricted reductions are more flexible as they 
can be applied to a wider range of numbers.  

\index{mp\_reduce\_2k\_setup}
\begin{alltt}
int mp_reduce_2k_setup(mp_int *a, mp_digit *d);
\end{alltt}

This will compute the required $d$ value for the given moduli $a$.  

\index{mp\_reduce\_2k}
\begin{alltt}
int mp_reduce_2k(mp_int *a, mp_int *n, mp_digit d);
\end{alltt}

This will reduce $a$ in place modulo $n$ with the pre--computed value $d$.  From my experience this routine is 
slower than mp\_dr\_reduce but faster for most moduli sizes than the Montgomery reduction.  

\chapter{Exponentiation}
\section{Single Digit Exponentiation}
\index{mp\_expt\_d}
\begin{alltt}
int mp_expt_d (mp_int * a, mp_digit b, mp_int * c)
\end{alltt}
This computes $c = a^b$ using a simple binary left-to-right algorithm.  It is faster than repeated multiplications by 
$a$ for all values of $b$ greater than three.  

\section{Modular Exponentiation}
\index{mp\_exptmod}
\begin{alltt}
int mp_exptmod (mp_int * G, mp_int * X, mp_int * P, mp_int * Y)
\end{alltt}
This computes $Y \equiv G^X \mbox{ (mod }P\mbox{)}$ using a variable width sliding window algorithm.  This function
will automatically detect the fastest modular reduction technique to use during the operation.  For negative values of 
$X$ the operation is performed as $Y \equiv (G^{-1} \mbox{ mod }P)^{\vert X \vert} \mbox{ (mod }P\mbox{)}$ provided that 
$gcd(G, P) = 1$.

This function is actually a shell around the two internal exponentiation functions.  This routine will automatically
detect when Barrett, Montgomery, Restricted and Unrestricted Dimminished Radix based exponentiation can be used.  Generally
moduli of the a ``restricted dimminished radix'' form lead to the fastest modular exponentiations.  Followed by Montgomery
and the other two algorithms.

\section{Root Finding}
\index{mp\_n\_root}
\begin{alltt}
int mp_n_root (mp_int * a, mp_digit b, mp_int * c)
\end{alltt}
This computes $c = a^{1/b}$ such that $c^b \le a$ and $(c+1)^b > a$.  The implementation of this function is not 
ideal for values of $b$ greater than three.  It will work but become very slow.  So unless you are working with very small
numbers (less than 1000 bits) I'd avoid $b > 3$ situations.  Will return a positive root only for even roots and return
a root with the sign of the input for odd roots.  For example, performing $4^{1/2}$ will return $2$ whereas $(-8)^{1/3}$ 
will return $-2$.  

This algorithm uses the ``Newton Approximation'' method and will converge on the correct root fairly quickly.  Since
the algorithm requires raising $a$ to the power of $b$ it is not ideal to attempt to find roots for large
values of $b$.  If particularly large roots are required then a factor method could be used instead.  For example,
$a^{1/16}$ is equivalent to $\left (a^{1/4} \right)^{1/4}$ or simply 
$\left ( \left ( \left ( a^{1/2} \right )^{1/2} \right )^{1/2} \right )^{1/2}$

\chapter{Prime Numbers}
\section{Trial Division}
\index{mp\_prime\_is\_divisible}
\begin{alltt}
int mp_prime_is_divisible (mp_int * a, int *result)
\end{alltt}
This will attempt to evenly divide $a$ by a list of primes\footnote{Default is the first 256 primes.} and store the 
outcome in ``result''.  That is if $result = 0$ then $a$ is not divisible by the primes, otherwise it is.  Note that 
if the function does not return \textbf{MP\_OKAY} the value in ``result'' should be considered undefined\footnote{Currently
the default is to set it to zero first.}.

\section{Fermat Test}
\index{mp\_prime\_fermat}
\begin{alltt}
int mp_prime_fermat (mp_int * a, mp_int * b, int *result)
\end{alltt}
Performs a Fermat primality test to the base $b$.  That is it computes $b^a \mbox{ mod }a$ and tests whether the value is
equal to $b$ or not.  If the values are equal then $a$ is probably prime and $result$ is set to one.  Otherwise $result$
is set to zero.

\section{Miller-Rabin Test}
\index{mp\_prime\_miller\_rabin}
\begin{alltt}
int mp_prime_miller_rabin (mp_int * a, mp_int * b, int *result)
\end{alltt}
Performs a Miller-Rabin test to the base $b$ of $a$.  This test is much stronger than the Fermat test and is very hard to
fool (besides with Carmichael numbers).  If $a$ passes the test (therefore is probably prime) $result$ is set to one.  
Otherwise $result$ is set to zero.  

Note that is suggested that you use the Miller-Rabin test instead of the Fermat test since all of the failures of 
Miller-Rabin are a subset of the failures of the Fermat test.

\subsection{Required Number of Tests}
Generally to ensure a number is very likely to be prime you have to perform the Miller-Rabin with at least a half-dozen
or so unique bases.  However, it has been proven that the probability of failure goes down as the size of the input goes up.
This is why a simple function has been provided to help out.

\index{mp\_prime\_rabin\_miller\_trials}
\begin{alltt}
int mp_prime_rabin_miller_trials(int size)
\end{alltt}
This returns the number of trials required for a $2^{-96}$ (or lower) probability of failure for a given ``size'' expressed
in bits.  This comes in handy specially since larger numbers are slower to test.  For example, a 512-bit number would
require ten tests whereas a 1024-bit number would only require four tests. 

You should always still perform a trial division before a Miller-Rabin test though.

\section{Primality Testing}
\index{mp\_prime\_is\_prime}
\begin{alltt}
int mp_prime_is_prime (mp_int * a, int t, int *result)
\end{alltt}
This will perform a trial division followed by $t$ rounds of Miller-Rabin tests on $a$ and store the result in $result$.  
If $a$ passes all of the tests $result$ is set to one, otherwise it is set to zero.  Note that $t$ is bounded by 
$1 \le t < PRIME\_SIZE$ where $PRIME\_SIZE$ is the number of primes in the prime number table (by default this is $256$).

\section{Next Prime}
\index{mp\_prime\_next\_prime}
\begin{alltt}
int mp_prime_next_prime(mp_int *a, int t, int bbs_style)
\end{alltt}
This finds the next prime after $a$ that passes mp\_prime\_is\_prime() with $t$ tests.  Set $bbs\_style$ to one if you 
want only the next prime congruent to $3 \mbox{ mod } 4$, otherwise set it to zero to find any next prime.  

\section{Random Primes}
\index{mp\_prime\_random}
\begin{alltt}
int mp_prime_random(mp_int *a, int t, int size, int bbs, 
                    ltm_prime_callback cb, void *dat)
\end{alltt}
This will find a prime greater than $256^{size}$ which can be ``bbs\_style'' or not depending on $bbs$ and must pass
$t$ rounds of tests.  The ``ltm\_prime\_callback'' is a typedef for 

\begin{alltt}
typedef int ltm_prime_callback(unsigned char *dst, int len, void *dat);
\end{alltt}

Which is a function that must read $len$ bytes (and return the amount stored) into $dst$.  The $dat$ variable is simply
copied from the original input.  It can be used to pass RNG context data to the callback.  The function 
mp\_prime\_random() is more suitable for generating primes which must be secret (as in the case of RSA) since there 
is no skew on the least significant bits.

\textit{Note:}  As of v0.30 of the LibTomMath library this function has been deprecated.  It is still available
but users are encouraged to use the new mp\_prime\_random\_ex() function instead.

\subsection{Extended Generation}
\index{mp\_prime\_random\_ex}
\begin{alltt}
int mp_prime_random_ex(mp_int *a,    int t, 
                       int     size, int flags, 
                       ltm_prime_callback cb, void *dat);
\end{alltt}
This will generate a prime in $a$ using $t$ tests of the primality testing algorithms.  The variable $size$
specifies the bit length of the prime desired.  The variable $flags$ specifies one of several options available
(see fig. \ref{fig:primeopts}) which can be OR'ed together.  The callback parameters are used as in 
mp\_prime\_random().

\begin{figure}[here]
\begin{center}
\begin{small}
\begin{tabular}{|r|l|}
\hline \textbf{Flag}         & \textbf{Meaning} \\
\hline LTM\_PRIME\_BBS       & Make the prime congruent to $3$ modulo $4$ \\
\hline LTM\_PRIME\_SAFE      & Make a prime $p$ such that $(p - 1)/2$ is also prime. \\
                             & This option implies LTM\_PRIME\_BBS as well. \\
\hline LTM\_PRIME\_2MSB\_OFF & Makes sure that the bit adjacent to the most significant bit \\
                             & Is forced to zero.  \\
\hline LTM\_PRIME\_2MSB\_ON  & Makes sure that the bit adjacent to the most significant bit \\
                             & Is forced to one. \\
\hline
\end{tabular}
\end{small}
\end{center}
\caption{Primality Generation Options}
\label{fig:primeopts}
\end{figure}

\chapter{Input and Output}
\section{ASCII Conversions}
\subsection{To ASCII}
\index{mp\_toradix}
\begin{alltt}
int mp_toradix (mp_int * a, char *str, int radix);
\end{alltt}
This still store $a$ in ``str'' as a base-``radix'' string of ASCII chars.  This function appends a NUL character
to terminate the string.  Valid values of ``radix'' line in the range $[2, 64]$.  To determine the size (exact) required
by the conversion before storing any data use the following function.

\index{mp\_radix\_size}
\begin{alltt}
int mp_radix_size (mp_int * a, int radix, int *size)
\end{alltt}
This stores in ``size'' the number of characters (including space for the NUL terminator) required.  Upon error this 
function returns an error code and ``size'' will be zero.  

\subsection{From ASCII}
\index{mp\_read\_radix}
\begin{alltt}
int mp_read_radix (mp_int * a, char *str, int radix);
\end{alltt}
This will read the base-``radix'' NUL terminated string from ``str'' into $a$.  It will stop reading when it reads a
character it does not recognize (which happens to include th NUL char... imagine that...).  A single leading $-$ sign
can be used to denote a negative number.

\section{Binary Conversions}

Converting an mp\_int to and from binary is another keen idea.

\index{mp\_unsigned\_bin\_size}
\begin{alltt}
int mp_unsigned_bin_size(mp_int *a);
\end{alltt}

This will return the number of bytes (octets) required to store the unsigned copy of the integer $a$.

\index{mp\_to\_unsigned\_bin}
\begin{alltt}
int mp_to_unsigned_bin(mp_int *a, unsigned char *b);
\end{alltt}
This will store $a$ into the buffer $b$ in big--endian format.  Fortunately this is exactly what DER (or is it ASN?)
requires.  It does not store the sign of the integer.

\index{mp\_read\_unsigned\_bin}
\begin{alltt}
int mp_read_unsigned_bin(mp_int *a, unsigned char *b, int c);
\end{alltt}
This will read in an unsigned big--endian array of bytes (octets) from $b$ of length $c$ into $a$.  The resulting
integer $a$ will always be positive.

For those who acknowledge the existence of negative numbers (heretic!) there are ``signed'' versions of the
previous functions.

\begin{alltt}
int mp_signed_bin_size(mp_int *a);
int mp_read_signed_bin(mp_int *a, unsigned char *b, int c);
int mp_to_signed_bin(mp_int *a, unsigned char *b);
\end{alltt}
They operate essentially the same as the unsigned copies except they prefix the data with zero or non--zero
byte depending on the sign.  If the sign is zpos (e.g. not negative) the prefix is zero, otherwise the prefix
is non--zero.  

\chapter{Algebraic Functions}
\section{Extended Euclidean Algorithm}
\index{mp\_exteuclid}
\begin{alltt}
int mp_exteuclid(mp_int *a, mp_int *b, 
                 mp_int *U1, mp_int *U2, mp_int *U3);
\end{alltt}

This finds the triple U1/U2/U3 using the Extended Euclidean algorithm such that the following equation holds.

\begin{equation}
a \cdot U1 + b \cdot U2 = U3
\end{equation}

Any of the U1/U2/U3 paramters can be set to \textbf{NULL} if they are not desired.  

\section{Greatest Common Divisor}
\index{mp\_gcd}
\begin{alltt}
int mp_gcd (mp_int * a, mp_int * b, mp_int * c)
\end{alltt}
This will compute the greatest common divisor of $a$ and $b$ and store it in $c$.

\section{Least Common Multiple}
\index{mp\_lcm}
\begin{alltt}
int mp_lcm (mp_int * a, mp_int * b, mp_int * c)
\end{alltt}
This will compute the least common multiple of $a$ and $b$ and store it in $c$.

\section{Jacobi Symbol}
\index{mp\_jacobi}
\begin{alltt}
int mp_jacobi (mp_int * a, mp_int * p, int *c)
\end{alltt}
This will compute the Jacobi symbol for $a$ with respect to $p$.  If $p$ is prime this essentially computes the Legendre
symbol.  The result is stored in $c$ and can take on one of three values $\lbrace -1, 0, 1 \rbrace$.  If $p$ is prime
then the result will be $-1$ when $a$ is not a quadratic residue modulo $p$.  The result will be $0$ if $a$ divides $p$
and the result will be $1$ if $a$ is a quadratic residue modulo $p$.  

\section{Modular Inverse}
\index{mp\_invmod}
\begin{alltt}
int mp_invmod (mp_int * a, mp_int * b, mp_int * c)
\end{alltt}
Computes the multiplicative inverse of $a$ modulo $b$ and stores the result in $c$ such that $ac \equiv 1 \mbox{ (mod }b\mbox{)}$.

\section{Single Digit Functions}

For those using small numbers (\textit{snicker snicker}) there are several ``helper'' functions

\index{mp\_add\_d} \index{mp\_sub\_d} \index{mp\_mul\_d} \index{mp\_div\_d} \index{mp\_mod\_d}
\begin{alltt}
int mp_add_d(mp_int *a, mp_digit b, mp_int *c);
int mp_sub_d(mp_int *a, mp_digit b, mp_int *c);
int mp_mul_d(mp_int *a, mp_digit b, mp_int *c);
int mp_div_d(mp_int *a, mp_digit b, mp_int *c, mp_digit *d);
int mp_mod_d(mp_int *a, mp_digit b, mp_digit *c);
\end{alltt}

These work like the full mp\_int capable variants except the second parameter $b$ is a mp\_digit.  These
functions fairly handy if you have to work with relatively small numbers since you will not have to allocate
an entire mp\_int to store a number like $1$ or $2$.

\input{bn.ind}

\end{document}


\end{document}


\end{document}


\end{document}
